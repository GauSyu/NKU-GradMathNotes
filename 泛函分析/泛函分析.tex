%\documentclass[UTF8]{ctexart}
%基本文件设置,Ctex中文+UTF8编码。需要XeTex编译。
%用这个生成中文索引USE:cctmkind -o 泛函分析.ind -s cct.ist -C pinyin 泛函分析.idx
%备选USE:cctmkind -o 泛函分析.ind -s cct.ist -C stroke 泛函分析.idx
%备选USE:cctmkind -o 泛函分析.ind -s cct.ist -C mixed 泛函分析.idx
%------------------文件设置-----------------
\documentclass[winfonts,UTF8,c5size,a4paper,fancyhdr,hyperref,titlepage,nocap]{ctexart}
%winfonts=使用Windows 的字体设置,默认为六种中易字体:宋体、仿宋、黑体、楷体、隶书、幼圆(在使用XeTEX时只有前四种)。该选项的结果将和老版本ctex宏包完全一致。这是默认设置。
%cs4size=使用小四字号为缺省字体大小。
%a4paper=页面大小为A4。
%fancyhdr=保持和fancyhdr宏包的兼容性。该选项将使得fancyhdr宏包被自动调用。
%hyperref=自动判断调用fancyhdr宏包的正确参数以避免产生乱码。该选项将使得hyperref宏包被自动调用。
%------------------整体格式-----------------
%\usepackage[xetex]{hyperref}%让生成的文章目录有链接和标签。
%\usepackage{ifthen}
%条件分支
%\ifthenelse{test}{then_text}{else_text}%
%测试test,如果为真,执行then_text语句;如果为假,执行else_text。
%\whiledo{test}{do_text}
%测试test,为真则执行do_text语句,为假则不执行。
\usepackage{amsmath,amsthm,amssymb,amsfonts,bm}         %AMS数学包。
\usepackage[all]{xy}                                    %交换图表。
\usepackage{epic}															%增强绘图。
\usepackage{graphicx}														%图像增强
\usepackage{mathrsfs,mathtools}
\usepackage{cite}
\usepackage{longtable}
\usepackage{enumerate}
\usepackage{upgreek}%直立的小写希腊字母
\usepackage[text={170mm,250mm},centering]{geometry}   %调整页面布局
%\CTEXsetup[number={},indent={-1ex}]{section}
\usepackage{mathrsfs}
\usepackage{makeidx}
\usepackage{color}
\CTEXoptions[today=small]
\renewcommand{\labelitemii}{\textbullet}
\renewcommand{\labelitemiii}{\textbullet}
\renewcommand{\labelitemiv}{\textbullet}
%-----------------定理命题-----------------
\newtheoremstyle{question}{1.5ex plus 1ex minus .2ex}{1.5ex plus 1ex minus .2ex}
     {\large\itshape}{}{\songti\bfseries}{}{1em}{}
\theoremstyle{question}
\newtheorem{Qut}{题目}
\theoremstyle{theorem}
\newtheorem{thm}{定理}
\newtheorem{cor}[thm]{推论}
\newtheorem{lem}[thm]{引理}
\newtheorem{prop}[thm]{命题}
\newtheorem{axiom}{公理}
\newtheorem{exa}[thm]{例}
\theoremstyle{definition}
\newtheorem{defn}[thm]{定义}
\theoremstyle{remark}
\newtheorem*{rem}{注}
\newtheorem*{note}{约定}
\numberwithin{equation}{subsection}
\renewcommand{\proofname}{\bf  证明}
% THEOREM Environments --------------------------------------------------------
%\newtheorem{thm}{Theorem}[subsection]
%\newtheorem{cor}[thm]{Corollary}
%\newtheorem{lem}[thm]{Lemma}
%\newtheorem{prop}[thm]{Proposition}
%\theoremstyle{definition}
%\newtheorem{defn}[thm]{Definition}
%\theoremstyle{remark}
%\newtheorem{rem}[thm]{Remark}
%\numberwithin{equation}{subsection}
%-----------------页眉页脚-----------------
%\usepackage{fancyhdr}                       %页眉页脚包
%\pagestyle{fancy}                           %设置页眉
%-----------------制定命令-----------------
% MATH ------------------------------------------------------------------------
\DeclareMathOperator{\RE}{Re}
\DeclareMathOperator{\IM}{Im}
\DeclareMathOperator{\ess}{ess}
\newcommand{\eps}{\varepsilon}
\newcommand{\To}{\longrightarrow}
\newcommand{\h}{\mathcal{H}}
\newcommand{\s}{\mathcal{S}}
\newcommand{\A}{\mathcal{A}}
\newcommand{\J}{\mathcal{J}}
%\newcommand{\M}{\mathcal{M}}
\newcommand{\W}{\mathcal{W}}
\newcommand{\X}{\mathcal{X}}
\newcommand{\BOP}{\mathbf{B}}
\newcommand{\BH}{\mathbf{B}(\mathcal{H})}
\newcommand{\KH}{\mathcal{K}(\mathcal{H})}
\newcommand{\zz}{\mathbb{Z}}
\newcommand{\N}{\mathbb{N}}
\newcommand{\Q}{\mathbb{Q}}
\newcommand{\Real}{\mathbb{R}}
\newcommand{\Complex}{\mathbb{C}}
\newcommand{\Field}{\mathbb{F}}
\newcommand{\K}{\mathbb{K}}
\newcommand{\RPlus}{\Real^{+}}
\newcommand{\pp}{\mathcal{P}}
\newcommand{\Polar}{\mathcal{P}_{\s}}
\newcommand{\Poly}{\mathcal{P}(E)}
\newcommand{\EssD}{\mathcal{D}}
\newcommand{\Lom}{\mathcal{L}}
\newcommand{\States}{\mathcal{T}}
\newcommand{\abs}[1]{\left\vert#1\right\vert}
\newcommand{\set}[1]{\left\{#1\right\}}
\newcommand{\seq}[1]{\left<#1\right>}
\newcommand{\norm}[1]{\left\Vert#1\right\Vert}
\newcommand{\essnorm}[1]{\norm{#1}_{\ess}}
\newcommand{\ob}{\mathrm{ob}}
\newcommand{\g}{\mathfrak{g}}
\newcommand{\base}[1]{\mathrm{base}(\mathscr{#1})}
\newcommand{\bun}[1]{\mathrm{bun}\mathcal{#1}}
\newcommand{\fib}{\mathrm{fib}}
\newcommand{\RP}{\mathbb{R}\mathbf{P}}
\newcommand{\red}{\color{red}}
\newcommand{\dd}{\operatorname{d}}
\newcommand{\dt}{\frac{\operatorname{d}}{\operatorname{d}t}}
\newcommand{\pfrac}[2]{\frac{\partial{#1}}{\partial{#2}}}
\newcommand{\px}[1]{\left.\pfrac{}{x^{#1}}\right|_p}
\newcommand{\py}[1]{\left.\pfrac{}{y^{#1}}\right|_q}
\newcommand{\rank}{\operatorname{rank}}
\newcommand{\spa}{\operatorname{span}}
\newcommand{\tr}{\operatorname{tr}}
\newcommand{\Xf}[1]{\mathfrak{X}(#1)}
\newcommand{\local}[2]{\left.{#1}\right|_{#2}}%Local #1 at #2
\newcommand{\localt}[1]{\local{#1}{t=0}}%Local #1 at  t=0
\newcommand{\localp}[1]{\local{#1}{p}}%Local #1 at  p
%==================================================
\newcommand{\defen}{\mathop{\iff}\limits^{\mathclap{\text{\scriptsize def}}}}%s上标箭头
\newcommand{\markar}[1]{\stackrel{{#1}}{\longrightarrow}}%带标号箭头(定长,与\xleftarrow不同)
\newcommand{\defeq}{\stackrel{{\mathrm{def}}}{=}}
\newcommand{\mexseq}[2]{\cdots \longrightarrow {#1}_{n-1} \markar{{#2}_{n-1}} {#1}_n \markar{{#2}_n} {#1}_{n+1} \longrightarrow \cdots}
  %middle exact sequence
\newcommand{\schain}[2]{{#1}_0 \markar{{#2}_0} {#1}_1 \markar{{#2}_1} {#1}_2 \longrightarrow \cdots}
  %起始链
\newcommand{\fchain}[2]{\cdots \longrightarrow {#1}_{n-2} \markar{{#2}_{n-2}} {#1}_{n-1} \markar{{#2}_{n-1}} {#1}_n}
  %结束链
\newcommand{\lchain}[2]{{#1}_0 \markar{{#2}_0} {#1}_1 \markar{{#2}_1} {#1}_2 \longrightarrow \cdots \longrightarrow {#1}_{n-2} \markar{{#2}_{n-2}} {#1}_{n-1} \markar{{#2}_{n-1}} {#1}_n}
  %完整长链
\newcommand{\mapdes}[5]
  {
    \begin{align*}
      #1\colon\  #2 & \longrightarrow  #3 \\
            #4 & \mapsto  #5
    \end{align*}
    %多于两个对齐点,align会出问题。换用alignat。
  }
\newcommand{\id}{\operatorname{id}}
\newcommand{\supp}{\operatorname{supp}}
\newcommand{\im}{\operatorname{im}}
\newcommand{\Inv}{\operatorname{Inv}}
%------------------标题作者-----------------
\title{泛函分析复习}
\author{Gau Syu}
%\setcounter{secnumdepth}{1}
\makeindex
\begin{document}
\maketitle
\pagestyle{plain}
\section*{Preface}
\addcontentsline{toc}{section}{Perface}
这是2012年下半年南开大学数学科学学院研究生课程“泛函分析”的期末复习五道题目。该课程由安桂梅老师讲授。
\tableofcontents
\newpage
\section{半范数与局部凸空间}
\begin{Qut}
设$\{p_{\alpha}\}_{\alpha\in\Lambda}$是$X$上分离点的半范数族,则由其生成的拓扑是Hausdorff的。
\end{Qut}
\begin{defn}
  $X$是域$\K$($\Real$或$\Complex$)上的线性空间,$X$上的实值函数$p$满足
\begin{enumerate}[1)]
    \setlength{\itemindent}{2ex}
    \item (次可加性)$p(x + y) \leqslant p(x) + p(y), \forall x, y\in X$;
    \item (绝对齐性)$p(\lambda x)=|\lambda|p(x), \forall\lambda\in\K, x\in X$
\end{enumerate}
则称$p$是$X$上的一个\emph{\red 半范数}\index{半范数 }。
\end{defn}

\begin{prop}
  设$p$是线性空间$X$上的半范数,则
\begin{enumerate}[1)]
    \setlength{\itemindent}{2ex}
    \item $p(0)=0$;
    \item $|p(x)-p(y)|\leqslant p(x-y)$;
    \item $p(x)\geqslant 0$;
    \item $\ker p$是$X$的子空间。
\end{enumerate}
\end{prop}
\begin{proof}
\begin{enumerate}[1)]
    \setlength{\itemindent}{2ex}
    \item $p(0)=p(0x)=|0|p(x)=0$;
    \item 因为$p(x)=p(x-y+y)\leqslant p(x-y)+p(y)$,且$p(y-x)=p(x-y)$;
    \item $p(x)\geqslant|p(x)-P(0)|\geqslant0$;
    \item $\forall \alpha,\beta\in\K, x,y\in\ker p$,
             \begin{equation*}
             p(\alpha x+\beta y)\leqslant p(\alpha x)+p(\beta y)=|\alpha|p(x)+|\beta|p(y)=0+0=0
             \end{equation*}
\end{enumerate}
\end{proof}
\begin{rem}
可见半范数就是范数定义中去掉$p(x)=0\Rightarrow x=0$这一条。
\end{rem}

\begin{defn}
  对于线性空间$X$中的集合$M$,
\begin{enumerate}[1)]
    \setlength{\itemindent}{2ex}
    \item 若$\forall x,y\in M, 0\leqslant\alpha\leqslant1, \alpha x+(1-\alpha)y\in M$,则称$M$是\emph{\red 凸的}\index{凸集};
    \item 若$\forall x\in M,|\lambda|\leqslant1, \lambda x\in M$,则称$M$是\emph{\red 平衡的}\index{平衡集};
    \item 若$M=-M$,则称$M$是\emph{\red 对称的}\index{对称集};
    \item 若$\forall x\in X$,存在$\varepsilon>0$,使得当$0<|\alpha|\leqslant\varepsilon$时,$\alpha x\in M$,则称$M$是\emph{\red 吸收的}\index{吸收集}。
\end{enumerate}
\end{defn}

\begin{prop}\label{M1}
  对任给的半范数$p$和实数$r>0$,
\begin{equation*}
M=\{x\in X\mid p(x)<r\},
M'=\{x\in X\mid p(x)<= r\}
\end{equation*}
均是平衡的、吸收的凸集。
\end{prop}
\begin{proof}
  先证$M$是\emph{凸集}:
  $\forall x,y\in M, 0\leqslant\alpha\leqslant1$,
  \begin{equation*}
  p(\alpha x+(1-\alpha)y)\leqslant|\alpha|p(x)+|1-\alpha|p(y)<\alpha r+(1-\alpha)r=r
  \end{equation*}
  所以$\alpha x+(1-\alpha)y\in M$。

  再证$M$是\emph{平衡集}:
  $\forall x\in M,|\lambda|\leqslant1$,
  \begin{equation*}
  p(\lambda x)=|\lambda|p(x)\leqslant p(x)<r
  \end{equation*}
  所以$\lambda x\in M$。

  最后证$M$是\emph{吸收集}:
  $\forall x\in X$,不妨设$p(x)>0$,则
  \begin{equation*}
  p(\frac{rx}{2p(x)})=\frac{r}{2p(x)}p(x)=\frac{r}{2}<r
  \end{equation*}
  即$\frac{r}{2p(x)}x\in M$。

  对$M'$的证明类似。
\end{proof}

\begin{defn}
  设$K$是线性空间$X$中的子集,则
  \begin{equation*}
    \mu_K(x)\defeq\inf\left\{\alpha\middle|\alpha>0,\frac{1}{\alpha}x\in K\right\}
  \end{equation*}
  称为$K$的\emph{\red Minkowski泛函}\index{Minkowski泛函}。
\end{defn}
由定义可知$K\subset L\Rightarrow\mu_K\geqslant\mu_L$。

\begin{prop}
  线性空间$X$中的平衡吸收凸集$K$的Minkowski泛函$\mu_K$是$X$上的半范数。
\end{prop}

\begin{defn}
  一个线性空间$X$上赋予一个使得加法和数乘运算连续的拓扑$\tau$,则称$(X,\tau)$为\emph{\red 拓扑线性空间}\index{拓扑线性空间}。
\end{defn}
\begin{rem}
拓扑线性空间的拓扑基可由$0$点的邻域基平移得到,故称$0$点的邻域基称为\emph{\red 局部基}\index{局部基}。
\end{rem}

\begin{rem}
设$X$是一个拓扑线性空间,$U$是$0$点的一个\emph{凸}邻域,则$U$包含一个$0$点的\emph{平衡的}、\emph{吸收的}\emph{凸}邻域。
\end{rem}

\begin{defn}
  设$(X,\tau)$是拓扑线性空间,如果$\tau$的局部基全由凸集组成,则称为\emph{\red 局部凸拓扑线性空间}\index{局部凸拓扑线性空间},简称\emph{\red 局部凸空间}\index{局部凸空间}。
\end{defn}

\begin{thm}
  设$P$是线性空间$X$上的半范数族,对每个$p\in P$及自然数$n$,记
\begin{equation*}
V(p,n)\defeq\left\{x\in X\middle|p(x)<\frac{1}{n}\right\}
\end{equation*}
其所有有限交构成的集合记为$\mathscr{B}$,则$\mathscr{B}$是平衡的凸的局部基。这样就由$P$诱导出一个$X$上的局部凸拓扑,称为由半范数族$P${\red 生成}\index{半范数诱导的拓扑}的拓扑。
\end{thm}
\begin{proof}
  首先,由命题\ref{M1},每个$V(p,n)$都是$0$点的平衡的凸邻域,进一步地,它们的有限交也是$0$点的平衡的凸邻域。

  然后证明加法运算连续:注意到加法运算在$0$处是连续的当且仅当对于任何$0$的邻域$U$存在另一个$0$的邻域$V$使得$V + V$ 被包含在$U$中。

  设$U$是$0$点的一个邻域,则存在$p_i\in P,n_i\in\N,i=1,2,\cdots,k$,使得
  \begin{equation*}
  U\supset V(p_1,n_1)\cap\cdots\cap V(p_k,n_k)
  \end{equation*}
  令$V=V(p_1,2n_1)\cap\cdots\cap V(p_k,2n_k)$,则由次可加性得$V+V\subset U$。

  最后证明乘法运算连续:即$\forall x\in X,\alpha\in\K, U\in N(0)$,存在$\varepsilon$和$V\in N(0)$,当$|\beta-\alpha|<\varepsilon,y-x\in V$时,$\beta y-\alpha x\in U$。

  取$V$同上,由于$V$平衡,所以当$s\geqslant1,x\in sV, t=s(1+|\alpha|s)^{-1},|\beta-\alpha|<s^{-1},y\in x+tV\in N(x)$时就有
  \begin{align*}
    \beta y-\alpha x &=\beta(y-x)+(\beta-\alpha)x \\
                             &\in\beta tV+s^{-1}sV\\
                             &\subset V+V\subset U
  \end{align*}
\end{proof}

\begin{prop}
  设$(X,\tau)$是局部凸空间,则必有一族$X$上的半范数$P$使得$V(p,n)$是$\tau$的局部基。
\end{prop}
\begin{proof}
  设$U$是$0$点的一个凸邻域,则$U$包含一个$0$点的平衡的吸收的凸邻域$V$,令$\mu_V$为其Minkowski泛函,则$\mu_V$是半范数,所有这样的半范数构成的半范数族生成该局部凸拓扑。
\end{proof}

\begin{defn}
  设$\{p_{\alpha}\}_{\alpha\in\Lambda}$是线性空间$X$上的一族半范数,如果
  \begin{equation*}
  \forall\alpha\in\Lambda, p_{\alpha}(x)=0\Longrightarrow x=0
  \end{equation*}
  则称$\{p_{\alpha}\}_{\alpha\in\Lambda}$\emph{\red 分离}\index{分离点 }$X$ 中的点。
\end{defn}

\begin{thm}
  由分离点的半范数族决定的的局部凸空间是Hausdorff的,反之亦然。
\end{thm}
\begin{proof}
  首先证明$\forall x\neq0\in X$,$x$和$0$可由开集分离:

  事实上,因为$x\neq0$,由于$P$分离点,故存在$p\in P$使得$p(x)>0$,取自然数$n$使得$np(x)>2$,则
  \begin{equation*}
  V(p,n)\cap(x+V(p,n))=\left\{y\in X\middle| p(y)<\frac{1}{n}, p(y-x)<\frac{1}{n}\right\}
  \end{equation*}
  然而
\begin{equation*}
    \begin{cases}
    p(y)<\frac{1}{n}\\
    p(y-x)<\frac{1}{n}
    \end{cases}
    \Longrightarrow
    p(x)\leqslant p(y)+p(y-x)<2n
\end{equation*}
  与$np(x)>2$矛盾,故$V(p,n)\cap(x+V(p,n))=\varnothing$。

  对于$x\neq y$,若已有开集$U,V$分离$0$和$x-y$,则
  \begin{align*}
  \begin{cases}
  0\in U\\
  x-y\in V
  \end{cases}
  &\Longrightarrow
  \begin{cases}
  x\in y+V\\
  y\in y+U
  \end{cases}\\
  U\cap V=\varnothing&\Longrightarrow(y+U)\cap(y+V)=\varnothing
  \end{align*}

  反之,若$X$是Hausdorff的局部凸空间,则取半范数族$P$为其局部基的Minkowski泛函。$\forall x\neq0$,由Hausdorff性,存在$0$点邻域$U$,使得$x\notin U$。进一步地,可取为邻域基中的元$V$,使得$x\notin V$,于是$\mu_V\neq 0$,这就证明了半范数族$P$分离$X$中的点。
\end{proof}


\section{值域定理}
\begin{Qut}
设$T$为从赋范线性空间$E$到$E_1$内的线性算子,且有$\overline{\mathcal{D}(T)}=E$,那么,为了$T^{-1}$存在且连续(即$T\in\Inv(\mathcal{B}(\mathcal{D}(T),E_1))$)必须且只须$\mathcal{W}(T^{\ast})=E^{\ast}$,即$T^{\ast}$ 满值域。
\end{Qut}
\begin{proof}
\begin{enumerate}[1)]
    \setlength{\itemindent}{2ex}
    \item ``$\Longrightarrow$'':
    如果$T^{-1}$存在且连续,须证对任意的$f\in E^{\ast}$,存在$g\in E_1^{\ast}$,使得$T^{\ast}(g)=f$。

  首先定义$\mathcal{W}(T)\subset E_1$上的有界线性泛函
  \begin{equation*}
  g_0(y)\defeq f(T^{-1}(y)), y\forall y\in\mathcal{W}(T)
  \end{equation*}
  从而由Hahn-Banach定理,可得到$g_0$的一个保范延拓$g\in g\in E_1^{\ast}$。又由于
  \begin{equation*}
  g(T(x))=g_0(T(x))=f(T^{-1}(T(x)))=f(x), \forall x\in \mathcal{D}(T)
  \end{equation*}
  故$g\in\mathcal{D}(T^{\ast}),T^{\ast}(g)=f$。
  \item ``$\Longleftarrow$'':
  为证$T^{-1}$存在,只须证$T$是$1-1$对应的。

若$T(x)=0$,则任取$f\in E^{\ast}$,由假设$\mathcal{W}(T^{\ast})=E^{\ast}$,故存在$g\in\mathcal{D}(T^{\ast})$,使得$f=T^{\ast}(g)$,于是
\begin{equation*}
f(x)=(T^{\ast}(g))(x)=g(T(x))=g(0)=0
\end{equation*}
故$x=0$。

为证$T^{-1}$是连续的,只须证$T^{-1}$是有界线性算子:

$\forall y\in\mathcal{W}(T), ||y||\leqslant1$,任取$f\in E^{\ast}$,由假设$\mathcal{W}(T^{\ast})=E^{\ast}$,故存在$g\in E_1^{\ast}$,使得$f=T^{\ast}(g)$。由于
\begin{align*}
|f(T^{-1}(y))|&=|(T^{\ast}(g))(T^{-1}(y))|\\
              &=|g(TT^{-1}(y))|\\
              &=|g(y)|\leqslant||g||
\end{align*}

故由共鸣定理\footnote{注意到共轭空间$E^{\ast}$一定是Banach空间,将$T^{-1}(y)$视为$E^{\ast}$上的连续线性泛函。}得
\begin{equation*}
\sup\{||T^{-1}(y)||\mid||y||\leqslant1, y\in\mathcal{W}(T)\}<\infty
\end{equation*}
\end{enumerate}
\end{proof}

\begin{Qut}
设$T$是从赋范线性空间$E$到$E_1$内的线性算子,且有$\overline{\mathcal{D}(T)}=E$,那么,为了$(T^{\ast})^{-1}$存在,必须且只须有$\overline{\mathcal{W}(T)}=E_1$。
\end{Qut}
\begin{proof}
\begin{enumerate}[1)]
    \setlength{\itemindent}{2ex}
    \item ``$\Longrightarrow$'':
    反之,假如存在$y_1\in E_1$且$y_1\notin\overline{\mathcal{W}(T)}$,则注意到$\overline{\mathcal{W}(T)}$是$E_1$内的一个闭线性子空间,因此由分割性定理可知,存在$g_1\in E_1^{\ast}$,使得
    \begin{equation*}
    g_1(y_1)=1, g_1(y)=0, \forall y\in\overline{\mathcal{W}(T)}
    \end{equation*}
    由此得
    \begin{equation*}
    g_1(T(x))=0,\forall x\in\mathcal{D}(T)
    \end{equation*}
    于是
    \begin{equation*}
    g_1\in\mathcal{D}(T^{\ast}), T^{\ast}(g_1)=0
    \end{equation*}
    由假设,$T^{\ast}$存在线性逆算子,故$T^{\ast}$是$1-1$对应的,于是$g_1=0$,与前面取法矛盾。
    \item ``$\Longleftarrow$'':
    只须证$T^{\ast}$是$1-1$对应的。

    如果存在泛函$g_0\in E_1^{\ast}$,使得
    \begin{equation*}
    T^{\ast}(g_0)=0
    \end{equation*}
    则
    \begin{equation*}
    g_0(T(x))=(T^{\ast}(g_0))(x)=0, \forall x\in\mathcal{D}(T)
    \end{equation*}
    即$\local{g_0}{\mathcal{W}(T)}=0$。由假设$\overline{\mathcal{W}(T)}=E_1$以及$g_0$在$E_1$上连续,故$g_0=0$。
\end{enumerate}
\end{proof}

\begin{Qut}
设$T$是从赋范线性空间$E$到$E_1$内的线性算子,且有$\overline{\mathcal{D}(T)}=E$,那么,当$E_1$为``第二纲''赋范线性空间且有$\mathcal{W}(T)=E_1$时,$(T^{\ast})^{-1}$存在且连续。
\end{Qut}
\begin{proof}
  首先,由前一题可知$(T^{\ast})^{-1}$存在。下证$(T^{\ast})^{-1}$有界:

若不然,则由于$(T^{\ast})^{-1}$非有界,故存在列$\{g_n\}\subset E_1^{\ast}$,使得
\begin{equation*}
||g_n||=1,||T^{\ast}(g_n)||\to 0(n\to\infty)
\end{equation*}
于是,可定义:
\begin{equation*}
g'_n=\frac{g_n}{||T^{\ast}(g_n)||}(n\in\N)
\end{equation*}
此时
\begin{equation}\label{4.1}
\lim_{n\to\infty}||g'_n||=\lim_{n\to\infty}\frac{||g_n||}{||T^{\ast}(g_n)||}=\lim_{n\to\infty}\frac{1}{||T^{\ast}(g_n)||}=\infty
\end{equation}

另外,由定理假设$E_1=\mathcal{W}(T)$知,$\forall y\in E_1$,存在$x\in E$使得$y=T(x)$。于是
\begin{align*}
|g'_n(y)|&=|g'_n(T(x))|=|(T^{\ast}g'_n)(x)|\\
         &\leqslant||T^{\ast}g'_n||||x||=||x||<\infty, \forall y\in E_1
\end{align*}

由$E_1$第二纲以及推广的共鸣定理,得$\{||g'_n||\}$是有界数列,与式\ref{4.1}矛盾。
\end{proof}

\begin{defn}
  从赋范线性空间$E$到$E_1$内的线性算子$T$称为\emph{\red 有界的}\index{有界线性算子},如果存在$\lambda>0$使得
  \begin{equation*}
  ||Tx||\leqslant\lambda||x||, \forall x\in E
  \end{equation*}
\end{defn}
\begin{rem}
  等价的条件是将有界集映到有界集。
\end{rem}
\begin{rem}
  反之,若算子$T$非有界,则存在$E$中的点列$\{x_n\}$,使得
  \begin{equation*}
    ||Tx_n||=1,||x_n||\to 0(n\to\infty)
  \end{equation*}
\end{rem}

\begin{defn}
  从赋范线性空间$E$到$E_1$内的全体有界线性算子构成的赋范空间记为$\mathcal{B}(E,E_1)$。特别地,$\mathcal{B}(E,\K)$称为$E$的\emph{\red 共轭空间}\index{共轭空间},记作$E^{\ast}$。
\end{defn}

\begin{thm}
  赋范线性空间$\mathcal{B}(E,E_1)$完备,当且仅当$E_1$完备。
\end{thm}
\begin{cor}
  赋范线性空间$E$的对偶空间$E^{\ast}$是Banach空间。
\end{cor}

\begin{defn}
线性空间$E$上的泛函$p(x)$称为
\begin{enumerate}[1)]
    \setlength{\itemindent}{2ex}
    \item \emph{\red 次加的}\index{次加泛函},如果
        \begin{equation*}
        p(x+y)\leqslant p(x)+p(y),\forall x,y\in E
        \end{equation*}
    \item \emph{\red 正齐性的}\index{正齐性的},如果
        \begin{equation*}
        p(\alpha x)=\alpha p(x), \forall\alpha\geqslant0,x\in E
        \end{equation*}
\end{enumerate}
\end{defn}




\begin{thm}[Hahn-Banach 延拓定理]\index{Hahn-Banach 延拓定理}
若
\begin{enumerate}[1)]
    \setlength{\itemindent}{2ex}
    \item $E$是实线性空间,$E_0\subset E$是其线性子空间;
    \item $p(y)$是$E$上的次加正齐性泛函,$f_0(y)$是定义在子空间$E_0$上的实线性泛函,且满足
      \begin{equation*}
        f_0(y)\leqslant p(y), \forall y\in E_0
      \end{equation*}
\end{enumerate}
则必存在定义在整个空间$E$上的实线性泛函$f(y)$,满足
\begin{enumerate}[1)]
    \setlength{\itemindent}{2ex}
    \item $f(y)=f_0(y), \forall y\in E_0$;
    \item $f(y)\leqslant p(y),\forall y\in E$。
\end{enumerate}
\end{thm}
\begin{rem}
  将定理中的``次加正齐性''控制泛函$p(x)$换成凸泛函时结论任成立。其中\emph{\red 凸泛函}\index{凸泛函}是指
  \begin{equation*}
  p(\lambda x+(1-\lambda)y)\leqslant\lambda p(x)+(1-\lambda)p(y), \forall\lambda\in[0,1],x,y\in E
  \end{equation*}
\end{rem}

\begin{defn}
线性空间$E$上的泛函$p(x)$称为\emph{\red 绝对齐性的}\index{绝对齐性的},如果
        \begin{equation*}
        p(\alpha x)=|\alpha|p(x), \forall\alpha\in\K,x\in E
        \end{equation*}
\end{defn}

\begin{thm}[Hahn-Banach 延拓定理]\index{Hahn-Banach 延拓定理}
若
\begin{enumerate}[1)]
    \setlength{\itemindent}{2ex}
    \item $E$是复线性空间,$E_0\subset E$是其复线性子空间;
    \item $p(y)$是$E$上的次加绝对齐性泛函,$f_0(y)$是定义在子空间$E_0$上的复线性泛函,且满足
      \begin{equation*}
        |f_0(y)|\leqslant p(y), \forall y\in E_0
      \end{equation*}
\end{enumerate}
则必存在定义在整个空间$E$上的复线性泛函$f(y)$,满足
\begin{enumerate}[1)]
    \setlength{\itemindent}{2ex}
    \item $f(y)=f_0(y), \forall y\in E_0$;
    \item $|f(y)|\leqslant p(y),\forall y\in E$。
\end{enumerate}
\end{thm}

\begin{thm}[Hahn-Banach 保范延拓定理]\index{Hahn-Banach 保范延拓定理}
设$E$为赋范线性空间,$E_0$为其线性子空间,$f_0$为$E_0$上定义的\emph{连续}线性泛函,则在$E$上必存在\emph{连续}线性泛函$f$,使得
\begin{enumerate}[1)]
    \setlength{\itemindent}{2ex}
    \item $\local{f}{E_0}=f_0$;
    \item $||f||=||f_0||$。
\end{enumerate}
\end{thm}

\begin{prop}[足够多有界线性泛函存在]
设$E$为赋范线性空间,则$\forall x_0\in E,||x_0||\neq0$,存在$f_1\in E^{\ast}$,使得
\begin{equation*}
f_1(x_0)=||x_0||, ||f_1||=1
\end{equation*}
\end{prop}
\begin{cor}
  设$E$为赋范线性空间,$x,y\in E$,则为使$x=y$必须且只须$\forall f\in E^{\ast}$,均有$f(x)=f(y)$。
\end{cor}

\begin{thm}[分割性定理]\index{分割性定理}
  设$E$是赋范线性空间,$E_0\subset E$是其线性子空间,则$\forall x_1\in E$,若
  \begin{equation*}
  d=d(x_1,E_0)\defeq\inf_{y\in E_0}||x_1-y||>0
  \end{equation*}
  则存在$f_1\in E^{\ast}$,使得
  \begin{equation*}
   f_1(x)=
    \begin{cases}
    1&x=x_1\\
    0&x\in E_0
    \end{cases}
  \end{equation*}
且$||f_1||=\frac{1}{d}$。
\end{thm}

\begin{thm}[Banach-Steinhaus定理(共鸣定理)]\index{Banach-Steinhaus定理}\index{共鸣定理}
  给定赋范线性空间$X$和$Y$,其中$X$是完备的,若算子族$\{T_{\alpha}\}_{\alpha\in\Lambda}\subset \mathcal{B}(X,Y)$是{\red 逐点连续}\index{逐点连续}的(即$\forall x\in X$,$\{||T_{\alpha}(x)||\}_{\alpha\in\Lambda}$都是有界数集),
  则必是{\red 一致连续}\index{一致连续}的(即$\{||T_{\alpha}||\}_{\alpha\in\Lambda}$是有界数集)。
\end{thm}

\begin{defn}
  $(X,d)$为距离空间,$A\subset X$, 若$\forall$开集$G\subset X$,存在开集$G_0\subset G$使得$G_0\cap A=\varnothing$,则称$A$在$X$中\emph{\red 稀疏}\index{稀疏集}。集合$A$称为\emph{\red 第一纲的}\index{第一纲集},如果它是可数个稀疏集的并,否则称为\emph{\red 第二纲}\index{第二纲集}的。
\end{defn}

\begin{thm}[推广的Banach-Steinhaus定理(共鸣定理)]\index{Banach-Steinhaus定理}\index{共鸣定理}
  给定赋范线性空间$X$和$Y$,其中$X$是{\red 第二纲的},若算子族$\{T_{\alpha}\}_{\alpha\in\Lambda}\subset \mathcal{B}(X,Y)$在$X$的一个第二纲的子集上是{\red 逐点连续}的,则必{\red 一致连续}。
\end{thm}

\section{可分性}
\begin{Qut}
  可分距离空间$X$的任意非空子集$X_0$必可分。
\end{Qut}
\begin{proof}
  由$X$可分,故存在可数稠密子集$\{x_n\}$。

  由于是距离空间,故可作一列开球$\{O(x_n,\frac{1}{m})\mid n,m\in\N\}$,并且
  \begin{equation*}
  X=\bigcup_{n,m}O(x_n,\frac{1}{m})
  \end{equation*}

  对于任意非空子集$X_0\subset X$,有
  \begin{equation*}
  X_0=\bigcup_{n,m}(X_0\cap O(x_n,\frac{1}{m}))
  \end{equation*}

  于是可从每个非空的$X_0\cap O(x_n,\frac{1}{m})$中取出一元。显见这些元的全体$\{y_k\}$至多可数。

  下面只须证$\{y_k\}$稠于$X_0$:

  $\forall y\in X_0,\varepsilon>0$,由于$\{x_n\}$在$X$中稠密,故存在$x_{n_0}$及$m_0>\frac{2}{\varepsilon}$,使$y\in O(x_{n_0},\frac{1}{m_0})$。

  因此,$X_0\cap O({n_0},\frac{1}{m_0})\neq\varnothing$,故必存在$y_{k_0}\in\{y_k\}$,使得$y_{k_0}\in X_0\cap O({n_0},\frac{1}{m_0})$

  于是
  \begin{equation*}
  d(y,y_{k_0})\leqslant d(y,x_{n_0})+d(x_{n_0},y_{k_0})<\frac{1}{m_0}+\frac{1}{m_0}<\varepsilon
  \end{equation*}

  即$\{y_k\}$在$X_0$中稠密。故$X_0$可分。
\end{proof}

\begin{defn}
  $(X,d)$为一个距离空间,$A,B\subset(X,d)$,若$B$的每一个点的每一个邻域中均含有$A$中的点,则称$A$在$B$中\emph{\red 稠密}\index{稠密}。
\end{defn}

\begin{prop}
  $A$在$B$中稠密$\Longleftrightarrow \overline{A}\supset B\Longleftrightarrow\forall x\in B$,存在$\{x_n\}\subset A$,使得$x_n\stackrel{d}{\to}x$

  $\Longrightarrow\forall\varepsilon>0,\bigcup_{x\in A}O(x,\varepsilon)\supset B$
\end{prop}

\begin{defn}
  $(X,d)$为一个距离空间,若$(X,d)$中存在可数的稠密子集,则称$(X,d)$\emph{\red 可分}\index{可分空间}。
\end{defn}

\section{内积空间与正交补}
\begin{Qut}
  $A$是内积空间$U$的子集,则$A^{\bot}$是$U$的闭子空间。
\end{Qut}

\begin{defn}
  设$E$是线性空间,在乘积空间$E\times E$上定义有有序二元泛函$(,)$,如果$\forall x,y,z\in E,\alpha,\beta\in\K$,其满足
\begin{enumerate}[1)]
    \setlength{\itemindent}{2ex}
    \item $(x,x)>=0$,且$(x,x)=0$当且仅当$x=0$;
    \item $(x,y)=(y,x)$;
    \item $(\alpha x+\beta y,z)=\alpha(x,z)+\beta(y,z)$
\end{enumerate}
则称$(,)$为$E$上的\emph{\red 内积}\index{内积},称定义了内积的线性空间$E$为\emph{\red 内积空间}\index{内积空间}。
\end{defn}

\begin{thm}[Cauchy-Schwarz不等式]\index{Cauchy-Schwarz不等式}
设$E$为内积空间,则$\forall x,y\in E$,有
\begin{equation*}
|(x,y)|^2\leqslant(x,x)(y,y)
\end{equation*}
且等号成立当且仅当$x,y$线性相关。
\end{thm}

\begin{defn}
  令$||x||=\sqrt{(x,x)}$,则$||x||$成为$E$上的范数。若$E$在该范数下成为Banach空间,则称为\emph{\red Hilbert空间}\index{Hilbert空间}。
\end{defn}

\begin{prop}
  内积$(x,y)$是关于$x$和$y$的二元连续函数,即
  \begin{equation*}
  \lim_{n\to\infty}x_n=x, \lim_{n\to\infty}y_n=y\Longrightarrow\lim_{n\to\infty}(x_n,y_n)=(x,y)
  \end{equation*}
\end{prop}

\begin{prop}[勾股定理]\index{勾股定理}
  设$U$是内积空间,$x,y\in U$。若$x\bot y$,则
  \begin{equation*}
  ||x+y||^2=||x||^2+||y||^2
  \end{equation*}
\end{prop}

\begin{defn}
  设$U$是内积空间
\begin{enumerate}[1)]
    \setlength{\itemindent}{2ex}
    \item 对于$x,y\in U$,如果$(x,y)=0$,则称$x$与$y$\emph{\red 正交}\index{正交},记作$x\bot y$;
    \item 设$\varnothing\neq A\subset U, x\in U$,如果$\forall y\in A, (x,y)=0$,则称$x$与$A$\emph{\red 正交},记作$x\bot A$;
    \item 类似可定义两个子集的\emph{\red 正交};
    \item 设$\varnothing\neq A\subset U$,把与$A$\emph{\red 正交}的元全体记为$A^{\bot}$,称为$A$的\emph{\red 正交补}\index{正交补}。
\end{enumerate}
\end{defn}

\begin{prop}
  若$x\bot y_i(i=1,2,\cdots,n)$,则$\forall\alpha_i\in\K$,有$x\bot \sum\alpha_iy_i$。
\end{prop}
\begin{proof}
$(x,\sum\alpha_iy_i) =\sum\overline{\alpha_i}(x,y_i) =\sum\overline{\alpha_i}0=0$,故$x\bot \sum\alpha_iy_i$。
\end{proof}

\begin{prop}
  若$x\bot y_i(n\in\N)$,且$\lim_{n\to\infty}y_n=y$,则$x\bot y$。
\end{prop}
\begin{proof}
  $(x,y)=\lim_{n\to\infty}(x,y_n)=\lim_{n\to\infty}0=0$,故$x\bot y$。
\end{proof}

\begin{proof}[题目的证明]
由前面两个命题立得。
\end{proof}

\section{闭算子}
\begin{Qut}
  闭算子的定义与等价定义。
\end{Qut}
\begin{defn}
  令$X,Y$为$\K$上的两个赋范线性空间,$T$是从$X$到$Y$内的线性算子,对$X\times Y$赋予范数$||(x,y)||=||x||_X+||y||_Y$,如果$T$的图像$G(T)\subset$是闭集,
  则称算子$T$是\emph{\red 闭算子}\index{闭算子}。
\end{defn}

\begin{prop}
  $T$是闭算子,当且仅当:

  $\forall\{x_n\}\subset\mathcal{D}(T)$满足
  \begin{equation*}
  \lim_{n\to\infty}x_n=x, \lim_{n\to\infty}Tx_n=y
  \end{equation*}
  都有$x\in\mathcal{D}(T)$且$Tx=y$。
\end{prop}
\begin{proof}
  一方面,如果$x_n\to x,Tx_n\to y$,即$(x_n,Tx_n)\to(x,y)$,则由于$T$是闭算子,故$(x,y)\in G(T)$。由$G(T)$的定义知$x\in\mathcal{D}(T)$且$Tx=y$。

 另一方面,对于$G(T)$中任何收敛列$\{(x_n,y_n)\}$,设其极限为$(x,y)$,则$x_n\to x,Tx_n\to y$,由于$y_n=Tx_n$,从而由假设条件知$x\in\mathcal{D}(T)$且$Tx=y$,于是$(x,y)=(x,Tx)\in G(T)$。
\end{proof}

\begin{prop}
  设$T$是从$X$到$Y$内的{\red 有界}线性算子,且$\mathcal{D}(T)$是闭集,则$T$是闭算子。
\end{prop}
\begin{proof}
  $\forall\{x_n\}\subset\mathcal{D}(T)$,如果$\lim_{n\to\infty}x_n=x, \lim_{n\to\infty}Tx_n=y$,则由$\mathcal{D}(T)$闭得$x\in\mathcal{D}(T)$,由$T$连续得$\lim_{n\to\infty}Tx_n=Tx$,故由极限的唯一性,得$y=Tx$,即$(x,y)\in G(T)$。
\end{proof}

\section{补充}
\subsection{准范数}
\begin{defn}
范数的定义:零元、三角不等式、绝对齐性。

准范数的定义:将范数定义中的绝对齐性换成:
\begin{enumerate}[a)]
    \setlength{\itemindent}{2ex}
    \item $||-x||=||x||$
    \item $||a_nx||\to0(a_n\to0)$
    \item $||\alpha x_n||\to0(||x_n||\to0)$
\end{enumerate}
\end{defn}

\subsection{商空间}
\begin{thm}
设$E$为一赋范线性空间,$E_0$为其闭线性子空间,在商空间$E/E_0$定义
\begin{equation*}
||\widetilde{x}||\defeq\inf_{x\in\widetilde{x}}||x||
\end{equation*}
则其必为一范数,从而使$E/E_0$成为赋范空间。
\end{thm}

\begin{thm}
设$E$为一赋范线性空间,$E_0$为其闭线性子空间,则$E$完备当且仅当$E_0$和$E/E_0$均完备。
\end{thm}

\subsection{Hilbert空间上的伴随算子}
\begin{defn}
  对于$T\in\mathcal{B}(H)$,若$(Tx,y)=(x,y')$,则$T^{\ast}\colon y\to y'$称为$T$的\emph{\red 伴随算子}\index{伴随算子}。
\end{defn}
\begin{prop}
设$A,B$是Hilbert空间上的线性算子,则
\begin{enumerate}[1)]
    \setlength{\itemindent}{2ex}
    \item $(\alpha A+\beta B)^{\ast}=\overline{\alpha}A^{\ast}+\overline{\beta}B^{\ast}$
    \item $A^{\ast\ast}=A$
    \item $(AB)^{\ast}=B^{\ast}A^{\ast}$
    \item $I^{\ast}=I$
    \item 若$A^{-1}$存在,则$(A^{-1})^{\ast}=(A^{\ast})^{-1}$。
\end{enumerate}
\end{prop}

\begin{prop}
$T\in\mathcal{B}(H,H')$,则
\begin{enumerate}[1)]
    \setlength{\itemindent}{2ex}
    \item $\ker T=(\im T^{\ast})^{\bot}$
    \item $\overline{\im T}=(\ker T^{\ast})^{\bot}$
\end{enumerate}  
\end{prop}

\begin{defn}
设$T\in\mathcal{B}(H)$,若$T=T^{\ast}$,则称$T$是\emph{\red 自伴的}\index{自伴算子}。
\end{defn}

\begin{prop}
\begin{enumerate}[1)]
    \setlength{\itemindent}{2ex}
    \item 设$A,B$自伴,则$A+B$自伴;
    \item 设$A$自伴,$\alpha\in\Real$,则$\alpha T$自伴;
    \item 设$A,B$自伴,则$AB$自伴,且$AB=BA$;
    \item 设$T_n$自伴,且$||T_n-T||\to0$,则$T$自伴。
\end{enumerate} 
\end{prop}

\begin{prop}
$T$是自伴算子当且仅当$\forall x\in H,(Tx,x)\in\Real$,且
\begin{equation*}
||T||=\sup_{||x||=1}|(Tx,x)|
\end{equation*}
\end{prop}

\begin{defn}
设$T\in\mathcal{B}(H)$,令
\begin{equation*}
A=\frac{T+T^{\ast}}{2}, B=\frac{T-T^{\ast}}{2}
\end{equation*}
则$A,B\in\mathcal{B}(H)$均为自伴算子,且$T=A+iB,T^{\ast}=A-iB$,分别称$A,B$为$T$的\emph{\red 实部}\index{实部}和\emph{\red 虚部}\index{虚部}。
\end{defn}

\begin{defn}
设$T\in\mathcal{B}(H)$,若$TT^{\ast}=T^{\ast}T$,则称$T$为\emph{\red 正规算子}\index{正规算子}。
\end{defn}

\begin{prop}
  $T\in\mathcal{B}(H)$为正规算子当且仅当$||Tx||=||T^{\ast}x||,\forall x\in H$。
\end{prop}

\begin{defn}
设$T\in\mathcal{B}(H)$,若$T^{\ast}=T^{-1}$,则称$T$为\emph{\red 酉算子}\index{酉算子}。
\end{defn}

\subsection{正交投影}
\begin{thm}[正交投影定理]\index{正交投影定理}
  设$E$为Hilbert空间,$M$为$E$内的闭线性子空间,则$\forall x\in E$,$x$必可写成$x=y+z$的形式,其中$y\in M,z\in M^{\bot}$,并且
  \begin{equation*}
  ||x-y||=\inf_{y'\in M}||x-y'||
  \end{equation*}
  此分解称为{\red 正交分解}\index{正交分解},$y$称为$x$在$M$上的{\red 正交投影}\index{正交投影}。
\end{thm}
\begin{prop}
  设$M$为Hilbert空间$E$内的闭线性子空间,$P_M$为从$E$到$M$的正交投影,则$P_M$满足
\begin{enumerate}[1)]
    \setlength{\itemindent}{2ex}
    \item $P_M$是$E$上的线性满算子;
    \item $||P_Mx||\leqslant||x||,\forall x\in E$;
    \item $||P_M||=1$或$0$;
    \item $P_M^2=P_M$;
    \item $\im P_M=M,\ker P_M=M^{\bot}$
\end{enumerate}
\end{prop}

\begin{prop}
  $P$是正交投影算子当且仅当$P$是自伴算子且$P^2=P$。
\end{prop}


\newpage
\phantomsection
\addcontentsline{toc}{section}{Index}
\printindex
\end{document}
