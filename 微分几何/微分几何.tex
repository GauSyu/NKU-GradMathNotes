%\documentclass[UTF8]{ctexart}
%基本文件设置,Ctex中文+UTF8编码。需要XeTex编译。
%用这个生成中文索引USE:cctmkind -o 微分几何.ind -s cct.ist -C pinyin 微分几何.idx
%备选USE:cctmkind -o 微分几何.ind -s cct.ist -C stroke 微分几何.idx
%备选USE:cctmkind -o 微分几何.ind -s cct.ist -C mixed 微分几何.idx
%------------------文件设置-----------------
\documentclass[winfonts,UTF8,c5size,a4paper,fancyhdr,hyperref,titlepage,nocap]{ctexart}
%winfonts=使用Windows 的字体设置,默认为六种中易字体:宋体、仿宋、黑体、楷体、隶书、幼圆(在使用XeTEX时只有前四种)。该选项的结果将和老版本ctex宏包完全一致。这是默认设置。
%cs4size=使用小四字号为缺省字体大小。
%a4paper=页面大小为A4。
%fancyhdr=保持和fancyhdr宏包的兼容性。该选项将使得fancyhdr宏包被自动调用。
%hyperref=自动判断调用fancyhdr宏包的正确参数以避免产生乱码。该选项将使得hyperref宏包被自动调用。
%------------------整体格式-----------------
%\usepackage[xetex]{hyperref}%让生成的文章目录有链接和标签。
%\usepackage{ifthen}
%条件分支
%\ifthenelse{test}{then_text}{else_text}%
%测试test,如果为真,执行then_text语句;如果为假,执行else_text。
%\whiledo{test}{do_text}
%测试test,为真则执行do_text语句,为假则不执行。
\usepackage{amsmath,amsthm,amssymb,amsfonts,bm}         %AMS数学包。
\usepackage[all]{xy}                                    %交换图表。
\usepackage{epic}															%增强绘图。
\usepackage{graphicx}														%图像增强
\usepackage{mathrsfs,mathtools}
\usepackage{cite}
\usepackage{longtable}
\usepackage{enumerate}
\usepackage{upgreek}%直立的小写希腊字母
\usepackage[text={170mm,250mm},centering]{geometry}   %调整页面布局
%\CTEXsetup[number={},indent={-1ex}]{section}
\usepackage{mathrsfs}
\usepackage{makeidx}
\usepackage{color}
\CTEXoptions[today=small]
\renewcommand{\labelitemii}{\textbullet}
\renewcommand{\labelitemiii}{\textbullet}
\renewcommand{\labelitemiv}{\textbullet}
%-----------------定理命题-----------------
\newtheorem{thm}{定理}
\newtheorem{cor}[thm]{推论}
\newtheorem{con}[thm]{结论}
\newtheorem{lem}[thm]{引理}
\newtheorem{prop}[thm]{命题}
\newtheorem{axiom}{公理}
\newtheorem{xiti}{习题}
\newtheorem{exa}[thm]{例}
\theoremstyle{definition}
\newtheorem{defn}[thm]{定义}
\theoremstyle{remark}
\newtheorem*{rem}{注}
\newtheorem*{note}{约定}
\numberwithin{equation}{subsection}
\renewcommand{\proofname}{\bf  证明}
% THEOREM Environments --------------------------------------------------------
%\newtheorem{thm}{Theorem}[subsection]
%\newtheorem{cor}[thm]{Corollary}
%\newtheorem{lem}[thm]{Lemma}
%\newtheorem{prop}[thm]{Proposition}
%\theoremstyle{definition}
%\newtheorem{defn}[thm]{Definition}
%\theoremstyle{remark}
%\newtheorem{rem}[thm]{Remark}
%\numberwithin{equation}{subsection}
%-----------------页眉页脚-----------------
%\usepackage{fancyhdr}                       %页眉页脚包
%\pagestyle{fancy}                           %设置页眉
%-----------------制定命令-----------------
% MATH ------------------------------------------------------------------------
\DeclareMathOperator{\RE}{Re}
\DeclareMathOperator{\IM}{Im}
\DeclareMathOperator{\ess}{ess}
\newcommand{\eps}{\varepsilon}
\newcommand{\To}{\longrightarrow}
\newcommand{\h}{\mathcal{H}}
\newcommand{\s}{\mathcal{S}}
\newcommand{\A}{\mathcal{A}}
\newcommand{\J}{\mathcal{J}}
%\newcommand{\M}{\mathcal{M}}
\newcommand{\W}{\mathcal{W}}
\newcommand{\X}{\mathcal{X}}
\newcommand{\BOP}{\mathbf{B}}
\newcommand{\BH}{\mathbf{B}(\mathcal{H})}
\newcommand{\KH}{\mathcal{K}(\mathcal{H})}
\newcommand{\zz}{\mathbb{Z}}
\newcommand{\Q}{\mathbb{Q}}
\newcommand{\Real}{\mathbb{R}}
\newcommand{\Complex}{\mathbb{C}}
\newcommand{\Field}{\mathbb{F}}
\newcommand{\RPlus}{\Real^{+}}
\newcommand{\pp}{\mathcal{P}}
\newcommand{\Polar}{\mathcal{P}_{\s}}
\newcommand{\Poly}{\mathcal{P}(E)}
\newcommand{\EssD}{\mathcal{D}}
\newcommand{\Lom}{\mathcal{L}}
\newcommand{\States}{\mathcal{T}}
\newcommand{\abs}[1]{\left\vert#1\right\vert}
\newcommand{\set}[1]{\left\{#1\right\}}
\newcommand{\seq}[1]{\left<#1\right>}
\newcommand{\norm}[1]{\left\Vert#1\right\Vert}
\newcommand{\essnorm}[1]{\norm{#1}_{\ess}}
\newcommand{\ob}{\mathrm{ob}}
\newcommand{\g}{\mathfrak{g}}
\newcommand{\base}[1]{\mathrm{base}(\mathscr{#1})}
\newcommand{\bun}[1]{\mathrm{bun}\mathcal{#1}}
\newcommand{\fib}{\mathrm{fib}}
\newcommand{\RP}{\mathbb{R}\mathbf{P}}
\newcommand{\red}{\color{red}}
\newcommand{\dd}{\operatorname{d}}
\newcommand{\dt}{\frac{\operatorname{d}}{\operatorname{d}t}}
\newcommand{\pfrac}[2]{\frac{\partial{#1}}{\partial{#2}}}
\newcommand{\px}[1]{\left.\pfrac{}{x^{#1}}\right|_p}
\newcommand{\py}[1]{\left.\pfrac{}{y^{#1}}\right|_q}
\newcommand{\rank}{\operatorname{rank}}
\newcommand{\spa}{\operatorname{span}}
\newcommand{\tr}{\operatorname{tr}}
\newcommand{\Xf}[1]{\mathfrak{X}(#1)}
\newcommand{\local}[2]{\left.{#1}\right|_{#2}}%Local #1 at #2
\newcommand{\localt}[1]{\local{#1}{t=0}}%Local #1 at  t=0
\newcommand{\localp}[1]{\local{#1}{p}}%Local #1 at  p
%==================================================
\newcommand{\defen}{\mathop{\iff}\limits^{\mathclap{\text{\scriptsize def}}}}%s上标箭头
\newcommand{\markar}[1]{\stackrel{{#1}}{\longrightarrow}}%带标号箭头(定长,与\xleftarrow不同)
\newcommand{\defeq}{\stackrel{{\mathrm{def}}}{=}}
\newcommand{\mexseq}[2]{\cdots \longrightarrow {#1}_{n-1} \markar{{#2}_{n-1}} {#1}_n \markar{{#2}_n} {#1}_{n+1} \longrightarrow \cdots}
  %middle exact sequence
\newcommand{\schain}[2]{{#1}_0 \markar{{#2}_0} {#1}_1 \markar{{#2}_1} {#1}_2 \longrightarrow \cdots}
  %起始链
\newcommand{\fchain}[2]{\cdots \longrightarrow {#1}_{n-2} \markar{{#2}_{n-2}} {#1}_{n-1} \markar{{#2}_{n-1}} {#1}_n}
  %结束链
\newcommand{\lchain}[2]{{#1}_0 \markar{{#2}_0} {#1}_1 \markar{{#2}_1} {#1}_2 \longrightarrow \cdots \longrightarrow {#1}_{n-2} \markar{{#2}_{n-2}} {#1}_{n-1} \markar{{#2}_{n-1}} {#1}_n}
  %完整长链
\newcommand{\mapdes}[5]
  {
    \begin{align*}
      #1\colon\  #2 & \longrightarrow  #3 \\
            #4 & \mapsto  #5
    \end{align*}
    %多于两个对齐点,align会出问题。换用alignat。
  }
\newcommand{\id}{\operatorname{id}}
\newcommand{\sgn}{\operatorname{sgn}}
\newcommand{\End}{\operatorname{End}}
\newcommand{\grad}{\operatorname{grad}}%梯度
\newcommand{\curl}{\operatorname{curl}}%旋度
\newcommand{\Div}{\operatorname{div}}%散度
\newcommand{\supp}{\operatorname{supp}}
\newcommand{\im}{\operatorname{im}}
\newcommand{\Ad}{\operatorname{Ad}}
\newcommand{\ad}{\operatorname{ad}}
%------------------标题作者-----------------
\title{微分几何笔记}
\author{Gau Syu}
\date{last version: \today}
%\setcounter{secnumdepth}{1}
\makeindex
\begin{document}
\maketitle
\pagestyle{plain}
\section*{Preface}
\addcontentsline{toc}{section}{Preface}
这是2012年下半年南开大学数学科学学院黄利兵老师的研究生课程“微分几何”的笔记。

过往版本:
\begin{itemize}
  \item ver1.0: 2013.1.2
  \item ver1.1: 2013.1.22
\end{itemize}

\subsection{主要内容}
\begin{enumerate}
  \item 微分流形
  \begin{enumerate}
    \item 向量场
    \item 张量场
    \item 切丛、余切丛
  \end{enumerate}
  \item 外微分
  \begin{enumerate}
    \item 活动标架法和李群
  \end{enumerate}
  \item 主纤维丛上的联络
\end{enumerate}
\subsection{教材}
陈省身,《微分几何讲义》
\subsection{参考书}
\begin{itemize}
  \item Kobayashi \& Nomizu\\
Foundations of Differential Geometry\\
Vol.I Chapts 1-2
  \item M.Spivak\\
A Comprehensive Introduction to Differential Geometry\\
Vol. II 1,Gauss's paper 2,Riemann's report 3,联络概念的5个版本
  \item J.M.Lee, Introduction to Smooth Manifolds(GTM218)
\end{itemize}
\tableofcontents
\newpage
\section{微分流形}
\subsection*{2012-9-13}
记号:$\Real^n$。
\begin{defn}
  $n$元函数$f:\Real^n\to\Real^n$,若$f$的任意$r$阶偏导数存在且连续,则称$f$为{\red$C^r$}\index{$C^r$函数}的。
\end{defn}
\begin{exa}
  $C^{\infty}$,{\red 光滑函数}\index{光滑函数}。

  $C^{\omega}$,{\red 解析函数}\index{解析函数}。
\begin{equation*}
f(x)=f(x_0)+\sum\frac{\partial f}{\partial x_i}(x_0)(x_i-(x_0)_i)+\frac{1}{2}\sum\frac{\partial^2f}{\partial x_i\partial x_j}(x_0)(x_i-(x_0)_i)(x_j-(x_0)_j)+\cdots
\end{equation*}
\end{exa}
\begin{defn}
  设$M$是Hausdorff空间,若对任意的$x\in M$,存在$x$的领域$U\subset M$使得存在同胚$\varphi\colon U\to\Real^n$,\footnote{每一处的维数相同。}则称$M$为{\red \emph{($n$维)流形}}\index{流形}。$(\varphi, U)$ 称为$x$ 处的{\red \emph{坐标系}}\index{坐标系 }。
\end{defn}
\begin{defn}
  设$M$为$n$维流形,若坐标系的集合$\mathscr{A}=\{(\varphi_i, U_i)\mid i\in I\}$使得
    \begin{enumerate}[1)]
    \setlength{\itemindent}{2ex}
    \item $M$为$\mathscr{A}$所覆盖,即$M=\bigcup\limits_{i\in I}U_i$。
    \item $\mathscr{A}$是{\red\emph{$C^r$相容}}\index{相容}的,即或者$U_i\cap U_j=\varnothing$,或者$\varphi_i\circ\varphi^{-1}_j\colon\varphi_j(U_i\cap U_j)\to\varphi_i(U_i\cap U_j)$是$C^r$的。
    \item $\mathscr{A}$具有极大性,即对任意坐标系$(\varphi, U)$,若其与任意$(\varphi_i, U_i)$是$C^r$相容的,则$(\varphi, U)\in\mathscr{A}$。
  \end{enumerate}
  则称$\mathscr{A}$为$M$上的一个{\red\emph{$C^r$微分结构}}\index{微分结构}。赋予$C^r$微分结构的流形$M$称为一个{\red\emph{$C^r$微分流形}}\index{$C^r$微分流形}。
\end{defn}
\begin{exa}
  $M=\Real^n$,取$\mathscr{A}=\{(\id, \Real^n)\}$。
\end{exa}
\begin{exa}
  $M=S^n=\{(x_0,x_1,\cdots,x_n)\in\Real^{n+1}\mid x^2_0+x^2_1+\cdots+x^2_n=1\}$,取$U_1=S^n\backslash\{(-1,\vec{0})\}$,$U_2=S^n\backslash\{(1,\vec{0})\}$,则$S^n=U_1\cup U_2$。 定义
  \begin{center}
  \parbox{0.4\linewidth}{\mapdes{\varphi_1}{U_1}{\Real^n}{(x_0,\vec{x})}{\frac{\vec{x}}{1+x_0}}}
  \parbox{0.4\linewidth}{\mapdes{\varphi_2}{U_2}{\Real^n}{(x_0,\vec{x})}{\frac{\vec{x}}{1-x_0}}}
  \end{center}
  则$(\varphi_1,U_1)$与$(\varphi_2,U_2)$是光滑相容的:
  \mapdes{\varphi_1\circ\varphi^{-1}_2}{\Real^n\backslash\{0\}}{\Real^n\backslash\{0\}}{\vec{u}}{\frac{\vec{u}}{|\vec{u}|^2}}
\end{exa}
\begin{proof}
\begin{align*}
    \begin{cases}
    \cfrac{\vec{x}}{1-x_0}=\vec{u}\\
    x_0^2+|\vec{x}|^2=1
    \end{cases}
    &\Longrightarrow
    \begin{cases}
    \vec{x}=(1-x_0)\vec{u}\\
    x_0^2+(1-x_0)^2|\vec{u}|^2=1
    \end{cases}
    \\
  &\Longrightarrow \varphi^{-1}_2(\vec{u})=(\frac{|\vec{u}|^2-1}{|\vec{u}|^2+1},\frac{2\vec{u}}{|\vec{u}|^2+1}) \\
  &\Longrightarrow \varphi_1\circ\varphi^{-1}_2(\vec{u})=\cfrac{\frac{2\vec{u}}{|\vec{u}|^2+1}}{\frac{2|\vec{u}|^2}{|\vec{u}|^2+1}}=\frac{\vec{u}}{|\vec{u}|^2}
\end{align*}
\end{proof}
\begin{exa}
  $M=\RP^n=(\Real^{n+1}\backslash\{0\})/\sim$,这里$(x_0,x_1,\cdots,x_n)\sim(y_0,y_1,\cdots,y_n)$指$\exists\lambda\in\Real$,使得$x_i=\lambda y_i$。
令$Z_i=\{(x_0,x_1,\cdots,x_n)\in\Real^{n+1}\mid x_i\neq0\}$,$U_i=Z_i/\sim$,则$\RP^n=U_0\cup U_1\cup\cdots\cup U_n$。

定义$\varphi_i([x_0,x_1,\cdots,x_n])=(\frac{x_0}{x_i},\frac{x_1}{x_i},\cdots,\frac{x_{i-1}}{x_i},\frac{x_{i+1}}{x_i},\cdots,\frac{x_n}{x_i})$,则任意$(\varphi_i,U_i)$与$(\varphi_j,U_j)$相容:
\begin{align*}
  \varphi_j^{-1}(r_1,r_2,\cdots,r_n)&=[r_1,r_2,\cdots,r_i,1,r_{j+1},\cdots,r_n] \\
  \varphi_i\circ\varphi_j^{-1}(r_1,r_2,\cdots,r_n)&=\varphi_i([r_1,r_2,\cdots,r_i,1,r_{j+1},\cdots,r_n])\\
  &=\begin{cases}
  (\frac{r_1}{r_i},\frac{r_2}{r_i},\cdots,\frac{r_j}{r_i},\frac{1}{r_i},\frac{r_{j+1}}{r_i},\cdots,\frac{r_{i-1}}{r_i},\frac{r_{i+1}}{r_i},\cdots,\frac{r_n}{r_i})&i>j\\
  (\frac{r_1}{r_i},\frac{r_2}{r_i},\cdots,\frac{r_{i-1}}{r_i},\frac{r_{i+1}}{r_i},\cdots,\frac{r_j}{r_i},\frac{1}{r_i},\frac{r_{j+1}}{r_i},\cdots,\frac{r_n}{r_i})&i<j
  \end{cases}
\end{align*}
\end{exa}
\begin{defn}
  设$M$、$N$为光滑流形,$f\colon M\to N$是连续映射。对于$x\in M$,若存在$x$处的坐标系$(\varphi,U)$和$f(x)$处的坐标系$(\psi,V)$,使得
  \begin{equation*}
  \psi\circ f\circ\varphi^{-1}\colon\Real^{\dim M}\longrightarrow\Real^{\dim N}
  \end{equation*}
  是光滑映射,则称$f$\emph{\red 在$x$处光滑}。

  若$\forall x\in M$,$f$在$x$处光滑,则称$f$为\emph{\red 光滑映射}\index{光滑映射},记作$f\in C^{\infty}(M,N)$。
\end{defn}
\begin{exa}
  光滑映射$f\colon M\to\Real$称为$M$上的{\red 光滑函数}\index{光滑函数}。$M$上的光滑函数全体记为$C^{\infty}(M)$。
\end{exa}
\begin{exa}
  若$f\colon M\to N$是同胚,且$f\in C^{\infty}(M,N),f^{-1}\in C^{\infty}(N,M)$,则称$f$ 为{\red 光滑同胚}\index{光滑同胚}。
\end{exa}
\begin{exa}
  光环映射$f\colon\Real\to M$称为$M$上的{\red 光滑曲线}\index{光滑曲线}。
\end{exa}
\begin{exa}
  考虑$f\colon S^3\to S^2$,其中
  \begin{align*}
  S^3&=\{(z_1,z_2)\in\Complex^2\mid|z_1|^2+|z_2|^2=1\} \\
  S^2&=\{(\alpha,\beta)\in\Real\times\Complex\mid\alpha^2+|\beta|^2=1\} \\
  f(z_1,z_2)&=(|z_1|^2-|z_2|^2,2\overline{z_1}z_2)
  \end{align*}
  这个$f$称为{\red Hopf fibration}\index{Hopf fibration}。
\begin{displaymath}
      \xymatrix{
        \Complex^2\ar[r]^-{\widetilde{f}}&\Real\times\Complex\\
        S^3\ar[r]^{f}\ar[u]^{i}&S^2\ar[u]^{i}
      }
\end{displaymath}

对于$\xi\in\Complex,|\xi|=1$,$f(\xi z_1,\xi z_2)=f(z_1,z_2)$。也就是说,$S^3$中的一个圆周被映射为$S^2$上的一个点。
\end{exa}
\begin{defn}
  设$M$和$N$为光滑流形,微分结构分别为
\begin{gather*}
\mathscr{A}=\{(\varphi_i,U_i)\mid i\in A\}\\
\mathscr{B}=\{(\psi_j,V_j)\mid j\in B\}
\end{gather*}
则
\begin{equation*}
\mathscr{A}\times\mathscr{B}=\{(\varphi_i\times\psi_j,U_i\times V_j)\mid i\in A, j\in B\}
\end{equation*}
定义$M\times N$上微分结构,使$M\times N$成为一个光滑流形,称为$M$和$N$的\emph{\red 乘积流形}\index{乘积流形}。
其中
\begin{equation*}
(\varphi_i\times\psi_j)(x,y)=(\varphi_i(x),\psi_j(y))\in\Real^{\dim M}\times\Real^{\dim N}
\end{equation*}
\end{defn}
\begin{rem}
  $S^3\neq S^2\times S^1$。
\end{rem}

\section{切空间和余切空间}
\subsection*{2012-9-20}
\begin{defn}
对于光滑曲线$\gamma\colon(-\varepsilon,\varepsilon)\to M$,设$\gamma(0)=p$,定义$\gamma$在$p$点的\emph{\red 切向量}\index{切向量 } $X_p=\dot{\gamma}(0)$为映射
\mapdes{X_p}{C^{\infty}_p}{\Real}{f}{\localt{\dt f(\gamma(t))}}
\end{defn}
\begin{prop}
切向量$X_p\colon C^{\infty}_p\to\Real$是实线性映射,且满足{\red Leibniz's Law}\index{Leibniz's Law}
\begin{equation*}
X_p(fg)=X_p(f)g(p)+X_p(g)f(p)
\end{equation*}
\end{prop}
\begin{proof}
  对任意的$\lambda, \mu\in\Real, f,g\in C^{\infty}_p$有
\begin{align*}
X_p(\lambda f+\mu g)&=\localt{\dt(\lambda f+\mu g)(\gamma(t))}\\
                    &=\localt{\dt(\lambda f(\gamma(t))+\mu g(\gamma(t)))}\\
                    &=\lambda X_p(f)+\mu X_p(g)\\
X_p(fg)&=\localt{\dt(f(\gamma(t))g(\gamma(t)))}\\
       &=\localt{\dt f(\gamma(t))}g(p)+\localt{\dt g(\gamma(t))}f(p)\\
       &=X_p(f)g(p)+X_p(g)f(p)
\end{align*}
\end{proof}
\begin{note}
  取$p\in M$处的坐标系$(\varphi,U)$,设$\varphi(p)=(x_0^1,x_0^2,\cdots,x_0^n)$,考虑过$p$的曲线
\begin{equation*}
\gamma_i(t)=\varphi^{-1}(x_0^1,x_0^2,\cdots,x_0^i+t,\cdots,x_0^n)
\end{equation*}
将它在$p$点的切向量记作$\px{i}$。
\end{note}
\begin{prop}
取定$p\in M$处的坐标系$(x^i)$,则$p$点的任一切向量$X_p$可写为$\px{1},\px{2},\cdots,\px{n}$的线性组合。
反之,$\forall \xi^i\in\Real, \xi^i\px{i}$必为切向量。
\end{prop}
\begin{proof}
  任一曲线$\gamma$,设$\varphi(\gamma(t))=(\gamma^1(t),\gamma^2(t),\cdots,\gamma^n(t))$。则$\gamma$ 在$p$点的切向量$X_p$满足
\begin{align*}
X_p(f)&=\localt{\dt(f(\gamma(t)))}\\
      &=\localt{\dt(f\circ\varphi^{-1}\circ\varphi\circ\gamma(t))}\\
      &=\localt{\dt(\widetilde{f}(\gamma^1(t),\gamma^2(t),\cdots,\gamma^n(t)))}\\
      &=\pfrac{\widetilde{f}}{x^i}(\gamma^1(0),\gamma^2(0),\cdots,\gamma^n(0))\gamma^i(0)
\end{align*}
下面说明它等于$(\px{i})(f)\gamma^i(0)$:
\begin{align*}
(\px{i})(f)&=\localt{\dt f(\gamma_i(t))}\\
           &=\localt{\dt f(\varphi^{-1}(x_0^1,x_0^2,\cdots,x_0^i+t,\cdots,x_0^n))}\\
           &=\pfrac{\widetilde{f}}{x^i}(x_0^1,x_0^2,\cdots,x_0^n)
\end{align*}

这表明$X_p$与$\dot{\gamma}^i(0)(\px{i})$在任何$f\in C^{\infty}_p$上的作用相等。所以$X_p=\dot{\gamma}^i(0)(\px{i})$。

下面证明$\px{1},\px{2},\cdots,\px{n}$线性无关:

若有$\xi^i\in\Real$使得$\xi^i\px{i}=0$,则
\begin{equation*}
0=(\xi^i\px{i})(x^j\circ\varphi)=\xi^i\delta^j_i=\xi^j
\end{equation*}
故$\px{1},\px{2},\cdots,\px{n}$线性无关。
\end{proof}
\begin{thm}
$p$点的全体切向量构成$n=\dim M$维线性空间,且任取$p$点的坐标系$(x^i)$则$\{\px{i}\}$ 构成它的一组基。将这个实线性空间称为$p$点的{\red 切空间}\index{切空间 },记作$T_pM$。
\end{thm}
\begin{rem}
切向量有局部性:$\forall f,g\in C^{\infty}_p$,若$f|_U=g|_U$,则$X_p(f)=X_p(g)$。等价地说,$\forall f\in C^{\infty}_p$,若$f|_U=0$,则$X_p(f)=0$。

定义$C^{\infty}_p$之等价关系为$f\sim g$当且仅当存在$p$的领域$U$使得$f|_U=g|_U$。该关系下的等价类称为在$p$点的\emph{\red 函数芽}\index{函数芽 }(germ\index{germ})。
\end{rem}
\begin{defn}
$T_pM$的对偶空间,称为$p$点的\emph{\red 余切空间}\index{余切空间},记作$T^{\ast}_p$,其中元素称为\emph{\red 余切向量}\index{余切向量}(也称为\emph{\red $1-$ 形式}\index{$1-$ 形式})。
\end{defn}
\begin{defn}
对于$f\in C^{\infty}_p$定义$f$在$p$点的\emph{\red 全微分}\index{全微分 }$\dd f|_p$为这样一个余切向量:
\begin{equation*}
(\dd f|_p)(X_p)=X_p(f),\forall X_p\in T_pM
\end{equation*}
\end{defn}
取$p$点的坐标系$(x^i)$,将$\dd(x^i\circ\varphi)|_p$简写为$\dd x^i|_p$,则有:
\begin{prop}
  $\{\dd x^i|_p\}$构成$T^{\ast}_pM$的一组基。
\end{prop}
\begin{proof}
  \begin{equation*}
  (\dd x^i|_p)(\px{j})=(\px{j})(x^i\circ\varphi)=\delta^i_j
  \end{equation*}
所以$\{\dd x^i|_p\}$与$\{\px{i}\}$是对偶基。
\end{proof}
\begin{exa}
考虑$M=O(n)=\{A\in\Real^{n\times n}\mid A'A=I\}$,有$\dim M=\cfrac{n(n-1)}{2}$。
\end{exa}
\begin{proof}
  有$I\in M$,下求$T_IM$。

取曲线$A(t)$过$I$,即$A(0)=I$,则
\begin{equation*}
A(t)^{T}A(t)=I
\end{equation*}
在$t=0$处求导得:
\begin{equation*}
\dot{A}(t)^{T}A(t)+A(t)^{T}\dot{A}(t)=0
\end{equation*}
取$t=0$得知切向量$X=\dot{A}(0)$满足
\begin{equation*}
X^{T}+X=0
\end{equation*}
即$X$为反对称矩阵。所以$T_IM\subset\{X\in\Real^{n\times n}\mid X^{T}+X=0\}$。

反之,对任何反对称矩阵$X$,考虑曲线
\begin{equation*}
A(t)=e^{tX}=I+tX+\cfrac{t^2}{2!}X^2+\cdots
\end{equation*}
则$A(t)\in O(n)$且$\dot{A}(0)=X$,于是$T_IM=\{X\in\Real^{n\times n}\mid X^{T}+X=0\}$。故$\dim M=\dim T_IM=\cfrac{n(n-1)}{2}$。
\end{proof}
\begin{note}
  设$f\colon M\to N$是光滑流形的光滑映射。对$M$上的任一曲线$\gamma(t)$,设$\gamma(0)=p,f(p)=q$,将曲线$f(\gamma(t))$在$q$点的切向量记作$f_{\ast p}(\dot{\gamma}(0))$。
\end{note}
\begin{prop}
  $f_{\ast p}\colon T_PM\to T_qN$是线性映射。
\end{prop}
\begin{proof}
  首先证明$f_{\ast p}$ well-defined,即,对任意两条过$p$的曲线$\gamma(t)$和$\widetilde{\gamma}(t)$,$\gamma(0)=\widetilde{\gamma}(0)=p$,若$\dot{\gamma}(0)=\dot{\widetilde{\gamma}}(0)=q$,则$f(\gamma(t))$和$f(\widetilde{\gamma}(t))$在$q$点切向量相同。

为此,只需验证:
\begin{equation*}
\forall g\in C^{\infty}_q, \localt{\dt g(f(\gamma(t)))}=\localt{\dt g(f(\widetilde{\gamma}(t)))}
\end{equation*}

令$\widetilde{g}=g\circ f\in C^{\infty}_p$,则
\begin{align*}
\localt{\dt g(f(\gamma(t)))}&=\localt{\dt \widetilde{g}(\gamma(t))}=\dot{\gamma}(0)(\widetilde{g})\\
\localt{\dt g(f(\widetilde{\gamma}(t)))}&=\dot{\widetilde{\gamma}}(0)(\widetilde{g})
\end{align*}

再证$f_{\ast p}$是实线性映射,即
\begin{equation*}
f_{\ast p}(X_p+\lambda Y_p)=f_{\ast p}X_p+\lambda f_{\ast p}Y_p
\end{equation*}
因为
\begin{equation*}
f_{\ast p}(X_p)(g)=\localt{\dt g(f(\gamma(t)))}=X_p(g\circ f)
\end{equation*}
所以
\begin{align*}
(f_{\ast p}(X_p+\lambda Y_p))(g)&=(X_p+\lambda Y_p)(g\circ f)\\
                                &=X_p(g\circ f)+\lambda Y_p(g\circ f)\\
                                &=(f_{\ast p}X_p)(g)+\lambda(f_{\ast p}Y_p)(g)
\end{align*}
故$f_{\ast p}$是实线性映射。
\end{proof}
\begin{defn}
$f_{\ast p}\colon T_PM\to T_qN$称为$f$在$p$点的\emph{\red 切映射}\index{切映射 }。$\rank_pf\defeq\rank f_{\ast p}$称为$f$在$p$点的\emph{\red 秩}\index{秩 }。
\end{defn}
设$f\colon M\to N, f(p)=q$,分取$p,q$之坐标系$(\varphi,U),(\psi,V)$使得$f(U)\subset V$,此时,$f$有局部表达式:
\begin{equation*}
\widetilde{f}(x^1,\cdots,x^m)=(f^1(x^1,\cdots,x^m),\cdots,f^n(x^1,\cdots,x^m))
\end{equation*}
即$f=\psi^{-1}\circ\widetilde{f}\circ\varphi$。
$(f_{\ast p})(\px{i})$就是曲线$f(\gamma_i(t))$在$t=0$处的切向量,其中$\gamma_i(t)=\varphi^{-1}(x_0^1,\cdots,x_0^i+t,\cdots,x_0^n)$。
\begin{gather*}
f(\gamma_i(t))=\varphi^{-1}\circ\widetilde{f}(x_0^1,\cdots,x_0^i+t,\cdots,x_0^n)
\gamma^{\alpha}(t)=f^{\alpha}(x_0^1,\cdots,x_0^i+t,\cdots,x_0^n)
\end{gather*}
所以
\begin{equation*}
\dot{\gamma}^{\alpha}(0)=\left.\pfrac{f^{\alpha}}{x^i}(\varphi(p))\pfrac{}{y^{\alpha}}\right|_q
\end{equation*}
\begin{prop}
  $f_{\ast p}\colon T_pM\to T_pN$在基$\{\px{i}\}$和$\{\py{\alpha}\}$下的矩阵为Jacobi矩阵。
\end{prop}
\begin{exa}
  $M=\Real^{n\times n}, N=\Real, f(A)=\det(A)$。
\end{exa}
\begin{proof}
$f_{\ast I}(B)$是曲线$f(I+tB)$在$t=0$处的切向量。
\begin{align*}
f_{\ast I}(B)&=\localt{\dt|I+tB|}\\
\dt|I+tB|&=t^n|B+\frac{1}{t}I|
\end{align*}
若$B$之特征多项式为$F(\lambda)$,则
\begin{equation*}
|I+tB|=(-t)^nF(\frac{1}{t})
\end{equation*}
由于
\begin{equation*}
F(\lambda)=\lambda^n-(\lambda_1+\cdots+\lambda_n)\lambda^{n-1}+\cdots
\end{equation*}
所以
\begin{align*}
|I+tB|&=(-t)^{n}((-\frac{1}{t})^n-\tr(B)(-\frac{1}{t})^{n-1}+\cdots)\\
      &=1+\tr(B) t+o(t)
\end{align*}
所以$f_{\ast I}(B)=\tr(B)$。
\end{proof}
\begin{defn}
对于$f\in C^{\infty}(M,N)$,切映射$f_{\ast}$的共轭映射$f^{\ast}\colon T^{\ast}_qN\to T^{\ast}_pM$ 称为\emph{\red 拉回}\index{拉回}。
\end{defn}
\begin{proof}(验证well-define)
$\forall\alpha\in T^{\ast}_qN,\forall X\in T_pM$,有
\begin{align*}
(f^{\ast}\alpha)(X)&=\alpha(f_{\ast}X)\\
f^{\ast}(\dd y^j)&=\dd f^j=\localp{\pfrac{f^j}{x^i}\dd x^i}
\end{align*}
\end{proof}

\section{子流形}
\subsection*{2012-9-27}
\begin{defn}
  设$f\colon M\to N$是光滑映射,$\forall p\in M$,定义$f$在$p$点的\emph{\red 秩}为$f_{\ast p}$的秩,记作$\rank_p f = \rank f_{\ast p}$。

若$\rank_p f=\dim M$,则称$f$在$p$点\emph{\red 浸入}\index{浸入}(immersion)\index{immersion};

若$\rank_p f=\dim N$,则称$f$在$p$点\emph{\red 淹没}\index{淹没}(submersion)\index{submersion};

若$f$在每一点浸入(淹没),则称$f$是\emph{\red 浸入(淹没)映射}\index{浸入(淹没)映射}。
\end{defn}
\begin{exa}
\mapdes{f}{(-\frac{\pi}{2},\frac{\pi}{2})\times(0,\pi)}{\Real^3}{(u,v)}{(\cos u\cos v,\cos u\sin v,\sin u)}
\end{exa}
\begin{thm}[浸入的局部典范表示]\index{浸入的局部典范表示 }
设$f\colon M\to N$是光滑映射且$f$在$p\in M$上是浸入,则存在$p$的坐标系$(\varphi,U)$ 和$q=f(p)$的坐标系$(\psi,V)$使得
\begin{equation*}
\psi\circ f\circ\varphi^{-1}(x^1,\cdots,x^m)=(x^1,\cdots,x^m,0,\cdots,0)
\end{equation*}
\end{thm}
\begin{proof}
  取$p$之坐标系$(\widetilde{\varphi},\widetilde{U})$和$q$之坐标系$(\widetilde{\psi},\widetilde{V})$,设$f_{\ast p}$在$\{\localp{\pfrac{}{\widetilde{x}^i}}\}$和$\{\local{\pfrac{}{\widetilde{y}^{\alpha}}}{q}\}$的矩阵为
  \begin{equation*}
  (\pfrac{\widetilde{f}^{\alpha}}{\widetilde{x}^i})=(A|\ast)
  \end{equation*}
  由于$\rank(\pfrac{\widetilde{f}^{\alpha}}{\widetilde{x}^i})=\dim M(=m)$,不妨设$A=(\pfrac{\widetilde{f}^{\alpha}}{\widetilde{x}^i})^{1\leqslant\alpha\leqslant m}_{1\leqslant i\leqslant m}$ 可逆。

  定义映射
\mapdes{F}{\widetilde{\varphi}(\widetilde{U})\times\Real^{n-m}}{\Real^n}{(x,z)}{\widetilde{\psi}\circ f\circ\widetilde{\varphi}^{-1}(x)+(0,z)}
则$F$在$(\widetilde{\varphi}(p),0)$处的Jacobi为
\begin{equation*}
\left(
  \begin{array}{cc}
  A&\ast\\
    &I_{n-m}
  \end{array}
\right)
\end{equation*}
这是可逆矩阵。

由反函数定理,存在$(\widetilde{\varphi}(p),0)$之邻域$U\times \widehat{U}$使$F|_{U\times \widehat{U}}$可逆,将反函数记作$F^{-1}$,令$\psi=F^{-1}\circ\widetilde{\psi}\colon F(U\times \widehat{U})\cap V\to\Real^n$,则$(\widetilde{\varphi},U),(\psi,V)$满足定理要求:
\begin{align*}
 &\psi\circ f\circ\varphi^{-1}(x^1,\cdots,x^m)\\
=&F^{-1}\circ(\widetilde{\psi}\circ f\circ\widetilde{\psi}^{-1})(x^1,\cdots,x^m)\\
=&F^{-1}(F(x,0))=(x^1,\cdots,x^m,0,\cdots,0)
\end{align*}
\end{proof}
\begin{exa}
  $f\colon\Real\to\Real^2$定义为
\begin{align*}
f(t)&=(2\cos(t-\frac{\pi}{2}),\sin(2t-\pi))\\
    &=(2\sin t,-\sin 2t)
\end{align*}
计算一下:
\begin{align*}
f'(t)&=(2\cos t,-2\cos 2t)\\
     &=(2\cos t,-2(2\cos^2t-1))\neq0
\end{align*}
可见$f$是浸入。然而$f(\Real)$并不是流形!
\end{exa}
\begin{defn}
  设$f\colon M\to N$是光滑映射且$f$在$p\in M$上是浸入,则称$(M,f)$是$N$的\emph{\red 浸入子流形}\index{浸入子流形 }。若$f$是单射,则称$(M,f)$是$N$的\emph{\red 嵌入子流形}\index{嵌入子流形 }。
\end{defn}
\begin{exa}
$f\colon\Real\to\Real^2$定义为
\begin{equation*}
f(t)=(2\cos(2\arctan t+\frac{\pi}{2}),\sin2(2\arctan t+\frac{\pi}{2}))
\end{equation*}
注意到$f(M)$与$M$同胚,但在$0$点的拓扑与$\Real^2$不同。
\end{exa}
\begin{defn}
  设$f\colon M\to N$是嵌入映射,若$f\colon M\to f(M)$是(微分)同胚(这里$f(M)$具有从$N$诱导的拓扑),则称$f$是\emph{\red 正则子流形}\index{正则子流形 }。
\end{defn}
\begin{defn}
给定光滑流形$N$,设$S\subset N$,若$\forall p\in S$,存在$p$在$N$中的坐标系$(\varphi,U)$使得$\varphi(S\cap U)$是由
\begin{equation*}
x^{k+1}=c^1,x^{k+2}=c^2,\cdots,x^n=c^{n-k}
\end{equation*}
定义的,则称$S$是$N$的一个$k$维\emph{\red 闭子流形}\index{闭子流形 }。

这里由$x^{k+1}=c^1,x^{k+2}=c^2,\cdots,x^n=c^{n-k}$所定义的$\Real^n$的子集称为$\Real^n$上的$k$维\emph{\red 切片}\index{切片}(slice)\index{slice}。
\end{defn}
\begin{thm}
  对于$N$的任一正则子流形$f\colon M\to N$,$f(M)$是$N$的闭子流形。反之,对于$N$之任一闭子流形$S$,$i\colon S\to N$是正则的嵌入。
\end{thm}
\begin{proof}
前一部分,用浸入的局部典范表示;后一部分,只需验证
\begin{equation*}
i\colon(x^1,\cdots,x^k)\to(x^1,\cdots,x^k,c^1,\cdots,c^{n-k})
\end{equation*}
是浸入。
\end{proof}
\begin{thm}\label{淹没原像}
  $f\colon M\to N$是淹没,即
\begin{equation*}
\rank_pf=\dim N, \forall p\in M
\end{equation*}
则$\forall q\in N$,$f^{-1}(q)$是$M$的闭子流形。
\end{thm}
\begin{proof}
先证,$\forall p\in M,f(p)=q$,则存在$p$之坐标系$(\varphi,U)$和$q$之坐标系$(\psi,V)$使得
\begin{equation*}
\psi\circ f\circ\varphi^{-1}(x^1,\cdots,x^m)=(x^1,\cdots,x^n)
\end{equation*}

为此,先取$(\widetilde{\varphi},\widetilde{U})$和$(\widetilde{\psi},\widetilde{V})$使$\widetilde{\varphi}(p)=0$。设
\begin{equation*}
\widetilde{\psi}\circ f\circ\widetilde{\varphi}^{-1}(x^1,\cdots,x^m)=(f^1(x),\cdots,f^n(x))
\end{equation*}

由于$\rank(\pfrac{\widetilde{f}^{\alpha}}{x^i})=\dim N$,不妨设$A=(\pfrac{\widetilde{f}^{\alpha}}{x^i})^{1\leqslant\alpha\leqslant n}_{1\leqslant i\leqslant n}$ 可逆,于是映射
\mapdes{F}{\Real^r}{\Real^n}{(x^1,\cdots,x^n)}{f^1(x^1,\cdots,x^n,0,\cdots,0),\cdots,f^n(x^1,\cdots,x^n,0,\cdots,0)}
在$0$点附近可逆。

令$\psi=F^{-1}\circ\widetilde{\psi}$,则$(\widetilde{\varphi},U),(\psi,V)$满足要求。

再证:$\forall p\in f^{-1}(q)$,存在坐标系$(\varphi,U)$,使得$\varphi(U\cap f^{-1}(q))$是$\Real^m$的切片。

取上一步得到的坐标系,则$\varphi(U\cap f^{-1}(q))=\{(x^1,\cdots,x^m)\in\varphi(U)\mid(x^1,\cdots,x^n)=\psi(q)\}$是切片。
\end{proof}
\begin{exa}[Hopf fibration]
\mapdes{f}{S^3}{S^2}{(z_1,z_2)}{(|z_1|^2-|z_2|^2,2\overline{z_1}z_2)}
对于$(x_1,x_2,x_3,x_4)\in S^3$,$(-x_2,x_1,-x_4,x_3),(-x_3,x_4,x_1,-x_2),(-x_4,-x_3,x_2,x_1)$是其切向量。
\end{exa}
\begin{defn}
  设$f\colon M\to N$是光滑映射且$f$在$p\in M$上是淹没,则称$p$是$f$的\emph{\red 正则点}\index{正则点 }。
对于$q\in N$,若$\forall p\in f^{-1}(q)$都是正则点,则称$q$是$f$的\emph{\red 正则值}\index{正则值 }。
\end{defn}
\begin{cor}[正则值原像定理]\index{正则值原像定理 }\label{正则值原像定理}
设$f\colon M\to N$是光滑映射,$q\in N$是$f$的正则值,则$f^{-1}(q)$是$M$的闭子流形。
\end{cor}
\begin{exa}
  $\begin{cases}
  x^2+y^2+z^2+t^2=1\\
  xy-zt=0
  \end{cases}$
是否是闭子流形?
\end{exa}
\begin{proof}
考虑映射
\mapdes{f}{\Real^4}{\Real^2}{(x,y,z,t)}{(x^2+y^2+z^2+t^2,xy-zt)}
由于
\begin{equation*}
\rank_pf=\rank\left(
                 \begin{array}{cccc}
                   2x & 2y & 2z & 2t \\
                   y & x & -t & -z \\
                 \end{array}
               \right)
               =2
\end{equation*}
故$f^{-1}(1,0)$是闭子流形。
\end{proof}
\begin{thm}[H.Whitney浸入定理]\index{H.Whitney浸入定理}
设$M$是光滑流形,$n\geqslant 2\dim M$,则$\forall f\in C^{\infty}(M,\Real^n), \varepsilon>0$,存在$g\in C^{\infty}(M,\Real^n)$使得
    \begin{enumerate}[1)]
    \setlength{\itemindent}{2ex}
    \item $g$是浸入
    \item $|f(p)-g(p)|<\varepsilon,\forall p\in M$
  \end{enumerate}
\end{thm}
\begin{proof}
Step1,局部上对$f\colon U\to\Real^n$作修改
\begin{equation*}
g(x)=f(x)+Ax, A\in\Real^{m\times m}
\end{equation*}
适当取$A$可使$q$是浸入且$|Ax|<\varepsilon$;

Step2,稳定;

Step3,延拓。
\end{proof}

\section{向量场和流}
\subsection*{2012-9-29}
\begin{defn}
  $M$上切向量的全体,记作$TM$,即$TM=\{u\in T_pM\mid p\in M\}$,称为$M$的\emph{\red 切丛}\index{切丛}。
\end{defn}
\begin{prop}
  若$M$是$m$维光滑流形,则$TM$可赋予微分结构,成为$2m$维光滑流形。
\end{prop}
\begin{proof}
设$\{(\varphi_i,U_i)\mid i\in I\}$为$M$的微分结构。
$\forall p\in M$,设$p\in U_i$的局部坐标为$(x^1,\cdots,x^m)$,$\forall y\in T_pM$,可写成$y=y^i\px{i}$,从而得到从$TM$之开集$\widehat{U}_i=\{y\in T_pM\mid p\in U_i\}$到$\Real^{2m}$之映射:
\mapdes{\widehat{\varphi}_i}{\widehat{U}_i}{\Real^m\times\Real^m}{y}{(x^1,\cdots,x^m,y^1,\cdots,y^m)}
于是:
\begin{enumerate}[1)]
    \setlength{\itemindent}{2ex}
    \item $TM=\bigcup_{i\in I}\widehat{U}_i$
    \item $(\widehat{U}_i,\widehat{\varphi}_i)$与$(\widehat{U}_j,\widehat{\varphi}_j)$光滑相容:
\begin{equation*}
    \widehat{U}_i\cap\widehat{U}_j\neq\varnothing\Longrightarrow U_i\cap U_j\neq\varnothing;
\end{equation*}
        \begin{align*}
   &\widehat{\varphi}_i\circ\widehat{\varphi}_j^{-1}(\widetilde{x}^1,\cdots,\widetilde{x}^m,\widetilde{y}^1,\cdots,\widetilde{y}^m)\\
=&\widehat{\varphi}_i(\widetilde{y}^i\px{i})\\
=&(x^1,\cdots,x^m,\widetilde{y}^i\pfrac{x^1}{\widetilde{x}^i},\cdots,\widetilde{y}^i\pfrac{x^m}{\widetilde{x}^i})
        \end{align*}
  \end{enumerate}
\end{proof}
\begin{defn}
  若映射$X\colon M\to TM$,满足\footnote{i.e. $X$ is a section of $\pi$.}$\pi\circ X=\id\colon M\to M$,则称$X$是$M$上的\emph{\red 向量场}\index{向量场 }。
  若$X$光滑,则称为\emph{\red 光滑向量场}\index{光滑向量场 }。
\end{defn}
\begin{prop}
  $X$是光滑向量场当且仅当$\forall f\in C^{\infty}(M)$,$X(f)$是光滑函数。这里$X(f)(p)=X(p)(f),\forall p\in M$。
\end{prop}
\begin{proof}
  “$\Rightarrow$”:$\forall p\in M$,取$p$点坐标系$(x^i)$,则可设$X(p)=X^i(p)\px{i}$,其中$X^i$是$M$上的函数,局部上:
\begin{align*}
 &\widehat{\varphi}\circ X\circ\varphi^{-1}(x^1,\cdots,x^m)\\
=&(x^1,\cdots,x^m,X^1\circ\varphi^{-1},\cdots,X^m\circ\varphi^{-1})
\end{align*}

于是$X$是光滑向量场$\Leftrightarrow$每个$X^i\circ\varphi^{-1}$是光滑函数。

$\forall f\in C^{\infty}(M)$,
\begin{align*}
X(f)(p)&=X_p(f)\\
       &=X^i(p)\px{i}(f)\\
       &=(X^i\circ\varphi^{-1})\pfrac{f\circ\varphi^{-1}}{x^i}
\end{align*}
是光滑函数。

“$\Leftarrow$”:取$f=x^i$,则$X(x^i)=X^i$光滑,由前面讨论知$X$是光滑向量场。

\end{proof}
\begin{rem}
  以后可将向量场$X$写成
\begin{equation*}
X=X^i\pfrac{}{x^i}
\end{equation*}
\end{rem}
\begin{note}
$M$上的光滑向量场全体记作$\Xf{M}$。
\end{note}
\begin{defn}
  给定光滑向量场$X$,若有曲线$\gamma\colon(-\varepsilon,\varepsilon)\to M$满足$\dot{\gamma}(t)=X(\gamma(t))$,则称$\gamma$为$X$的一条\emph{\red 积分曲线}\index{积分曲线}。
\end{defn}
\begin{prop}
  若$X\in\Xf{M}$,则存在唯一的过$p$的积分曲线$\gamma$,使得$\gamma(0)=p$。
\end{prop}
\begin{proof}
取$p$点的坐标系$(x^i)$,设
\begin{equation*}
X=X^i\pfrac{}{x^i}
\end{equation*}
又设$\gamma(t)=(\gamma^1(t),\cdots,\gamma^m(t))$,则$\dot{\gamma}(t)=\localt{\dot{\gamma}^i(t)\pfrac{}{x^i}}$。要使$\gamma$是$X$的积分曲线,则$\gamma^i$满足
\begin{equation*}
\dot{\gamma}^i(t)=X^i(\gamma(t))
\end{equation*}
由ODE理论,上述ODEs有解。
\end{proof}
\begin{defn}
  设$M$是光滑流形,光滑映射$\psi\colon(-\varepsilon,\varepsilon)\times M\to M$若满足:
\begin{enumerate}[1)]
    \setlength{\itemindent}{2ex}
    \item $\psi_0=\id\colon M\to M$
    \item $\psi_{s}\circ\psi_{t}=\psi_{s+t}, \forall s,t,s+t\in(-\varepsilon,\varepsilon)$
\end{enumerate}
其中$\psi_{t}(p)=\psi(t,p), t\in(-\varepsilon,\varepsilon), p\in M$

则称$\psi$为$M$上的一个\emph{\red (局部)流}\index{局部流 }(flow)\index{flow}
\end{defn}
\begin{prop}
  $\forall X\in\Xf{M}$,存在$M$上的一个流$\psi$,使得
\begin{equation*}
X(p)=\localt{\dt\psi_t(p)}
\end{equation*}
称$\psi$为$X$生成的流。
\end{prop}
\begin{proof}
取过$p$的积分曲线$\gamma(t)$,使$\gamma(0)=p$,令$\psi_t(p)=\gamma_p(t)$,则
\begin{enumerate}[1)]
    \setlength{\itemindent}{2ex}
    \item $\psi_0(p)=p$
    \item $\psi_s\circ\psi_t(q)=\psi_{s+t}(q)$
\end{enumerate}
\end{proof}
\begin{thm}[管状流定理(Flow Box)]\index{管状流定理 }\index{Flow Box}
设$X\in\Xf{M}$,$X_p\neq0$,则存在$p$点的坐标系,使$X=\pfrac{}{x^i}$
\end{thm}
\begin{proof}
  先取坐标系$(y^i)$使$X_p=\py{i}$且$p$点的坐标为$(0,\cdots,0)$。设$X$生成的流为$\psi_t$,定义
\mapdes{\theta}{\Real^m}{U}{(y^1,\cdots,y^m)}{\psi_{y^1}(0,y^2,\cdots,y^m)}
则
\begin{align*}
\theta_{\ast0}\local{\pfrac{}{y^1}}{0}&=X_p=\py{1}\\
\theta_{\ast0}\pfrac{}{y^i}&=\py{i}
\end{align*}

于是$\theta_{\ast0}$可逆,由反函数定理,存在$0$和$p$的领域$\widetilde{U}$,使$\theta$ 在$\widetilde{U}$上可逆。

取$p$之坐标系$(\widetilde{U},\theta^{-1})$则
\begin{align*}
\theta_{\ast}\pfrac{}{y^1}&=\localt{\dt\psi_{y^1+t}(0,y^2,\cdots,y^m)}\\
                          &=X(\psi_{y^1}(0,y^2,\cdots,y^m))
\end{align*}

故$\theta_{\ast}^{-1}X=\pfrac{}{y^1}$。
\end{proof}
\begin{defn}
  设$X_1,X_2,\cdots,X_s\in\Xf{M}$,若$\forall p\in M$,$\rank\{X_1(p),X_2(p),\cdots,X_s(p)\}=k$为常数,则称$X_1,X_2,\cdots,X_s$构成$M$ 上的$k$维\emph{\red 切空间场}\index{切空间场}。
\end{defn}
\begin{defn}
给定$M$上的$k$维切子空间场$\{X_1,X_2,\cdots,X_s\}$,若浸入$f\colon N\to M$满足
$f_{\ast}(T_pN)\subset \spa\{X_1,X_2,\cdots,X_s\}_{f(p)},\forall p\in M$
则称$N$是上述切子空间场的\emph{\red 积分子流形}\index{积分子流形 }。
\end{defn}
\begin{exa}
  $M=\Real^3\backslash\{0\}$中定义
\begin{equation*}
  \begin{cases}
    X_1=z\pfrac{}{z}-y\pfrac{}{y}\\
    X_2=y\pfrac{}{x}+x\pfrac{}{z}\\
    X_3=z\pfrac{}{x}+x\pfrac{}{y}
  \end{cases}
\end{equation*}
则
\begin{equation*}
xX_1-zX_2+yX_3=0\Rightarrow\rank\{X_1,X_2,X_3\}=2
\end{equation*}
因此$\{X_1,X_2,X_3\}$是$2$维切子空间场,而且曲面$N\colon\frac{1}{2}x^2-yz=1$是其积分子流形,$(x,-z,-y)$是其法向量。
对$(2,1,1)\in N$有$X_1=(0,-1,1),X_2=(1,0,2),X_3=(1,2,0)$,法向量为$(2,-1,-1)$。
\end{exa}
\subsection*{总结}
\begin{description}
  \item[$X$生成流$\psi_t$]
  \begin{equation*}
  X(\psi_s(p))=\local{\dt\psi_t(p)}{t=s}
  \end{equation*}
  \item[流$\psi_t$固定一点变成积分曲线]
  \begin{equation*}
  \gamma(t)=\psi_t(p), X(\gamma(t))=X(\psi_t(p))=\dot{\gamma}(t)
  \end{equation*}
  \item[$\psi_t$诱导向量场$X$]
  \begin{equation*}
  \localt{\dt\psi_t(p)}=X(p)
  \end{equation*}
  \item[Flow Box]  $X\in\Xf{M},X(p)\neq0$,则存在$p$点的坐标系$(x^i)$使得$X=\pfrac{}{x^i}$。
\end{description}

\section{Frobenius定理}
\subsection*{2012-10-11}
\begin{exa}
  在$\Real^3$上定义
\begin{equation*}
\begin{cases}
X_1=y\pfrac{}{x}+\pfrac{}{z}\\
X_2=\pfrac{}{y}
\end{cases}
\end{equation*}
$E=\spa\{X-1,X_2\}$是$2$维切子空间场,然而:
\begin{enumerate}[1)]
    \setlength{\itemindent}{2ex}
    \item $E$没有$2$维积分子流形;
    \item $E$有$1$维积分子流形
$\begin{cases}
x=c_1\\
z=x_2
\end{cases}$
\end{enumerate}
\end{exa}
\begin{defn}\label{Lie括号}
  对于$X,Y\in \Xf{M}, [X,Y]\defeq X\circ Y-Y\circ X$也是$M$上的光滑向量场,称为$X$和$Y$ 的\emph{\red Lie括号}\index{Lie括号}或\emph{\red Poisson括号}\index{Poisson括号}。
\end{defn}
\begin{proof}[验证]
将$X$视为泛函$C^{\infty}(M)\to C^{\infty}(M)$
\begin{align*}
X(f)(p)&=X_p(f)\\
       &=X^i(p)\px{i}(f)
\end{align*}
故$X(f)=X^i\pfrac{f}{x^i}$。

设$Y(f)=Y^j\pfrac{f}{x^j}$,则
\begin{align*}
[X,Y](f)&=X(Y(f))-Y(X(f))\\
        &=X(Y^j\pfrac{f}{x^j})-Y(X^i\pfrac{f}{x^i})\\
        &=X^i\pfrac{}{x^i}(Y^j\pfrac{f}{x^j})-Y^j\pfrac{}{x^j}(X^i\pfrac{f}{x^i})\\
        &=X^i\pfrac{Y^j}{x^i}\pfrac{f}{x^j}-Y^j\pfrac{X^i}{x^j}\pfrac{f}{x^i}+X^iY^j\frac{\partial^2{f}}{\partial{x^i}\partial{x^j}}-Y^jX^i\frac{\partial^2{f}}{\partial{x^j}\partial{x^i}}\\
        &=(X^i\pfrac{Y^j}{x^i}-Y^i\pfrac{X^j}{x^i})\pfrac{f}{x^j}
\end{align*}
若令$Z^j=X^i\pfrac{Y^j}{x^i}-Y^i\pfrac{X^j}{x^i}$,$Z=Z^j\pfrac{}{x^j}$,则
\begin{equation*}
[X,Y](f)=Z(f),\forall f\in C^{\infty}(M)
\end{equation*}
即$[X,Y]=Z\in\Xf{M}$。
\end{proof}
\begin{defn}
  $X$是向量场,$f\colon N\to M$为浸入,若$\forall q\in N$,$X(f(q))\in f_{\ast}(T_qN)$则称$X$与$f$\emph{\red 相切}\index{相切}。
\end{defn}
\begin{exa}
  在$\Real^3$中,$X=-y\pfrac{}{x}+x\pfrac{}{y}$与$S^2$相切。
\end{exa}
\begin{prop}
  若$X,Y\in\Xf{M}$都与某个浸入子流形$f\colon N\to M$相切,则$[X,Y]$也与$f$相切。
\end{prop}
\begin{proof}
  $\forall q\in N$,取$q$点坐标系$(y^i)$及$f(q)$点坐标系$(x^i)$使得
\begin{equation*}
f(y^1,y^2,\cdots,y^n)=(x^1,x^2,\cdots,x^n,0,\cdots,0)
\end{equation*}

又设$X=X^i\pfrac{}{x^i},Y=Y^j\pfrac{}{x^j}$
\begin{equation*}
f_{\ast}(T_qN)=\spa\{\pfrac{}{x^1},\cdots,\pfrac{}{x^n}\}_{f(q)}
\end{equation*}

由于$X,Y$与$f$相切,故$X^k(f(q))=0, Y^k)f)q))=0, n<k\leqslant m$
\begin{equation*}
[X,Y]=(X^i\pfrac{Y^j}{x^i}-Y^i\pfrac{X^j}{x^i})\pfrac{}{x^j}
\end{equation*}

由于
\begin{equation*}
(X^i\pfrac{Y^k}{x^i}-Y^i\pfrac{X^k}{x^i})(f(q))=0, n<k\leqslant m
\end{equation*}
所以$[X,Y]$也与$f$相切。
\end{proof}
\begin{prop}
  若$X,Y\in\Xf{M}$且$X$生成的流为$\psi t$,则
\begin{equation*}
[X,Y]=\lim_{t\to0}\frac{Y-\psi_{t\ast}Y}{t}
\end{equation*}
即
\begin{equation*}
[X,Y]_p=\lim_{t\to0}\frac{Y_p-\psi_{t\ast}(Y_{\psi_t(p)})}{t}
\end{equation*}
\end{prop}
\begin{proof}
  $\forall f\in C^{\infty}_p$,设
\begin{equation*}
f(\psi_t(p))=c_0(p)+c_1(p)t+c_2(p)t^2+o(t^2)
\end{equation*}
于是
\begin{align*}
c_0(p)&=f(p)\\
c_1(p)&=\localt{\dt f(\psi_t(p))}\\
      &=X_p(f)=X(f)(p)
\end{align*}
\begin{align*}
  &(Y_p-\psi_{t\ast}(Y_{\psi_t(p)}))(f)\\
=&Y_p(f)-Y_{\psi_t(p)}(f\circ\psi_t)\\
=&Y_p(f)-Y_{\psi_t(p)}(f+tX(f)+o(t))\\
=&Y_p(f)-Y_{\psi_t(p)}(f)-tY_{\psi_t(p)}(X(f))-Y_{\psi_t(p)}(o(t))\\
\end{align*}
于是
\begin{align*}
  &\lim_{t\to0}\frac{Y_p-\psi_{t\ast}(Y_{\psi_t(p)})}{t}(f)\\
=&\lim_{t\to0}\frac{Y(f)(p)-Y(f)(\psi_t(p))}{t}-Y_p(X(f))\\
=&X_p(Y(f))-Y_p(X(f))\\
=&[X,Y]_p(f)
\end{align*}
\end{proof}
\begin{prop}
设$X\in\Xf{M}$生成的流为$\psi_t$,$\varphi\colon M\to M$是光滑同胚,则$\varphi_{\ast}X$生成的流是
\begin{equation*}
  \varphi\circ\psi_t\circ\varphi^{-1}
\end{equation*}
\end{prop}
\begin{proof}
\begin{enumerate}[1)]
    \setlength{\itemindent}{2ex}
    \item $\varphi\circ\psi_t\circ\varphi^{-1}$是流。
    \item $\varphi\circ\psi_t\circ\varphi^{-1}$诱导的向量场是$\varphi_{\ast}X$:
  \begin{align*}
     &\localt{\dt\varphi\circ\psi_t\circ\varphi^{-1}(p)}\\
  =&\varphi_{\ast}(\localt{\dt\psi_t(\varphi^{-1}(p))})\\
  =&\varphi_{\ast}X_{\varphi^{-1}(p)}
  \end{align*}
\end{enumerate}
\end{proof}
\begin{cor}
  $\varphi_{\ast}X=X\Longleftrightarrow\varphi\circ\psi_t\circ\varphi^{-1}=\psi_t$
\end{cor}
\begin{prop}
  设$X,Y\in\Xf{M}$生成的流分别为$\psi_t,\phi_t$,则$[X,Y]=0$的充要条件是$\psi_t\circ\phi_t=\phi_t\circ\psi_t$。
\end{prop}
\begin{proof}
\begin{align*}
  [X,Y]=0&\Longleftrightarrow\lim_{t\to0}\frac{Y-\psi_{t\ast}Y}{t}=0\\
       &\Longleftrightarrow Y=\psi_{t\ast}Y\\
       &\Longleftrightarrow \phi_s\circ\psi_t=\psi_t\circ\phi_s
\end{align*}
\end{proof}
\begin{prop}[Lie括号性质]
\begin{enumerate}[1)]
    \setlength{\itemindent}{2ex}
    \item $[X,Y]=–[Y,X]$;
    \item $[X+Y,Z]=[X,Z]+[Y,Z]$;
    \item $[X,fY]=X(f)Y+f[X,Y]$;
    \item $[X,[Y,Z]]+[Y,[Z,X]]+[Z,[X,Y]]$
\end{enumerate}
\end{prop}
\begin{proof}
  \begin{enumerate}[1)]
    \setlength{\itemindent}{2ex}
    \item trivial;
    \item trivial;
    \item \begin{align*}
[X,fY](g)&=X(fY(g))-fY(X(g))\\
         &=X(f)Y(g)+fX(Y(g))-fY(X(g))\\
         &=X(f)Y(g)+f[X,Y](g)
          \end{align*}
    \item \begin{align*}
[X,[Y,Z]](f)&=X([Y,Z](f))-[Y,Z](Xf)\\
            &=XYZ(f)-XZY(f)-YZX(f)+ZYX(f)
                                         \end{align*}
  \end{enumerate}
\end{proof}
\begin{thm}[同步管状流定理]\index{同步管状流定理 }
$X_1,X_2,\cdots,X_r\in\Xf{M}$,且$\rank\{X_1,X_2,\cdots,X_r\}=r, [X_i,X_j]=0$,则$\forall p\in M$,存在坐标系$(x^i)$,使得$X_i=\pfrac{}{x^i},1\leqslant i\leqslant r$。
\end{thm}
\begin{proof}
  任取$p$点之坐标系$(u^i)$使$p$点坐标为$(0,\cdots,0)$。考虑
\mapdes{\theta}{\Real^m}{U}{(x^1,\cdots,x^m)}{(\varphi_1)_{x^1}\circ(\varphi_2)_{x^2}\cdots\circ(\varphi_r)_{x^r}(0,\cdots,0,x^{r+1},\cdots,x^m)}
其中$(\varphi_i)_{t_i}$是$X_i$生成的流。

下面验证$\theta_{\ast}\pfrac{}{x^i}=X_i$:
\begin{align*}
\theta_{\ast u}\pfrac{}{x^i}&=\localt{\dt\theta(x^1,\cdots,x^i+t,\cdots,x^m)}\\
                            &=\localt{\dt(\varphi_i)_{x^i+t}\circ(\varphi_1)_{x^1}\circ\cdots\circ\widehat{(\varphi_i)_{x^i}}\cdots\circ(\varphi_r)_{x^r}}\\
                            &=\localt{\dt(\varphi_i)_t\circ(\varphi_1)_{x^1}\circ(\varphi_2)_{x^2}\cdots\circ(\varphi_r)_{x^r}}\\
                            &=X(\theta(u)), 1\leqslant i\leqslant r\\
\theta_{\ast 0}\pfrac{}{x^k}&=\dt\theta(0,\cdots,\widehat{t},\cdots,0)\\
                            &=\localt{\dt(0,\cdots,t,\cdots,0)}\\
                            &=\localp{\pfrac{}{u^k}}, r<k\leqslant m
\end{align*}
只要取$(U,u^i)$s使$\localp{X_i},\localp{\pfrac{}{u^i}}$线性无关即可。
\end{proof}
\begin{thm}
  设$E=\spa\{X_1,\cdots,X_r\}$是$M$上的$r$维子空间,且满足{\red Frobenius条件}\index{Frobenius条件}
\begin{equation*}
[X_i,X_j]_p\in E_p
\end{equation*}
则过任一点$p\in M$,有唯一的极大积分子流形$f\colon N\to M$,且$\dim N=r$。
\end{thm}
换言之,$\forall p\in M$,存在坐标系$(x^i)$使得
\begin{equation*}
E=\spa\{\pfrac{}{x^1},\cdots,\pfrac{}{x^r}\}
\end{equation*}
这时由$x^{r+1}=c^{r+1},\cdots,x^m=c^m$所定义的闭子流形与$E$相切。
\begin{proof}
  先取坐标系$(U,u^i)$设
\begin{equation*}
X_j=X_j^i\pfrac{}{u^i}
\end{equation*}
\begin{equation*}
\left(
  \begin{array}{ccc}
    X_1^1 & \cdots & X_1^m \\
    \cdots & \cdots & \cdots \\
    X_r^1 & \cdots & X_r^m \\
  \end{array}
\right)
\stackrel{\text{初等行变换}}{\Longrightarrow}
\left(
  \begin{array}{ccc}
    1 & & 0 \\
     & \ddots &  \\
    0 & & 1 \\
  \end{array}
\middle|
\text{\Huge$\ast$}
\right)
\end{equation*}
即可找到$Y_1,\cdots,Y_r\in\Xf{M}$使得$\spa\{Y_1,\cdots,Y_r\}=E$且$Y_i=\pfrac{}{x^i}+Y_i^k\pfrac{}{x^k}, r+1\leqslant k\leqslant m$。

这时
\begin{align*}
[Y_i,Y_j]&=[A_i^sX_s,A_j^tX_t]\\
         &=A_i^sX_s(A_j^t)X_t-A_j^tX_t(A_i^s)X_s+A_i^sA_j^t[X_s,X_t]\in E\\
[\pfrac{}{x^i}+Y_i^k\pfrac{}{x^k},\pfrac{}{x^j}+Y_j^l\pfrac{}{x^l}]&=Y_i(Y_j^l)\pfrac{}{x^l}-Y_j(Y_i^k)\pfrac{}{x^k}\in E
\end{align*}

故$[Y_i,Y_j]=0 i,j\leqslant r$,故可用同步管状流。
\end{proof}
向量场$X\colon C^{\infty}(M)\to C^{\infty}(M)$
\begin{equation*}
[X,Y]=X\circ Y-Y\circ X
\end{equation*}
$X,Y$都与子流形$N\subset M$相切,则$[X,Y]$与$N$相切。
\begin{equation*}
[X,Y]=\lim_{t\to0}\frac{Y-\varphi_{t\ast}Y}{t}
\end{equation*}

切子空间场$E$
\begin{align*}
\pi&\colon G_k(TM)=\{P\subset T_xM\mid\dim P=k,x\in M\}\longrightarrow M\\
E&\colon M\longrightarrow G_k(TM)
\end{align*}
\begin{exa}
  $S^2$上不存在$1$维切子空间场,即不存在处处非零的光滑向量场。
\end{exa}
\begin{exa}[Clifford代数]
若$S^n$上存在$1,2,\cdots,k$维光滑切子空间场,求$k$的最大值。
\begin{align*}
2\mid^n&\Longrightarrow k=0\\
n=3&\Longrightarrow k=3\\
n\equiv3\mod4&\Longrightarrow  k\geqslant3\\
n=7&\Longrightarrow k=7\\
n=5& k=1
\end{align*}
具体的:
\begin{align*}
S^1&\rightsquigarrow \text{复数}\\
S^3&\rightsquigarrow \text{四元数}\\
S^7&\rightsquigarrow \text{八元数}
\end{align*}
\end{exa}
\begin{defn}[Frobenius条件]\index{Frobenius条件}
\begin{equation*}
\forall X,Y\in\Xf{M}, X_p,Y_p\in E_p(X,Y\in E)\Longrightarrow[X,Y]\in E
\end{equation*}
\end{defn}
\begin{rem}
  等价于“取$E$的一组基$X_1,X_2,\cdots,X_k$时$[X_i,X_j]\in E$”。
\end{rem}
\begin{thm}[Frobenius定理]\index{Frobenius定理}
若$k$维切子空间场$E$满足Frobenius条件,则$\forall p\in M$,存在唯一的极大积分子流形$N$经过$p$。
\end{thm}
\begin{proof}
  设$X_1,\cdots,X_k$是$E$的一组基,则
\begin{equation*}
X_i=X^j_i\pfrac{}{x^i}
\end{equation*}
\begin{equation*}
\left(
  \begin{array}{ccc}
    X_1^1 & \cdots & X_1^n \\
    \cdots & \cdots & \cdots \\
    X_k^1 & \cdots & X_k^n \\
  \end{array}
\right)
\stackrel{\text{初等行变换}}{\Longrightarrow}
\left(
  \begin{array}{ccc}
    1 & & 0 \\
     & \ddots &  \\
    0 & & 1 \\
  \end{array}
\middle|
\text{\Huge$\ast$}
\right)
\end{equation*}
得到另一组基$Y_i=\pfrac{}{u^i}+Y^{\alpha}_i\pfrac{}{u^{\alpha}}, k+1\leqslant\alpha\leqslant n$
\begin{equation*}
[Y_i,Y_j]=Y_i(Y^{beta}_j)\pfrac{}{u^{\beta}}-Y_j(Y^{\alpha}_i)\pfrac{}{u^{\alpha}}
\end{equation*}

由Frobenius条件,
\begin{align*}
 [Y_i,Y_j]&=C^l_{ij}Y_l\\
         &=C^l_{ij}(\pfrac{}{u^l}+Y^{\gamma}_l\pfrac{}{\gamma})
\end{align*}
比较得$C^l_{ij}=0$,于是$[Y_i,Y_j]=0$。

由同步管状流,存在坐标系$(x^i)$使得
\begin{equation*}
Y_i=\pfrac{}{x^i}
\end{equation*}
于是$Y_1,Y_2,\cdots,Y_k$都与切片$x^{k+1}=cst,\cdots,x^n=cst$相切。
\end{proof}

\section{外代数}
\subsection*{2012-10-18}
\begin{defn}
  设$V,W$是实线性空间,$V^{\ast},W^{\ast}$是其对偶空间。
对于$\alpha\in V^{\ast}, \beta\in W^{\ast}$,定义
\mapdes{\alpha\otimes\beta}{V\times W}{\Real}{(v,w)}{\alpha(v)\beta(w)}
可见$\alpha\otimes\beta$是双线性函数。

以$V^{\ast}\otimes W^{\ast}$记所有形如$\alpha\otimes\beta$所张成的线性空间,称为$V^{\ast}$和$W^{\ast}$的\emph{\red 张量积}\index{张量积 }。
\begin{enumerate}[1)]
    \setlength{\itemindent}{2ex}
    \item $V^{\ast}\otimes W^{\ast}$的元素就是$V\times W$上的双线性函数;
    \item 设$\alpha^1,\cdots,\alpha^m$和$\beta^1,\cdots,\beta^n$分别是$V^{\ast}$和$W^{\ast}$的一组基,则$\{\alpha^i\otimes\beta^j\}$是$V^{\ast}\otimes W^{\ast}$ 的一组基。
\end{enumerate}
同理可定义$V\otimes W$。
\end{defn}
\begin{defn}
  $T^r_s(V)=\underbrace{V\otimes\cdots\otimes V}_{r}\otimes \underbrace{V^{\ast}\otimes\cdots\otimes V^{\ast}}_{s}$中的元素称为\emph{\red $(r,s)$ 型张量}\index{$(r,s)$型张量 }。
\end{defn}
\begin{defn}
  对于$(r,0)$型或$(0,s)$型张量$T$,若
\begin{equation*}
T(\theta^1,\cdots,\theta^r)=T(\theta^{\sigma(1)},\cdots,\theta^{\sigma(r)})
\end{equation*}
则称$T$是\emph{\red 对称的}\index{对称张量};
\begin{equation*}
T(\theta^1,\cdots,\theta^r)=\sgn(\sigma)T(\theta^{\sigma(1)},\cdots,\theta^{\sigma(r)})
\end{equation*}
则称$T$是\emph{\red 反对称的}\index{反对称张量 }。
\end{defn}
\begin{align*}
V\otimes V^{\ast}&\cong\End(V)\\
f&=f^i_jv_i\otimes\alpha^j\\
f(v_j)&=f^j_iv_j
\end{align*}
\begin{note}
  记$T^r_0(V)$中反对称张量全体为$A^r(V)$。
\end{note}
\begin{defn}
$\forall \xi\in A^k(V), \eta\in A^l(V)$,\emph{\red 外积}\index{外积}定义为
\begin{equation*}
\xi\wedge\eta\defeq\frac{1}{(k+l)!}\sum_{\sigma}\sgn(\sigma)\sigma(\xi\otimes\eta)
\end{equation*}
\end{defn}
\begin{prop}
  外积满足以下运算律
\begin{enumerate}[1)]
    \setlength{\itemindent}{2ex}
    \item 分配律
    \begin{align*}
      (\xi_1+\xi_2)\wedge\eta&=\xi_1\wedge\eta+\xi_2\wedge\eta\\
      \xi\wedge(\eta_1+\eta_2)&=\xi\wedge\eta_1+\xi\wedge\eta_2
    \end{align*}
    \item 反交换律
    \begin{equation*}
      \xi\wedge\eta=(-1)^{kl}\eta\wedge\xi
    \end{equation*}
    \item 结合律
    \begin{equation*}
      (\xi\wedge\eta)\wedge\zeta=\xi\wedge(\eta\wedge\zeta)
    \end{equation*}
\end{enumerate}
\end{prop}
\begin{thm}
  矢量$v_1,\cdots,v_r\in V$线性相关的充要条件是$v_1\wedge\cdots\wedge v_r=0$。
\end{thm}
\begin{thm}[Cartan 引理]\index{Cartan 引理}
设$v_1,\cdots,v_r;w^1,\cdots,w^r$是$V$中两组矢量,使得
\begin{equation*}
v_i\wedge w^i=0
\end{equation*}

如果$v_1,\cdots,v_r$线性无关,则存在满足$h^{ij}=h^{ji}$的$h^{ij}$使得
\begin{equation*}
w^j=h^{ij}v_i, 1\leqslant j\leqslant r
\end{equation*}
\end{thm}

\section{外微分}
\subsection*{2012-10-25}
\begin{defn}
  $A^r(M)=\bigcup\limits_{p\in M}A^r(T^{\ast}_pM)$称为$M$上的$r$次\emph{\red 外形式丛}\index{外形式丛}。
\end{defn}
$\forall p\in M$,取坐标系$(x^i)$,$\dd x^1,\cdots,\dd x^m$是$T^{\ast}_pM$的一组基,故$\forall \omega\in A^r(T^{\ast}_pM)$可写为
\begin{equation*}
\omega=\frac{1}{r!}\omega_{i_1\cdots i_r}\dd x^{i_1}\wedge\cdots\wedge\dd x^{i_r}
\end{equation*}
其中$\omega_{i_1\cdots i_r}$关于任意两个角标反对称。于是$(x^i,\omega_{i_1\cdots i_r})$可构成$A^r(M)$上的坐标系。容易验证,这些坐标系是光滑相容的,从而$A^r(M)$关于这一微分结构成为光滑流形。
\begin{defn}
  若$\omega\colon M\to A^r(M)$是光滑映射且满足$\pi\circ\omega=\id$,这里$\pi\colon A^r(M)\to M$是自然投影,则称$\omega$是一个$r$次\emph{\red 外微分式}\index{外微分式},
  也称为\emph{\red $r$形式}\index{$r$形式}。
\end{defn}
\begin{lem}
  局部上,$\omega$是$r$形式等价于
\begin{equation*}
\omega(p)=\frac{1}{r!}\omega_{i_1\cdots i_r}(p)\dd x^{i_1}\wedge\cdots\wedge\dd x^{i_r}
\end{equation*}
其中系数$\omega_{i_1\cdots i_r}$是光滑函数。
\end{lem}
\begin{note}
  记$A(M)$为$M$上所有$0$形式、$1$形式、$\cdots$、$m$形式的全体。
\end{note}
\begin{thm}
  存在唯一的算子$\dd\colon A(M)\to A(M)$,满足
\begin{enumerate}[1)]
    \setlength{\itemindent}{2ex}
    \item $\dd(A^r(M))\subset A^{r+1}(M)$;
    \item $\dd$是实线性的;
    \item 对任意$r$形式$\omega_1$,$k$形式$\omega_2$
    \begin{equation*}
    \dd(\omega_1\wedge\omega_2)=\dd\omega_1\wedge\omega_2+(-1)^r\omega_1\wedge\dd\omega_2
    \end{equation*}
    \item 对$0$形式$f\in C^{\infty}(M)$,$\dd f$是$f$的全微分且$\dd(\dd f)=0$。
\end{enumerate}
称$\dd$为{\red 外微分算子}\index{外微分算子 }。
\end{thm}
\begin{proof}
  先证明这样的$\dd$若存在,则具有\emph{局部性},即
\begin{equation*}
\local{\omega_1}{U}=\local{\omega_2}{U}\Longrightarrow\local{\dd\omega_1}{U}=\local{\dd\omega_2}{U}
\end{equation*}
由于$\dd$具有实线性,只需证明$\local{\omega}{U}=0\Rightarrow\local{\dd\omega}{U}$。

$\forall p\in U$,取邻域$\widehat{U}\subset U$,并取光滑函数$f$,使$\local{f}{\widehat{U}}=1, \local{f}{M\backslash U}=0$。则
\begin{align*}
f(\omega)=0&\Longrightarrow\dd(f(\omega))=0\\
                   &\Longrightarrow\dd f\wedge(\omega)+f(\dd\omega)=0
\end{align*}
由于在$p$点
\begin{equation*}
\localp{\dd f}=0, \localp{f}=1
\end{equation*}
故$\localp{\dd\omega}=0$。

再证明\emph{局部存在唯一性}

\emph{存在性}:
对单项式$\omega=f(x)\dd x^1\wedge\cdots\wedge\dd x^r$,规定$\dd\omega=\dd f\wedge\dd x^1\wedge\cdots\wedge\dd x^r$。由实线性,得
\begin{align*}
  \omega&=\frac{1}{r!}\omega_{i_1\cdots i_r}\dd x^{i_1}\wedge\cdots\wedge\dd x^{i_r}\\
  \dd\omega&=\frac{1}{r!}\dd\omega_{i_1\cdots i_r}\wedge\dd x^{i_1}\wedge\cdots\wedge\dd x^{i_r}
\end{align*}

验证$3)$:
\begin{align*}
  \omega_1&=\frac{1}{r!}\omega_I\dd x^I\\
  \omega_2&=\frac{1}{k!}\eta_J\dd x^J
\end{align*}
\begin{align*}
  &\omega_1\wedge\omega_2=\frac{1}{r!}\frac{1}{k!}\omega_I\eta_J\dd x^I\wedge\dd x^J\\
\Longrightarrow&\dd(\omega_1\wedge\omega_2)=\frac{1}{r!}\frac{1}{k!}\dd(\omega_I\eta_J)\wedge\dd x^I\wedge\dd x^J
\end{align*}
其中$\dd(\omega_I\eta_J)=\omega_I\dd(\eta_J)+\eta_J\dd(\omega_I)$。

故
\begin{align*}
\dd(\omega_1\wedge\omega_2)&=\frac{1}{r!}\frac{1}{k!}(\omega_I\dd(\eta_J)\wedge\dd x^I\wedge\dd x^J+\eta_J\dd(\omega_I)\wedge\dd x^I\wedge\dd x^J)\\
                           &=\dd\omega_1\wedge\omega_2+(-1)^r\omega_1\wedge\dd\omega_2
\end{align*}

验证$4)$:
\begin{align*}
\dd f&=\pfrac{f}{x^i}\dd x^i\\
\dd(\dd f)&=\dd(\pfrac{f}{x^i})\wedge\dd x^i\\
          &=\frac{\partial^2f}{\partial x^i\partial x^j}\dd x^j\wedge\dd x^i\\
          &=\sum_{j<i}(\frac{\partial^2f}{\partial x^i\partial x^j}-\frac{\partial^2f}{\partial x^j\partial x^i})\dd x^j\wedge\dd x^i\\
          &=0
\end{align*}

\emph{唯一性}:对$r$用数学归纳法证明$\dd(\dd x^I)=0$,于是
\begin{equation*}
\dd(f\dd x^I)=\dd f\wedge\dd x^I+(-1)^{0}f\dd(\dd x^I)=\dd f\wedge\dd x^I
\end{equation*}

最后证明\emph{整体的存在唯一性}:
$\forall p\in U\cap W$,由于
\begin{equation*}
\local{\dd(\omega_U)}{U\cap W}=\dd(\omega_{U\cap W})=\local{\dd(\omega_W)}{U\cap W}
\end{equation*}
故$\dd$在$U\cap W$上是一致的。
\end{proof}
\begin{thm}[Poinc\'{a}re引理]\index{Poinc\'{a}re引理}
  外微分算子$\dd$具有性质$\dd\circ\dd=0$。
\end{thm}
\begin{proof}
  由于实线性性质,只需证明对于单项式$\omega=f\dd x^I$有$\dd(\dd\omega)=0$即可。而
\begin{align*}
 \dd(\dd\omega)&=\dd(\dd f\wedge\dd x^I)\\
              &=\dd(\dd f)\wedge\dd x^I-\dd f\wedge\dd(\dd x^I)\\
              &=0
\end{align*}
\end{proof}
\begin{exa}
  对于$f\in C^{\infty}(\Real^3)$
\begin{equation*}
\dd f=\pfrac{f}{x}\dd x+\pfrac{f}{y}\dd y+\pfrac{f}{z}\dd z
\end{equation*}
$\grad f=(\pfrac{f}{x},\pfrac{f}{y},\pfrac{f}{z})$通常称为$f$的{\red 梯度}\index{梯度}\footnote{有一个定理:函数$F$在一点的梯度与其过该点的\emph{\red 水平集}\index{水平集 }$\{x\mid F(x)=cst\}$垂直,故梯度可以用来求法向量。}。
\end{exa}
\begin{exa}
  对于$1$形式
\begin{align*}
\omega&=P\dd x+Q\dd y+R\dd z\\
\dd\omega&=\dd P\wedge\dd x+\dd Q\wedge\dd y+\dd R\wedge\dd z\\
         &=(\pfrac{Q}{x}-\pfrac{P}{y})\dd x\wedge\dd y\\
         &+(\pfrac{R}{y}-\pfrac{Q}{z})\dd y\wedge\dd z\\
         &+(\pfrac{P}{z}-\pfrac{R}{x})\dd z\wedge\dd x
\end{align*}
通常,把$(\pfrac{Q}{x}-\pfrac{P}{y},\pfrac{R}{y}-\pfrac{Q}{z},\pfrac{P}{z}-\pfrac{R}{x})$称为向量场$X=(P,Q,R)$的{\red 旋度场}\index{旋度},记为$\curl X$。
\end{exa}
\begin{exa}
  对于$2$形式
\begin{align*}
\omega&=P\dd y\wedge\dd z+Q\dd z\wedge\dd x+R\dd x\wedge\dd y\\
\dd\omega&=(\pfrac{P}{x}+\pfrac{Q}{y}+\pfrac{R}{z})\dd x\wedge\\ y\wedge\dd z
\end{align*}
通常,把$\pfrac{P}{x}+\pfrac{Q}{y}+\pfrac{R}{z}$称为$X=(P,Q,R)$的{\red 散度}\index{散度},记为$\Div X$。
\end{exa}
由Poinc\'{a}re引理,得
\begin{align*}
\curl(\grad f)=0,& \forall f\in C^{\infty}(\Real^3)\\
\Div(\curl X)=0,& \forall X\in \Xf{\Real^3}
\end{align*}
\begin{thm}[外微分求值公式]\index{外微分求值公式 }
设$\omega$是$1$形式,$X,Y\in\Xf{M}$,则
\begin{equation*}
\dd\omega(X,Y)=X(\omega(Y))-Y(\omega(X))-\omega[X,Y]
\end{equation*}
\end{thm}
\begin{proof}
  \begin{align*}
\omega&=f\dd g\\
\dd\omega&=\dd f\wedge\dd g\\
\dd\omega(X,Y)&=(\dd f\wedge\dd g)(X,Y)\\
              &=
                \begin{vmatrix}
                 \dd f(X) & \dd g(X) \\
                 \dd f(Y) & \dd g(Y) \\
                \end{vmatrix}\\
              &=X(f)Y(g)-X(g)Y(f)
   \end{align*}

   另一方面,
\begin{align*}
\omega(Y)&=f\dd g(Y)=fY(g)\\
\omega(X)&=fX(g)
\end{align*}
   所以
   \begin{align*}
   &X(\omega(Y))-Y(\omega(X))-\omega[X,Y]\\
  =&X(fY(g))-Y(fX(g))-f(X(Y(g))-Y(X(g)))\\
  =&X(f)Y(g)-X(g)Y(f)
   \end{align*}
\end{proof}

\section{外微分(续)}
\subsection*{2012-11-1}
\begin{lem}\label{lem8.1}
  设$\{X_1,\cdots,X_m\}$是$m$维流形$M$上的一个局部标架场(自动满足Frobenius条件,即存在光滑函数$C^i_{jk}$,使$[X_j,X_k]=C^i_{jk}X_i$),又设$\omega^1,\cdots,\omega^m$是与$X_1,\cdots,X_m$对偶的余标架场,则
  \begin{equation*}
    \dd\omega^i=-\frac{1}{2}C^i_{jk}\omega^j\wedge\omega^k
  \end{equation*}
\end{lem}
\begin{proof}
  一方面:
\begin{align*}
\dd\omega^i(X_p,X_q)&=X_p(\omega^i(X_q))-X_q(\omega^i(X_p))-\omega^i[X_p,X_q]\\
                    &=-\omega^i[X_p,X_q]\\
                    &=-\omega^i(C^r_{pq}X_r)\\
                    &=-C^r_{pq}\omega^iX_r=-C^i_{pq}
\end{align*}

另一方面:
\begin{align*}
  &-\frac{1}{2}C^i_{jk}\omega^j\wedge\omega^k(X_p,X_q)\\
 =&-\frac{1}{2}C^i_{jk}
                \begin{vmatrix}
                 \dd \omega^j(X_p) & \dd \omega^k(X_p) \\
                 \dd \omega^j(X_q) & \dd \omega^k(X_q) \\
                \end{vmatrix}\\
 =&-\frac{1}{2}C^i_{jk}(\delta^j_p\delta^k_q-\delta^k_p\delta^j_q)\\
 =&-\frac{1}{2}(C^i_{pq}-C^i_{qp})=-C^i_{pq}
\end{align*}
\end{proof}
\begin{defn}
  设$\{X_1,\cdots,X_r\}$是$r$维切子空间场,将它扩充为一个局部标架场$X_1,\cdots,X_m$并取与之对偶的余标架场$\omega^1,\cdots,\omega^m$。若$1$形式$\omega^{r+1},\cdots,\omega^m$线性无关,且对于$r$维子空间场$E$,满足
\begin{equation*}
\omega^{\alpha}(X)=0,\forall X\in E
\end{equation*}
则称$\omega^{r+1},\cdots,\omega^m$是$E$的\emph{\red 定义方程}\index{定义方程}(\emph{\red 零化子}\index{零化子 })。
\end{defn}
\begin{defn}
  线性无关的$1$形式$\omega^{r+1},\cdots,\omega^m$若满足
\begin{equation*}
\dd\omega^{\alpha}\equiv0\mod\omega^{r+1},\cdots,\omega^m, \forall r+1\leqslant\alpha\leqslant m
\end{equation*}
即存在$1$形式$\theta^{\alpha}_{\beta}$使得
\begin{equation*}
\dd\omega^{\alpha}=\theta^{\alpha}_{\beta}\wedge\omega^{\beta}
\end{equation*}
则称$\omega^{r+1},\cdots,\omega^m$满足\emph{\red Frobenius条件}\index{Frobenius条件}。
\end{defn}
\begin{thm}[Frobenius定理]\index{Frobenius定理}
设$r$维切子空间场$E$的定义方程$\omega^{r+1},\cdots,\omega^m$满足Frobenius条件,则$E$是完全可积的,即存在局部坐标系$(x^i)$使
\begin{equation*}
E=\{\pfrac{}{x^1},\cdots,\pfrac{}{x^r}\}
\end{equation*}
或者写成外微分式:
\begin{equation*}
\{\omega^{r+1},\cdots,\omega^m\}=\{\dd x^{r+1},\cdots,\dd x^m\}
\end{equation*}
\end{thm}
\begin{proof}
  不妨设$X_1,\cdots,X_m$与$\omega^{r+1},\cdots,\omega^m$对偶,只需验证$E=\{X_1,\cdots,X_m\}$满足(向量场版)Frobenius条件。

设$[X_a,X_b]=C^i_{ab}X_i$,则Frobenius条件等价于
\begin{equation*}
C^{\alpha}_{jk}=0,\forall r+1\leqslant\alpha\leqslant m, 1\leqslant j,k\leqslant r
\end{equation*}

由引理\ref{lem8.1},
\begin{equation*}
\dd\omega^c=-\frac{1}{2}C^c_{ab}\omega^a\wedge\omega^b,\forall a,b,c
\end{equation*}

于是
\begin{equation*}
\dd\omega^{\alpha}=-\frac{1}{2}C^{\alpha}_{ij}\omega^i\wedge\omega^j-C^{\alpha}_{\beta i}\omega^{\beta}\wedge\omega^i-\frac{1}{2}C^{\alpha}_{\beta\gamma}\omega^{\beta}\wedge\omega^{\gamma}
\end{equation*}

与Frobenius条件比较,得
\begin{equation*}
\dd\omega^{\alpha}=\theta^{\alpha}_{\beta}\wedge\omega^{\beta}=C^{\alpha}_{\beta i}\omega^i\wedge\omega^{\beta}+\frac{1}{2}C^{\alpha}_{\beta\gamma}\omega^{\gamma}\wedge\omega^{\beta}
\end{equation*}

于是$C^{\alpha}_{ij}=0$。
\end{proof}
\begin{rem}
  Frobenius条件也可看作$\local{\dd\omega^{\alpha}}{E}=0$。
\end{rem}

\begin{defn}
  设$X\in\Xf{M}$,定义算子
\begin{align*}
i_X\colon A^k(M)&\longrightarrow A^{k-1}(M)\\
(i_X\omega)(Y_1,\cdots,Y_{k-1})&=\omega(X,Y_1,\cdots,Y_{k-1})
\end{align*}
称为\emph{\red 内置算子}\index{内置算子}。
\end{defn}
\begin{prop}
  设$\omega^{r+1},\cdots,\omega^m$是$r$维切子空间场$E$的定义方程,则Frobenius条件等价于
  \begin{equation*}
    \local{i_X(\dd\omega^{\alpha})}{E}=0,\forall X\in E, r+1\leqslant\alpha\leqslant m
  \end{equation*}
\end{prop}
\begin{proof}
  $\Longrightarrow$:
\begin{align*}
\dd\omega^{\alpha}&=\theta^{\alpha}_{\beta}\wedge\omega^{\beta}=C^{\alpha}_{\beta i}\omega^i\wedge\omega^{\beta}+\frac{1}{2}C^{\alpha}_{\beta\gamma}\omega^{\gamma}\wedge\omega^{\beta}\\
(i_X(\dd\omega^{\alpha}))(Y)&=\dd\omega^{\alpha}(X,Y)\\
                            &=(\theta^{\alpha}_{\beta}\wedge\omega^{\beta})(X,Y)\\
                            &=
                \begin{vmatrix}
                 \theta^{\alpha}_{\beta}(X) & \theta^{\alpha}_{\beta}(Y) \\
                 \omega^{\beta}(X) & \omega^{\beta}(Y) \\
                \end{vmatrix}=0
\end{align*}

$\Longleftarrow$:
$i_X(\dd\omega^{\alpha})(Y)=0$中,取$X=X_i,Y=X_j$并将$\dd\omega^{\alpha}$代入得$C^{\alpha}_{ij}=0$。
\end{proof}

\begin{defn}
  设$f\colon N\to M$是光滑映射,则如下定义的$f^{\ast}\colon A^k(M)\to A^k(N)$称为\emph{\red 拉回}\index{拉回}:
   \begin{equation*}
    (f^{\ast}\omega)(Y_1,\cdots,Y_k)\defeq\omega(f_{\ast}Y_1,\cdots,f_{\ast}Y_k)
   \end{equation*}
\end{defn}

\begin{thm}
  $f^{\ast}\circ\dd=\dd\circ f^{\ast}$
\end{thm}
\begin{proof}
  取$M,N$上的坐标系$(x^i),(y^i)$,设$f=(f^1,\cdots,f^m)$。对于$\omega=\frac{1}{r!}\omega_I\dd x^I$有
\begin{align*}
  &(f^{\ast}\omega)(\pfrac{}{y^{j_1}},\cdots,\pfrac{}{y^{j_r}})\\
 =&\omega(f_{\ast}\pfrac{}{y^{j_1}},\cdots,f_{\ast}\pfrac{}{y^{j_r}})\\
 =&\pfrac{f^{i_1}}{y^{j_1}}\cdots\pfrac{f^{i_r}}{y^{j_r}}\omega(\pfrac{}{x^{i_1}},\cdots,\pfrac{}{x^{i_r}})\\
 =&\frac{1}{r!}\pfrac{f^I}{y^J}\omega_K\dd x^K(\pfrac{}{x^I})\\
 =&\frac{1}{r!}\omega_I(f(y))\dd f^I(\pfrac{}{y^J})
\end{align*}
故
\begin{equation*}
f^{\ast}\omega=\frac{1}{r!}\omega_I(f(y))\dd f^I
\end{equation*}

于是对$\eta\in A(M)$,设$\eta=\eta_I\dd x^I$,则
\begin{align*}
f^{\ast}(\dd\eta)&=f^{\ast}(\dd\eta_I\wedge\dd x^I)\\
                 &=f^{\ast}(\dd\eta_I)\wedge f^{\ast}(\dd x^I)\\
                 &=\dd(\eta_I\circ f)\wedge\dd f^I\\
                 &=\dd((\eta_I\circ f)\dd f^I)\\
                 &=\dd(f^{\ast}\eta)
\end{align*}
\end{proof}
\begin{rem}
$E$是$M$上的$r$维切子空间场,其积分子流形是指浸入$f\colon N\to M$满足:
\begin{align*}
  &f_{\ast}(T_pN)\subset E\\
\Longleftrightarrow&\omega^{\alpha}(f_{\ast}v)=0, \forall v\in T_pN\\
\Longleftrightarrow&f^{\ast}\omega^{\alpha}=0
\end{align*}

Frobenius定理说明:
\begin{equation*}
\{\omega^{r+1},\cdots,\omega^m\}=\{\dd x^{r+1},\cdots,\dd x^m\}
\end{equation*}
于是
\begin{equation*}
f^{\ast}(\dd x^{r+1})=\cdots=f^{\ast}(\dd x^{m})=0
\end{equation*}
即
\begin{equation*}
\dd f^{r+1}=\cdots=\dd f^{m}=0
\end{equation*}
\end{rem}
\begin{exa}
  求$f(x,y,z)$,使得
\begin{equation*}
\begin{cases}
xf_x+f_y+x(1+y)f_z=0\\
f_x+yf_z=0
\end{cases}
\end{equation*}
令
\begin{align*}
X_1&=x\pfrac{}{x}+\pfrac{}{y}+x(1+y)\pfrac{}{z}\\
X_2&=\pfrac{}{x}+y\pfrac{}{z}
\end{align*}
则
\begin{equation*}
[X_1,X_2]=\pfrac{}{z}-\pfrac{}{x}-(1+y)\pfrac{}{z}=-X_2
\end{equation*}
故$E=\spa\{X_1,X_2\}$满足Frobenius条件。

再取$X_3=\pfrac{}{z}$以及$X_1,X_2,X_3$的对偶$\omega^1,\omega^2,\omega^3$,设
\begin{equation*}
\omega^3=P\dd x+Q\dd y+\dd z
\end{equation*}
有
\begin{equation*}
\begin{cases}
xP+Q+x(1+y)=0\\
P+y=0
\end{cases}
\end{equation*}
故
\begin{equation*}
\omega^3=\dd z-y\dd x-x\dd y=\dd(z-xy)
\end{equation*}

可见$E$的积分子流形为
\begin{equation*}
z-xy=cst
\end{equation*}
故$f(x,y,z)=g(z-xy)$,$g$为任意函数、
\end{exa}
\begin{exa}
  考虑$u=u(x,y)$的PDES:
\begin{equation*}
\begin{cases}
u_x=\alpha(x,y,z)\\
u_y=\beta(x,y,z)
\end{cases}
\end{equation*}
在$\Real^3=\{(x,y,u)\}$中来看,相当于由
\begin{equation*}
\omega=\dd u-\alpha\dd x-\beta\dd y
\end{equation*}
定义的$2$维切子空间场的积分问题。

\begin{align*}
\dd\omega&=\alpha_y\dd x\wedge\dd y-\alpha_u\dd u\wedge\dd x-\beta_x\dd x\wedge\dd y+\beta_u\dd y\wedge\dd u\\
          &(\because\dd u\equiv\alpha\dd x+\beta\dd y\mod \omega)\\
          &\equiv(\alpha_y-\beta_x+\alpha_u\beta-\beta_u\alpha)\dd x\wedge\dd y\mod\omega
\end{align*}

注意到$\dd x\wedge\dd y\neq0$,故$\dd\omega\equiv0$等价于
\begin{equation*}
  \alpha_y-\beta_x+\alpha_u\beta-\beta_u\alpha=0
\end{equation*}
即当上式成立时此PDES有解。
\end{exa}

\begin{defn}
  设$\dim M=m$,若$M$上存在处处非零的$\omega=f\dd x^1\wedge\cdots\wedge\dd x^m$,则称$M$\emph{\red 可定向}\index{可定向 }。
\end{defn}

\begin{thm}[单位分解定理]\index{单位分解定理}
设$\Sigma$是光滑流形$M$的一个开覆盖,则$M$上存在一族光滑函数$\{g_{\alpha}\}$满足:
\begin{enumerate}[1)]
    \setlength{\itemindent}{2ex}
    \item 对每个$\alpha$,有$0\leqslant g_{\alpha}\leqslant1$,支集$\supp g_{\alpha}$ 紧,且包含于某个$U_i\in\Sigma$;
    \item 对每点$p$,存在一个邻域$U$,它只与有限个支集$\supp g_{\alpha}$相交;
    \item $\sum_{\alpha}g_{\alpha}=1$。
\end{enumerate}

\end{thm}
\begin{defn}
  如下定义外微分式$\omega$的\emph{\red 积分}\index{外微分式的积分 }
  \begin{equation*}
    \int_{M}\omega=\sum_i\int_{U_i}g_{\alpha}\omega
  \end{equation*}
\end{defn}

\begin{thm}[Stokes公式]\index{Stokes公式}
\begin{equation*}
\int_D\dd\omega=\int_{\partial D}\omega
\end{equation*}
\end{thm}

\begin{defn}
  外微分式$\omega$称为\emph{\red 闭的}\index{闭形式 },如果$\dd\omega=0$,换而言之$\omega\in\ker\dd$;称为\emph{\red 恰当的}\index{恰当形式},如果存在$\theta$使得$\omega=\dd\theta$,换而言之,$\omega\in\im\dd$。
\end{defn}
\begin{prop}
  $\omega$是闭的$1$形式,当且仅当对于任意零伦闭路径$\gamma$,有
\begin{equation*}
\int_{\gamma}\omega=0
\end{equation*}

$\omega$是恰当的$1$形式,当且仅当对于任意闭路径$\gamma$,有
\begin{equation*}
\int_{\gamma}\omega=0
\end{equation*}
\end{prop}

\section{李群}
\subsection*{2012-11-8}
\begin{defn}
  若群$G$赋予了一个光滑结构,使得群运算是光滑映射,则称为\emph{\red 李群}\index{李群}。
\end{defn}
\begin{defn}
  设$G$是李群,$\forall a\in G$,定义
  \begin{center}
  \parbox{0.4\linewidth}{\mapdes{L_a}{G}{G}{b}{ab}}
  \parbox{0.4\linewidth}{\mapdes{R_a}{G}{G}{b}{ba}}
  \end{center}
则它们都是光滑同胚,分别称为\emph{\red 左移动}\index{左移动 }和\emph{\red 右移动}\index{右移动 }。
\end{defn}
\begin{exa}
  $(\Real^n,+)$、$(S^1,\cdot)$是abel李群。
\end{exa}
\begin{exa}
  若$G_1,G_2$是李群,则$G_1\times G_2$也是李群。特别地,$T^n=\underbrace{S^1\times S^1\times\cdots\times S^1}_{n\text{个}}$是abel李群,称为$n$维环面。
\end{exa}
\begin{exa}
  $GL(n,\Real)=\{A\in\Real^{n\times n}\mid\det A\neq0\}$称为$n$阶{\red 一般线性群}\index{一般线性群}。
\end{exa}
\begin{exa}
  $SL(n,\Real)=\{A\in\Real^{n\times n}\mid\det A=1\}$称为$n$阶{\red 特殊线性群}\index{特殊线性群}。
\end{exa}
\begin{exa}
  $O(n,\Real)=\{A\in\Real^{n\times n}\mid AA^T=I_n\}$称为$n$阶{\red 正交群}\index{正交群 }。
\end{exa}
\begin{exa}
  $SO(n,\Real)=SL(n,\Real)\cap O(n,\Real)$称为$n$阶{\red 特殊正交群}\index{特殊正交群 }。
\end{exa}

\begin{defn}
  对李群$G$,若其子群$H$同时也是$G$的闭子流形,则称$H$为$G$的\emph{\red 李子群}\index{李子群 }。
\end{defn}
\begin{exa}
  $G=T^2=S^1\times S^1$,取$H<G$如下
  \begin{equation*}
    H=\{(e^{i\theta},e^{ik\theta})\mid\theta\in\Real\}
  \end{equation*}
其中$k$为给定实数。

当$k\notin\Q$时,$H$不是$G$的闭子流形(事实上$H$稠于$G$),因此不是$G$的李子群。
\end{exa}

\begin{defn}
设$X\in\Xf{G}$,若$\forall a\in G$,
\begin{equation*}
(R_a)_{\ast}X=X
\end{equation*}
则称$X$是$G$上的\emph{\red 右不变向量场}\index{右不变向量场}。
$G$上右不变向量场的全体,记作$\g$。
\end{defn}

\begin{prop}
  $\g$是$\dim G$维的实线性空间,且
\begin{equation*}
X\mapsto X_1
\end{equation*}
给出$\g\to T_1G$的同构。
\end{prop}
\begin{proof}
  $\forall v\in T_1G$,可构造右不变向量场$V$
\begin{equation*}
V_a=(R_a)_{\ast}v
\end{equation*}
反之,右不变向量场$X$满足$X_1=v$,则
\begin{equation*}
X_a=(R_a)_{\ast}X_1=V_a
\end{equation*}
故任一右不变向量场由其在$1$的值唯一确定。
\end{proof}

\begin{thm}
  $\forall X,Y\in\g$,有$[X,Y]\in\g$。
\end{thm}
\begin{proof}
  $\forall a\in G$,$(R_a)_{\ast}[X,Y]=[(R_a)_{\ast}X, (R_a)_{\ast}Y]=[X,Y]$。
\end{proof}

\begin{defn}
  称$(\g,[,])$为李群$G$的\emph{\red 李代数}\index{李代数 }。
\end{defn}
\begin{rem}
  有时也称$T_1G$为$G$的李代数。
\end{rem}
对于$v\in T_1G$,将满足$X_1=v$的右不变向量场记作$\widehat{v}$,并定义
  \begin{equation*}
    [v,w]\defeq[\widehat{v},\widehat{w}]_1
  \end{equation*}

\begin{defn}
取$T_1G$的一组基$v_1,\cdots,v_n$,则可设
\begin{equation*}
[v_i,v_j]=C^k_{ij}v_k
\end{equation*}
其中$C^k_{ij}$称为$G$的\emph{\red 结构常数}\index{结构常数}。
\end{defn}
此时$\widehat{v}_1,\cdots,\widehat{v}_n$是$\g$的一组基,且
\begin{equation*}
[\widehat{v}_i,\widehat{v}_j]=C^k_{ij}\widehat{v}_k
\end{equation*}
\begin{thm}
  结构常数满足方程组:
\begin{equation*}
  \begin{cases}
    C^k_{ij}=-C^k_{ji}\\
    C^i_{sj}C^s_{kl}+C^i_{sk}C^s_{lj}+C^i_{sl}C^s_{jk}=0
  \end{cases}
\end{equation*}
\end{thm}
\begin{rem}
  反之,满足上述方程组的一组常数$C^k_{ij}$就定义了一个李代数$\g$,并有一个局部李群$G$ 以$\g$为李代数。
\end{rem}

\begin{defn}
  对于$x\in G$,定义
  \begin{equation*}
    \alpha_x(a)=L_xR_{x^{-1}}(a)=xax^{-1}
  \end{equation*}
称为$G$的\emph{\red 内自同构}\index{内自同构}。此时,
  \begin{equation*}
    (\alpha_x)_{\ast1}\colon T_1G\longrightarrow T_1G
  \end{equation*}
是李代数的同构,记作$\Ad x$。
\end{defn}
\mapdes{\Ad}{G}{GL(T_1G)\cong GL(n,\Real)}{x}{\Ad x}
\begin{defn}
$\ad=\Ad_{\ast1}\colon T_1G\to gl(n,\Real)$,称为$G$的\emph{\red 伴随表示}\index{伴随表示}。
\end{defn}

\begin{defn}
$\theta\in A^1(G)$,若$\forall a\in G, (R_a)^{\ast}\theta=\theta$,则称$\theta$为$G$上的\emph{\red 右不变$1$形式}\index{右不变$1$形式}。
\end{defn}
\begin{rem}
右不变$1$形式全体与$T^{\ast}_1G$同构,恰是$\g$的对偶空间,记作$\g^{\ast}$。

具体地,设$v_1,\cdots,v_n$是$T_1G$的一组基,则$\widehat{v}_1,\cdots,\widehat{v}_n$是$\g$的一组基,它们构成$G$的整体标架场。记它们的对偶为$\omega^1,\cdots,\omega^n$,则
\begin{equation*}
\dd\omega^i=-\frac{1}{2}C^i_{jk}\omega^j\wedge\omega^k
\end{equation*}
\end{rem}

\begin{defn}
  形式地定义一个$T_1G$值$1$形式
  \begin{equation*}
    \omega=v_i\omega^i
  \end{equation*}
  称为李群$G$的\emph{\red Maurer-Cartan形式}\index{Maurer-Cartan形式}。
\end{defn}

\begin{thm}
  设$\omega^i$是$\g^{\ast}$的一组基,若$\sigma\colon G\to G$满足$\sigma^{\ast}\omega^i=\omega^i$,则$\sigma$是右移动。
\end{thm}
\begin{proof}[Cartan graph technique]\index{Cartan graph technique}
考虑$G\times G$上$n$维切子空间场$E$,定义方程为
\begin{equation*}
\theta^i=\pi_1^{\ast}\omega^i-\pi_2^{\ast}\omega^i
\end{equation*}
其中$\pi_1,\pi_2$是$G\times G$到两个分量的技能。

由于
\begin{align*}
\dd\theta^i&=\pi_1^{\ast}\dd\omega^i-\pi_2^{\ast}\dd\omega^i\\
           &=\pi_1^{\ast}(-\frac{1}{2}C^i_{jk}\omega^j\wedge\omega^k)-\pi_2^{\ast}(-\frac{1}{2}C^i_{jk}\omega^j\wedge\omega^k)\\
           &=-\frac{1}{2}C^i_{jk}(\pi_1^{\ast}\omega^j\wedge\pi_1^{\ast}\omega^k-\pi_2^{\ast}\omega^j\wedge\pi_2^{\ast}\omega^k)\\
           &\equiv0\mod\theta^1,\cdots,\theta^n
\end{align*}

可见$E$满足Frobenius条件,过$G\times G$上任一点总存在唯一的极大积分子流形。

注意到$\sigma$之图像$\{(x,\sigma(x))\mid x\in G\}$恰是$E$的一个积分子流形
\begin{align*}
f\colon N&\to G\times G\to \pi_1(G)\\
f^{\ast}\theta^i&=f^{\ast}(\pi_1^{\ast}\omega^i)-f^{\ast}(\pi_2^{\ast}\omega^i)\\
\theta^i(Y)&=\omega^i(\pi_{1\ast}Y)-\omega^i(\pi_{2\ast}Y)\\
           &=\omega^i(X)-\omega^i(\sigma_{\ast}X)\\
           &=\omega^i(X)-(\sigma^{\ast}\omega^i)(X)=0
\end{align*}

同理,$\forall a\in G$,$R_a$的图像也是$E$的积分子流形。

取$a=\sigma(1)$,则$\sigma$与$R_a$的图像都经过$(1,a)$,故$\sigma=R_a$。
\end{proof}

\begin{exa}
  $GL(n,\Real)$之Maurer-Cartan形式为
  \begin{equation*}
    \omega=(\dd A)A^{-1}
  \end{equation*}
\end{exa}

\section{李群(续)}
\subsection*{2012-11-15}
\begin{defn}
  取$T_1G$的一组基$v_1,\cdots,v_n$,并取$\widehat{v}_1,\cdots,\widehat{v}_n$的对偶$1$形式$\omega^1,\cdots,\omega^n\in A^1(G)$,形式地定义
  \begin{equation*}
    \omega=v_i\omega^i
  \end{equation*}
  称为李群$G$的\emph{\red 右基本微分式}\index{右基本微分式},或\emph{\red Maurer-Cartan形式}。
\end{defn}
\begin{rem}
$\omega(\widehat{v}_i)=v_i$。
\end{rem}
\begin{exa}
$G=GL(n,\Real)$的李代数与右基本微分式:

因为$G$中曲线$I+tA$在$t=0$处的切向量就是$A\in\Real^{n\times n}$,所以
\begin{equation*}
T_IG=\Real^{n\times n}
\end{equation*}
取$A\in T_IG$,注意到这里的$R_ax=xa$是线性变换,故
\begin{align*}
\widetilde{A_x}&=R_xA=Ax\in T_xG\\
\widehat{A}(f)(x)&=\widehat{A}_x(f)\\
                 &=\localt{\dt f(x+tAx)}
\end{align*}
\begin{align*}
[\widehat{A},\widehat{B}](f)(x)&=\widehat{A}_x(\widehat{B}(f))-\widehat{B}_x(\widehat{A}(f))\\
                               &=\localt{\dt\widehat{B}(f)(x+tAx)}-\local{\frac{\dd}{\dd s}\widehat{A}(f)(x+sBx)}{s=0}\\
                               &=\local{\frac{\partial^2}{\partial s\partial t}(f(x+tAx+s(Bx+tBAx))-f(x+sBx+t(Ax+sABx)))}{t=s=0}\\
                               &=\local{\frac{\partial^2}{\partial s\partial t}(f(x+stBAx)-f(x+stABx))}{t=s=0}\\
                               &=\localt{\dt(f(x+t(BA-AB)x))}\\
                               &=(BA-AB)^{\wedge}(f)(x)
\end{align*}
所以$[\widehat{A},\widehat{B}]=(BA-AB)^{\wedge}$。

换言之,$T_IG$中的$[,]$运算由
\begin{equation*}
[A,B]=BA-AB
\end{equation*}
给出。

右基本微分式为
\begin{equation*}
\omega=\dd x\cdot x^{-1}
\end{equation*}
其中$x=(x_{ij}), \dd x=(\dd x_{ij})$。

这是因为
\begin{align*}
 \omega(\widehat{A})&=\dd x((Ax)_{ij}\pfrac{}{x_{ij}})x^{-1}\\
                   &=(Ax)x^{-1}=A
\end{align*}
\end{exa}

\begin{defn}
  对李群$G$,称李群同态$\gamma\colon(\Real.+)\to G$为$G$的一个\emph{\red 单参数子群}\index{单参数子群 }。
\end{defn}

\begin{thm}
  单参数子群$\gamma$由$\dot{\gamma}(0)\in T_1G$唯一决定。
\end{thm}
\begin{proof}
  设$v=\dot{\gamma}(0)\in T_1G$,考虑$\widehat{v}$的流$\theta_t$,则
\begin{equation*}
\begin{cases}
\theta_0=\id\\
\theta_{t+s}=\theta_t\circ\theta_s
\end{cases}
\end{equation*}
考虑$\widetilde{\gamma}(t)=\theta_t(1)$,则$\widetilde{\gamma}$是单参数子群。
\begin{align*}
\dot{\gamma}(t)&=\local{\frac{\dd}{\dd s}\gamma(t+s)}{s=0}\\
               &=\local{\frac{\dd}{\dd s}(R_{\gamma(t)}\gamma(s))}{s=0}\\
               &=(R_{\gamma(t)})_{\ast}\dot{\gamma}(0)
\end{align*}

可见$\gamma(t)$是右不变向量$\widehat{v}$的积分曲线。
\end{proof}

\begin{exa}
$G=S^1=\{e^{it}\mid t\in\Real\}$,固定$k$,$\gamma(t)=e^{itk}$就是$S^1$的一个单参数子群,$\dot{\gamma}(0)=ik$。
\end{exa}

\begin{defn}
  对于$v\in T_1G$,设$v$对应的单参数子群为$\gamma(t)$,定义
  \mapdes{\exp}{T_1G}{G}{v}{\gamma(1)}
称为\emph{\red 指数映射}\index{指数映射}。
\end{defn}
\begin{rem}
  $\exp(kv)=\gamma(k),\forall k\in\Real$。
\end{rem}

\begin{exa}
$G=GL(n,\Real)$
\begin{equation*}
\exp(A)=I+A+\frac{A^2}{2!}+\frac{A^3}{3!}+\cdots
\end{equation*}
\end{exa}
\begin{exa}
$G=S^1$
\begin{equation*}
\exp(t)=e^{it}
\end{equation*}
\end{exa}

\begin{defn}
  对于李群$G$和流形$M$,若光滑映射$\sigma\colon G\times M\to M$满足:
\begin{enumerate}[1)]
    \setlength{\itemindent}{2ex}
    \item $\sigma(1,x)=x$;
    \item $\sigma(a,\sigma(b,c))=\sigma(ab,c)$
\end{enumerate}
则称之为$G$在$M$上的一个\emph{\red (左)作用}\index{作用}。简记$\sigma(a,x)$为$ax$。
\end{defn}

\begin{exa}
  $S^1$在$S^3$上的左作用
\begin{equation*}
\sigma(z,(\xi,\eta))=(z\xi,z\eta)
\end{equation*}
\end{exa}

\begin{exa}
  $SL(2,\Real)$在$S^2\cong\Complex\cup\{\infty\}$上的左作用:
\begin{equation*}
\sigma(\begin{pmatrix}
         a & b \\
         c & d \\
       \end{pmatrix}
,z)=\frac{az+b}{cz+d}
\end{equation*}
\end{exa}

\begin{defn}
  设李群G左作用在流形M上,
\begin{enumerate}[1)]
    \setlength{\itemindent}{2ex}
    \item 若$\forall a\in G, a\neq1$,存在$x\in M$使得$ax\neq x$,则称为\emph{\red 有效的}\index{有效作用};
    \item 若$\forall a\in G, a\neq1, \forall x\in M$,都有$ax\neq x$,则称为\emph{\red 自由的}\index{自由作用}。
\end{enumerate}
\end{defn}

\begin{defn}
  设李群G左作用在流形M上,对$v\in T_1G$,单参数子群$\exp(tv)$在$M$上之作用形成一个流
  \begin{equation*}
    \theta_t(x)=\exp(tv).x
  \end{equation*}
将$\theta_t$在$M$上生成的向量场记作$\widetilde{v}$,称为(相应于$v$的)\emph{\red 基本向量场}\index{基本向量场 }。
\end{defn}

\begin{thm}
  设李群$G$左作用在流形$M$上,则$M$上的全体基本向量场构成一个李代数,且这个李代数是$T_1G$的同态像。进一步,若该作用有效,则这个李代数同构于$T_1G$。
\end{thm}
\begin{proof}
  在$\sigma\colon G\times M\to M$中固定$p\in M$,则得到
\mapdes{\sigma_p}{G}{M}{a}{a.p}
此时$(\sigma_p)_{\ast}\colon T_1G\to T_pM$满足$(\sigma_p)_{\ast1}v=\widetilde{v}_p$。(因为$\sigma_p(\exp(tv))=\exp(tv)p=\theta_t(p)$,而流在$t=0$处求导,即得$(\sigma_p)_{\ast1}v=\widetilde{v}_p$。)

考虑$(\sigma_p)_{\ast a}\colon T_aG\to T_{a.p}M$,$\widehat{v}_a$是$\exp(tv)a$的切向量。

所以$(\sigma_p)_{\ast a}(\widehat{v}_a)$是$\sigma_p(\exp(tv)a)$之切向量。其中$\exp(tv)a.p=\exp(tv).ap=\theta_t(ap)=\widetilde{v}_{ap}$。

可见$(\sigma_p)_{\ast}\widehat{v}=\widetilde{v}$,故$[\widetilde{v},\widetilde{w}]=[\sigma_{p\ast}\widehat{v},\sigma_{p\ast}\widehat{w}]=\sigma_{p\ast}[\widehat{v},\widehat{w}]$。
\end{proof}

\begin{exa}
  $SL(2,\Real)$在$S^2$上的作用:

对于$v=\begin{pmatrix}
  a & b \\
  c & -a \\
\end{pmatrix}\in \mathfrak{sl}(2,\Real)$
有$
\exp(tv)=\begin{pmatrix}
  a(t) & b(t) \\
  c(t) & d(t) \\
\end{pmatrix}$
,其中
\begin{equation*}
\begin{cases}
a(0)=d(0)=1\\
b(0)=c(0)=0\\
\end{cases}
\begin{cases}
\dot{a}(0)=a&
\dot{b}(0)=b\\
\dot{c}(0)=c&
\dot{d}(0)=-a
\end{cases}
\end{equation*}

\begin{align*}
\local{\widetilde{v}}{z}&=\localt{\dt\exp(tv).z}\\
                        &=\localt{\dt\frac{a(t)z+b(t)}{c(t)z+d(t)}}\\
                        &=\dot{a}(0)z+\dot{b}(0)-(a(0)z+b(0))(\dot{c}(0)z+\dot{d}(0))\\
                        &=az+b-z(cz-a)\\
                        &=-cz^2+2az+b
\end{align*}
\end{exa}

\section{李氏变换群的应用}
\subsection*{2012-11-22}
\subsection{李型微分方程}
\begin{defn}
  给定李群$G$在$M$上的一个左作用,设$A\colon\Real\to T_1G$是光滑曲线,称$M$上的常微分方程
  \begin{equation*}
  \gamma'{t}=\widetilde{A(t)}(\gamma(t))
  \end{equation*}
  为\emph{\red 李型微分方程}\index{李型微分方程}。
\end{defn}

\begin{exa}[Ricatti方程]\index{Ricatti方程}
\begin{equation*}
x'(t)=a_0(t)+2a_1(t)x(t)+a_2(t)x(t)^2
\end{equation*}
其中$a_0,a_1,a_2$已知。
\end{exa}
\begin{proof}
考虑$G=SL(2,\Real)$在$\RP^1$上的作用。此时,相应于
$v=\begin{pmatrix}
  a_1 & a_0 \\
  -a_2 & -a_1 \\
\end{pmatrix}\in\mathfrak{sl}(2,\Real)$的基本向量场为
\begin{equation*}
\widetilde{v}(x)=a_0+2a_1+a_2x^2
\end{equation*}
取$T_1G$中的曲线
$A(t)=\begin{pmatrix}
  a_1(t) & a_0(t) \\
  -a_2(t) & -a_1(t) \\
\end{pmatrix}$则
\begin{align*}
\gamma'(t)&=\widetilde{A(t)}(\gamma(t))\\
          &=a_0(t)+2a_1(t)\gamma(t)+a_2(t)\gamma(t)^2
\end{align*}
是李型微分方程。
\end{proof}

\begin{exa}[一阶线性微分方程]\index{一阶线性微分方程 }
\begin{equation*}
x'(t)=a(t)x(t)+b(t)
\end{equation*}
这里$x(t),b(t)\in\Real^{n\times1}, a(t)\in\Real^{n\times n}$,而$a(t),b(t)$已知。
\end{exa}
\begin{proof}
考虑
\begin{equation*}
G=\left\{\begin{pmatrix}
            a & b \\
            0 & 1 \\
          \end{pmatrix}\in\Real^{(n+1)\times(n+1)}
\middle| a\in GL(n,\Real), b\in\Real^{n\times1}\right\}
\end{equation*}
$G$在$\Real^n$上的左作用(\emph{仿射变换}\index{仿射变换})如下:
\begin{equation*}
\begin{pmatrix}
  a & b \\
  0 & 1 \\
\end{pmatrix}.x=ax+b
\end{equation*}
经计算得:
\begin{equation*}
T_1G=\left\{\begin{pmatrix}
            a & b \\
            0 & 0 \\
          \end{pmatrix}\in\Real^{(n+1)\times(n+1)}
\middle| a\in\Real^{n\times n}, b\in\Real^{n\times1}\right\}
\end{equation*}
对应于$A=\begin{pmatrix}
            a & b \\
            0 & 0 \\
          \end{pmatrix}$的基本向量场是
\begin{align*}
  \widetilde{A}(x)&=\localt{\dt\exp(tA).x}\\
                         &=\localt{\dt e^{tA}.x}\\
                         &=ax+b
\end{align*}
因此,取$T_1G$中曲线$A(t)=\begin{pmatrix}
            a(t) & b(t) \\
            0 & 0 \\
          \end{pmatrix}$,则微分方程
\begin{align*}
\gamma'(t)&=\widetilde{A(t)}(\gamma(t))\\
          &=a(t)\gamma(t)+b(t)
\end{align*}
是李型微分方程。
\end{proof}

\begin{thm}
  给定李型微分方程
\begin{equation*}
\gamma'{t}=\widetilde{A(t)}(\gamma(t))
\end{equation*}
设$S\colon\Real\to G$是微分方程
\begin{equation*}
S'(t)=(R_{s(t)})_{\ast}A(t)
\end{equation*}
的满足$S(0)=1$的唯一解,则
\begin{equation*}
\gamma'{t}=\widetilde{A(t)}(\gamma(t))
\end{equation*}
的满足$\gamma(0)=m\in M$的解是$\gamma(t)=S(t).m$,称$S(t)$为李型微分方程的{\red 基解}\index{基解}。
\end{thm}
\begin{proof}
\begin{align*}
\sigma\colon G\times M&\to M\\
\sigma_p(g)&=g.p\\
(\sigma_p)_{\ast}\widehat{v}_a&=\widetilde{v}_{a.p}
\end{align*}
由
\begin{equation*}
\gamma(t)=\sigma_m(S(t))
\end{equation*}
得
\begin{align*}
\gamma'(t)&=(\sigma_m)_{\ast}S'(t)\\
          &=(\sigma_m)_{\ast}(R_{s(t)})_{\ast}A(t)\\
          &=\widetilde{A(t)}(S(t).m)\\
          &=\widetilde{A(t)}(\gamma(t))
\end{align*}
\end{proof}

\begin{exa}
  $x'(t)=a(t)x(t)+b(t)$的基解是$S(t)=\begin{pmatrix}
  x(t) & y(t) \\
  0 & 1 \\
\end{pmatrix}$
\end{exa}
\begin{proof}
\begin{align*}
  S'(t)&=(R_{s(t)})_{\ast}A(t)\\
\begin{pmatrix}
  x'(t) & y(t) \\
  0 & 0 \\
\end{pmatrix}
&=
\begin{pmatrix}
  a(t) & b(t) \\
  0 & 0 \\
\end{pmatrix}
\begin{pmatrix}
  x(t) & y(t) \\
  0 & 1 \\
\end{pmatrix}\\
\Longrightarrow&\begin{cases}
x'(t)=a(t)x(t)\\
x(0)=I
\end{cases}\\
\Longrightarrow&\begin{cases}
y'(t)=a(t)y(t)+b(t)\\
y(0)
\end{cases}
\end{align*}
\end{proof}

\subsubsection*{Lie的约化方法}(由Ricatti方程的研究得到)

假设已有李型微分方程$\gamma'{t}=\widetilde{A(t)}(\gamma(t))$的一个特解$\gamma_0(t)$,且$\gamma_0(0)=m$。
可取$g(t)\in G$,使$\gamma_0(t)=g(t).m$,但$g(t)$一般不是基解。

设基解为$S(t)=g(t)h(t)$,其中$h(t)\in G_m\defeq\{g\in G\mid g.m=m\}$,则$h(t)$需满足
\begin{align*}
(g(t)h(t))'&=(R_{g(t)h(t)})_{\ast}A(t)\\
(L_{g(t)})_{\ast}h'(t)+(R_{h(t)})_{\ast}g'(t)&=(R_{h(t)})_{\ast}(R_{g(t)})_{\ast}A(t)\\
\Longrightarrow h'(t)&=(L_{g(t)})_{\ast}^{-1}(R_{h(t)})_{\ast}((R_{g(t)})_{\ast}A(t)-g'(t))\\
                     &=(R_{h(t)})_{\ast}(L_{g(t)})_{\ast}^{-1}((R_{g(t)})_{\ast}A(t)-g'(t))\\
\end{align*}
令$B(t)=L_{g(t)\ast}^{-1}((R_{g(t)})_{\ast}A(t)-g'(t))$,则$B(t)\in T_1G_m=\ker\sigma_{m\ast}$。

于是$h(t)$需满足
\begin{equation*}
h'(t)=R_{h(t)\ast}B(t)
\end{equation*}
这是$G_m$上的一个基解方程。

\begin{exa}[Ricatti方程]
设$x'=a_0+2a_1x+a_2x^2$有特解$x_0(t)$,取
\begin{equation*}
g(t)=\begin{pmatrix}
  1 & x_0(t) \\
  0 & 1 \\
\end{pmatrix}\in SL(2,\Real)
\end{equation*}
则$x_0(t)=g(t).0$。

\begin{align*}
  G_0&=\left\{\begin{pmatrix}
                a & b \\
                c & d \\
              \end{pmatrix}\in SL(2,\Real)
              \middle|b=0,d\neq0\right\}
              \\
       &=\left\{\begin{pmatrix}
                u & 0 \\
                v & u^{-1} \\
              \end{pmatrix}
              \middle|u\in\Real^{\times}, v\in\Real\right\}
\end{align*}

设基解为$S(t)=g(t)h(t)=\begin{pmatrix}
                     1 & x_0(t) \\
                     0 & 1 \\
                   \end{pmatrix}\begin{pmatrix}
                    u & 0 \\
                    v & u^{-1} \\
                  \end{pmatrix}$,则
\begin{align*}
  B(t)&=L^{-1}_{g(t)\ast}(R_{g(t)\ast}A(t)-g'(t))\\
       &=\begin{pmatrix}
                     1 & -x_0 \\
                     0 & 1 \\
                   \end{pmatrix}
            \left(\begin{pmatrix}
               a_1 & a_0 \\
               -a_2 & -a_1 \\
             \end{pmatrix}
                 \begin{pmatrix}
                     1 & x_0 \\
                     0 & 1 \\
                   \end{pmatrix}
             -\begin{pmatrix}
                     0 & a_0+2a_1x_0+a_2x_0^2 \\
                     0 & 0 \\
                   \end{pmatrix}
            \right)\\
       &=\begin{pmatrix}
               a_1+a_0x_0 & 0 \\
               -a_2 & -(a_1+a_2x_0) \\
             \end{pmatrix}
\end{align*}

由$h'(t)=R_{h(t)\ast}B(t)$得
\begin{equation*}
\begin{cases}
u'(t)=u(t)(a_1+a_2x_0)\\
v'(t)=-a_2u-(a_1+a_0x_0)v
\end{cases}
\end{equation*}
可见$u,v$可积。
\end{exa}

\paragraph{问题}求基解之难度?
\begin{exa}
$G$是abel群,则
\begin{align*}
S'(t)&=R_{s(t)\ast}A(t)\\
     &=A(t)
\end{align*}
\end{exa}

\begin{exa}
由交换的矩阵构成的李群$G$
\begin{align*}
&S'(t)=A(t)S(t)\\
\Longrightarrow&S(t)=\exp(\int A(t)\dd t)
\end{align*}
\end{exa}

\paragraph{参考文献}
Peter.J.Olver ,Application of Lie groups to Differential Equations


\section{习题一}
\begin{xiti}
  $f\colon M\to N,g\colon N\to K$都是光滑流形之间的光滑映射,证明:
  \begin{equation*}
    g_{\ast}\circ f_{\ast}=(g\circ f)_{\ast}
  \end{equation*}
\end{xiti}
\begin{proof}
  \emph{主要方法是取曲线并注意到切映射与曲线选取无关。}

对于$p\in M$,设$f(p)=q,g(q)=r$,$\forall v\in T_pM$,设$\gamma(t)\subset M$在$t=0$处的切向量是$v$,则有$\widetilde{\gamma}(t)=f(\gamma(t))\subset N$在$t=0$处的切向量是$f_{\ast}v$。

于是$g(f(\gamma(t)))\subset K$在$t=0$处的切向量,一方面是$(g\circ f)(\gamma(t))$的切向量,即$(g\circ f)_{\ast}v$;另一方面也是$g(\widetilde{\gamma})$的切向量,即$g_{\ast}(f_{\ast}v)$。

于是$g_{\ast}\circ f_{\ast}=(g\circ f)_{\ast}$。
\end{proof}

\begin{xiti}
  $f\colon\Real^{n\times n}\to\Real$是行列式函数$f(p)=\det(p)$,$p\in GL(n,\Real)$,$e$是单位矩阵,求
\begin{enumerate}[1)]
    \setlength{\itemindent}{2ex}
    \item $\rank_pf$;
    \item $f_{\ast e}$。
\end{enumerate}
\end{xiti}
\begin{proof}
  \emph{主要是注意到$f_{\ast p}(v)$是$f(p+tv)$在$f(p)$点的切向量。}

  由于$\dim\Real=1$,所以$\rank_pf\leqslant1$,为使$\rank_pf=1$,只须在每点$p\in GL(n,\Real)$找到切向量$v$,使得$f_{\ast p}(v)\neq0$。

  为此,取$v\in GL(n,\Real)$,则
  \begin{equation*}
  f(p+tv)=|v||pv^{-1}+te|=|v|P_{-pv^{-1}}(t)
  \end{equation*}
  其中$P_{-pv^{-1}}(t)$是关于$-pv^{-1}$的特征多项式。

  特别地,取$v=p\in\Real^{n\times n}\cong T_p\Real^{n\times n}$,有
  \begin{align*}
  f_{\ast p}(p)&=\localt{\dt f(p+tp)}\\
             &=|p|\localt{(1+t)^n}\neq0
  \end{align*}

这就证明了$\rank_pf=1$。

由于$f_{\ast e}(v)$是曲线$f(e+tv)$在$t=0$处的切向量。
\begin{align*}
f_{\ast e}(v)&=\localt{\dt|e+tv|}\\
\dt|e+tv|&=(-t)^n|v-\frac{1}{t}e|
\end{align*}
若$v$之特征多项式为$F(\lambda)$,则
\begin{equation*}
|e+tv|=(-t)^nF(\frac{1}{t})
\end{equation*}
由于
\begin{equation*}
F(\lambda)=\lambda^n-(\lambda_1+\cdots+\lambda_n)\lambda^{n-1}+\cdots
\end{equation*}
所以
\begin{align*}
|e+tv|&=(-t)^{n}((-\frac{1}{t})^n-\tr(v)(-\frac{1}{t})^{n-1}+\cdots)\\
      &=1+\tr(v) t+o(t)
\end{align*}
所以$f_{\ast e}(v)=\tr(v)$。
\end{proof}

\begin{xiti}
  证明$SU(2)$与$S^3$微分同胚。其中
  \begin{align*}
  SU(2)&=\left\{
  \begin{pmatrix}
  a & b \\
  c & d \\
  \end{pmatrix}
  \in\Complex^{2\times2}
  \middle|
  |a|^2+|b|^2=1, |c|^2+|d|^2=1, ad-bc=1, a\bar{c}+b\bar{d}=0
  \right\}\\
  S^3&=\left\{(z_1,z_2)\in\Complex^2\mid |z_1|^2+|z_2|^2=1\right\}
  \end{align*}
\end{xiti}
\begin{proof}
  \emph{先通过解方程化简$SU(2)$的表达式,最后看出这个同胚来。}

  \begin{equation*}
  a\bar{c}+b\bar{d}=0\Longrightarrow ad|c|^2+bc|d|^2=0
  \end{equation*}

  将之代入$ad-bc=1$得
  \begin{equation*}
  ad(1+\frac{|c|^2}{|d|^2})=1
  \end{equation*}
  又$|c|^2+|d|^2=1$,故得
  \begin{equation*}
  a=\frac{\bar{d}}{|c|^2+|d|^2}=\bar{d}
  \end{equation*}

  同理可得$b=-\bar{c}$。于是
  \begin{equation*}
  SU(2)=\left\{
  \begin{pmatrix}
  a & b \\
  -\bar{b} & \bar{a} \\
  \end{pmatrix}
  \in\Complex^{2\times2}
  \middle|
  |a|^2+|b|^2=1
  \right\}
  \end{equation*}

  对应
  \begin{equation*}
  \begin{pmatrix}
  a & b \\
  -\bar{b} & \bar{a} \\
  \end{pmatrix}
  \in SU(2)
  \longleftrightarrow
  (a,b)\in S^3
  \end{equation*}
  显然是光滑的双射,于是$SU(2)\cong S^3$。
\end{proof}

\begin{xiti}
  证明$SL(n,\Real), O(n,\Real)$是$GL(n,\Real)$的闭子流形。
\end{xiti}
\begin{proof}
  \emph{运用正则值原像定理(\ref{正则值原像定理})或者淹没原像定理(\ref{淹没原像})。}

  考虑
\mapdes{f}{GL(n,\Real)}{\Real}{p}{\det(p)}

则由于$\forall p\in GL(n,\Real),\rank_pf=1$,故$\forall c\neq0$,$f^{-1}(c)$是$GL(n,\Real)$的闭子流形。特别地,取$c=1$,则$SL(n,\Real)$是$GL(n,\Real)$的闭子流形。

令$S$为实对称矩阵全体,考虑
\mapdes{f}{GL(n,\Real)}{S}{A}{A^{T}A}

下证$\forall p\in GL(n,\Real),\rank_pf=\dim S$。注意到对$f(p)=q$,有$T_qS\cong S$,于是$\forall v\in S$有
\begin{align*}
f_{\ast p}(v)&=\localt{\dt (p+tv)^{T}(p+tv)}\\
             &=p^Tv+v^Tp
\end{align*}

故$\forall B\in S$,取$v=\frac{1}{2}(p^{-1})^TB$,则$f_{\ast p}(v)=B$。即$f_{\ast p}$是满射,故满秩。

于是$O(n,\Real)=f^{-1}(I)$是$GL(n,\Real)$的闭子流形。
\end{proof}

\begin{xiti}
  证明下述$M$是$\Real^{2n+2}$的闭子流形($\Complex^{n+1}$与$\Real^{2n+2}$视为等同)。
  \begin{equation*}
  M=\{(z_0,z_1,\cdots,z_n)\in\Complex^{n+1}\mid z_0^2+z_1^2+\cdots+z_n^2=1\}
  \end{equation*}
\end{xiti}
\begin{proof}
  \emph{注意到$f_{\ast p}\colon T_pM\to T_pN$在基$\{\px{i}\}$和$\{\py{\alpha}\}$下的矩阵为Jacobi矩阵。}

先将$\Complex^{n+1}$下的坐标改写为$\Real^{2n+2}$下的坐标:$z_j=x_j+iy_j$。则$M$的等式成为:
\begin{equation*}
\begin{cases}
\sum_j(x_j^2-y_j^2)=1\\
\sum_jx_jy_j=0
\end{cases}
\end{equation*}

考虑
\mapdes{f}{\Real^{2n+2}}{\Real^2}{(x_0,\cdots,x_n,y_0,\cdots,y_n)}{(\sum(x_j^2-y_j^2),\sum x_jy_j)}

则
\begin{equation*}
\rank_pf=\rank
    \begin{pmatrix}
      2x_0 & \cdots & 2x_n & -2y_0 & \cdots & -2y_n \\
      y_0 & \cdots & y_n & x_0 & \cdots & x_n \\
    \end{pmatrix}
=2
\end{equation*}

故$M=f^{-1}(1,0)$是$\Real^{2n+2}$的闭子流形。
\end{proof}

\begin{xiti}
  证明上题中的$M$与$TS^n$(即$S^n$的切丛)微分同胚。
\end{xiti}
\begin{proof}
  注意到:
  \begin{equation*}
  TS^n=\{(x,y)\in\Real^{n+1}\times\Real^{n+1}\mid |x|=1, x\cdot y=0\}
  \end{equation*}

  考虑
  \begin{center}
  \parbox{0.4\linewidth}{\mapdes{f}{M}{TS^n}{(x,y)}{(\frac{x}{\sqrt{1+|y|^2}},y)}}
  \parbox{0.4\linewidth}{\mapdes{f^{-1}}{TS^n}{M}{(x,y)}{(x(1+|y|^2)^{\frac{1}{2}},y)}}
  \end{center}

  则$f$为微分同胚。
\end{proof}

\begin{xiti}
  设$f\colon M\to N$是光滑映射,$X,Y\in\Xf{M}$,如果$f_{\ast}X$和$f_{\ast}Y$是$N$上的向量场,证明:$f_{\ast}[X,Y]$也是$N$上的向量场,且$f_{\ast}[X,Y]=[f_{\ast}X,f_{\ast}Y]$。
\end{xiti}
\begin{proof}
  \emph{善用下面等式:}
   \begin{equation*}
     (f_{\ast}X)(g)(f(p))=f_{\ast p}(X_p)(g)=\localt{\dt g(f(\gamma(t)))}=X_p(g\circ f)
   \end{equation*}

  由Lie括号的定义(\ref{Lie括号}),由于$f_{\ast}X$和$f_{\ast}Y$是$N$上的向量场,故$[f_{\ast}X,f_{\ast}Y]$也是$N$上的向量场。于是只须证$f_{\ast}[X,Y]=[f_{\ast}X,f_{\ast}Y]$:
  \begin{align*}
    ([f_{\ast}X,f_{\ast}Y](g))\circ f&=(f_{\ast}X(f_{\ast}Y(g)))\circ f-(f_{\ast}Y(f_{\ast}X(g)))\circ f\\
                                 &=X(f_{\ast}Y(g)\circ f)-Y(f_{\ast}X(g)\circ f)\\
                                 &=X(Y(g\circ f))-(Y(X(g\circ f)))\\
                                 &=[X,Y](g\circ f)\\
                                 &=(f_{\ast}[X,Y](g))\circ f
  \end{align*}
\end{proof}

\begin{xiti}
  在$\Real^3$中,定义向量场$X=y\pfrac{}{z}-z\pfrac{}{y},Y=z\pfrac{}{x}-x\pfrac{}{z}$。请计算$Z=[X,Y]$,并证明$X,Y,Z$都与$\Real^3$的子流形$S^2$相切。
\end{xiti}
\begin{proof}
  \emph{通过证明$X,Y,Z$是$SO(3)$的基本向量场,而$S^2$又是$SO(3)$在$\Real^3$上左作用的一个轨道,来证明$X,Y,Z$与$S^2$相切。}

  首先计算$Z=[X,Y]$:
  \begin{equation*}
  Z=[X,Y]=XY-YX=y\pfrac{}{x}-x\pfrac{}{y}
  \end{equation*}

  考虑$SO(3)$的一组基
  \begin{equation*}
    e_1=
         \begin{pmatrix}
           0 &  &  \\
            & 0 & 1 \\
            & -1 & 0 \\
         \end{pmatrix}
    ,  e_2=
         \begin{pmatrix}
           0 &  & 1 \\
            & 0 &  \\
           -1 &  & 0 \\
         \end{pmatrix}
    ,  e_3=
         \begin{pmatrix}
           0 & 1 &  \\
           -1 & 0 &  \\
            &  & 0 \\
         \end{pmatrix}
  \end{equation*}

  $\forall p=(x,y,z)^T$,有
  \begin{align*}
  \widetilde{e_1}(p)&=\localt{\dt(\exp(te_1).p)}\\
                  &=\localt{\dt
         \begin{pmatrix}
           1 &  &  \\
            & \cos t & \sin t \\
            & -\sin t & \cos t \\
         \end{pmatrix}
         \begin{pmatrix}
           x \\
           y \\
           z \\
         \end{pmatrix}}\\
                  &=
         \begin{pmatrix}
           0 \\
           z \\
           -y \\
         \end{pmatrix}=z\pfrac{}{y}-y\pfrac{}{z}
  \end{align*}

  所以$\widetilde{e_1}=-X$同理可得$\widetilde{e_2}=-Y,\widetilde{e_3}=-Z$。

  取$SO(3)$在$\Real^3$上的作用为:
  \begin{equation*}
  \sigma(A,X)=AX, \forall A\in SO(3),X\in\Real^3
  \end{equation*}

  则$S^2$是该作用的一个轨道,故$\forall p\in S^2,X(f(p))\in f_{\ast}(T_pS^2)$,其中$f$是嵌入,即$X,Y,Z$与$S^2$相切。
\end{proof}
\begin{proof}
  \emph{直接验证$X$与$S^2$相切,余类似。}

  设$X$生成的流为$\phi_t$,且
  \begin{equation*}
  \phi_t(x,y,z)=(a,b,c)
  \end{equation*}

由于
\begin{equation*}
X(x,y,z)=(0,-z,y)
\end{equation*}

故
\begin{align*}
\local{(\dot{a},\dot{b},\dot{c})}{t=s}&=\local{\dt\phi_t(x,y,z)}{t=s}\\
                                      &=\localt{\dt\phi_{t+s}(x,y,z)}\\
                                      &=\localt{\dt\phi_t(a(s),b(s),c(s))}\\
                                      &=(0,-c(s),b(s))
\end{align*}

注意到$\phi_0=\id$,从而
\begin{equation*}
\phi_t(x,y,z)=(x,y\cos t+z\sin t,y\sin t-z\cos t)
\end{equation*}

故$\forall p\in S^2$,$\phi_t(p)$是$S^2$上的以$X_p$为切向量的曲线。
\end{proof}
\begin{proof}
  用$\Real^3$的整体坐标,$S^2=\{(x,y,z)\mid x^2+y^2+z^2=1\}$的法向量\footnote{比如可以用梯度来求得。}表示为
  \begin{equation*}
    (x,y,z)
  \end{equation*}

  而$X,Y,Z$则表示为
  \begin{equation*}
    (0,-z,y), (z,0,-x),(y,-x,0)
  \end{equation*}

  易知$X,Y,Z$与$S^2$相切。
\end{proof}

\begin{xiti}
  在$\Real^3$上定义向量场
  \begin{equation*}
  X=-y\pfrac{}{x}+x\pfrac{}{y}+(1+z^2)\pfrac{}{z}
  \end{equation*}
  试计算$X$生成的流。
\end{xiti}
\begin{proof}
  设
\begin{equation*}
\phi_t(x,y,z)=(a,b,c)
\end{equation*}

由于
\begin{equation*}
X(x,y,z)=(-y,x,1+z^2)
\end{equation*}

故
\begin{align*}
\local{(\dot{a},\dot{b},\dot{c})}{t=s}&=\local{\dt\phi_t(x,y,z)}{t=s}\\
                                      &=\localt{\dt\phi_{t+s}(x,y,z)}\\
                                      &=\localt{\dt\phi_t(a(s),b(s),c(s))}\\
                                      &=(-b,a,1+c^2)
\end{align*}

注意到$\phi_0=\id$,从而
\begin{equation*}
\phi_t(x,y,z)=(x\cos t+y\sin t,x\sin t-y\cos t,\tan(t+\arctan z))
\end{equation*}
\end{proof}

\begin{xiti}
  设向量场$X\in\Xf{M}$生成的流是$\phi_t$,$\psi\colon M\to M$是微分同胚,求证:$\psi_{\ast}X$生成的流是$\psi\circ\phi_t\circ\psi^{-1}$。
\end{xiti}
\begin{proof}
\begin{enumerate}[1)]
    \setlength{\itemindent}{2ex}
    \item $\psi\circ\phi_t\circ\psi^{-1}$是流。
    \item $\psi\circ\phi_t\circ\psi^{-1}$诱导的向量场是$\psi_{\ast}X$:
  \begin{align*}
     &\localt{\dt\psi\circ\phi_t\circ\psi^{-1}(p)}\\
  =&\psi_{\ast}(\localt{\dt\phi_t(\psi^{-1}(p))})\\
  =&\psi_{\ast}X_{\psi^{-1}(p)}
  \end{align*}
\end{enumerate}
\end{proof}

\begin{xiti}
  定义变换$\phi_t\colon\Real^2\to\Real^2$如下:
  \begin{equation*}
    \phi_t(x,y)=(x\cosh t+y\sinh t,x\sinh t+y\cosh t)
  \end{equation*}
  证明:$\phi_t$是$\Real^2$上的流,并求其诱导的向量场$X$。
\end{xiti}
\begin{proof}
  首先证明$\phi_t$是$\Real^2$上的流。注意到
  \begin{equation*}
    \phi_t(x,y)=\begin{pmatrix}
                  \cosh t & \sinh t \\
                  \sinh t & \cosh t \\
                \end{pmatrix}\begin{pmatrix}
                               x \\
                               y \\
                             \end{pmatrix}
  \end{equation*}

  故只须证
  \begin{equation*}
      \begin{pmatrix}
                  \cosh t & \sinh t \\
                  \sinh t & \cosh t \\
      \end{pmatrix}
      \begin{pmatrix}
                  \cosh s & \sinh s \\
                  \sinh s & \cosh s \\
      \end{pmatrix}
      =
      \begin{pmatrix}
                  \cosh (t+s) & \sinh (t+s) \\
                  \sinh (t+s) & \cosh (t+s) \\
      \end{pmatrix}
  \end{equation*}

  \begin{align*}
    X(x,y) &=\localt{\dt\phi_t(x,y)} \\
     &=\localt{(x\sinh t+y\cosh t, x\cosh t+y\sinh t)}\\
     &=(y,x)
  \end{align*}

  即$X=y\pfrac{}{x}+x\pfrac{}{y}$。
\end{proof}

\begin{xiti}
  在$M:=\Real^3\backslash\{0\}$上定义如下$3$个向量场:
  \begin{equation*}
    X=2y\pfrac{}{x}+z\pfrac{}{y}, Y=-2x\pfrac{}{x}+2z\pfrac{}{z}, Z=x\pfrac{}{y}+2y\pfrac{}{z}
  \end{equation*}
  证明$\spa\{X,Y,Z\}$满足Frobenius条件,并找出积分子流形。
\end{xiti}
\begin{proof}
  \emph{Frobenius条件:当$X_1,\cdots,X_k$是$E$的一组基时,$[X_i,X_j]\in E$。}

  注意到$xX+yY=zZ$,故$\spa\{X,Y,Z\}=\spa\{X,Y\}$,而
  \begin{equation*}
    [X,Y]=XY-YX=-4y\pfrac{}{x}-2z\pfrac{}{y}=-2X
  \end{equation*}

  故$E$满足Frobenius条件。

  将$X,Y$扩充为一组基并取其对偶$1$形式$\omega^1,\omega^2, \omega^3=P\dd x+Q\dd y+R\dd z$。
  则
  \begin{equation*}
    \begin{cases}
      2yP+zQ=0\\
      -2xP+2zR=0
    \end{cases}
  \end{equation*}
  故
  \begin{equation*}
  \omega^3=z\dd x-2y\dd y+x\dd z=\dd(xz-y^2)
  \end{equation*}
  即积分子流形为
  \begin{equation*}
    xz-y^2=cst
  \end{equation*}
\end{proof}

\section{习题二}
\begin{xiti}
  在向量空间$V$中,$v_1,\cdots,v_k$和$\omega_1,\cdots,\omega_k$分别是子空间$V_1$和$V_2$的一组基,证明:$V_1=V_2$当且仅当存在非零实数$c$,使得
  \begin{equation*}
    v_1\wedge\cdots\wedge v_k=c\omega_1\wedge\cdots\wedge\omega_k
  \end{equation*}
\end{xiti}
\begin{proof}
  当$V_1=V_2$时,$\omega_i=a_i^jv_j$,故
  \begin{align*}
    \omega_1\wedge\cdots\wedge\omega_k &=a_1^{i_1}v_{i_1}\wedge a_2^{i_2}v_{i_2}\wedge\cdots\wedge a_k^{i_k}v_{i_k} \\
     &=cv_1\wedge\cdots\wedge v_k
  \end{align*}
  其中$c=\sum\sgn(i_1\cdots i_k)a_1^{i_1}\cdots a_k^{i_k}=\det(a_i^j)$,由于$(a_i^j)$是过度矩阵,故$c\neq0$。

  反之,由于
  \begin{equation*}
  \omega_i\wedge v_1\wedge\cdots\wedge v_k=\omega_i\wedge c\omega_1\wedge\cdots\wedge\omega_k=0
  \end{equation*}
  故$\omega_i\in V_1$,余同理,得$V_2\subset V_1$,反之亦然,故$V_1=V_2$。
\end{proof}

\begin{xiti}
  在$\Real^3$中定义$2$形式$\omega=x\dd y\wedge\dd z+y\dd z\wedge\dd x+z\dd x\wedge\dd y$。设$f\colon S^2\to\Real^3$是标准嵌入,求$f^{\ast}\omega$,并证明$f^{\ast}\omega$在$S^2$上处处非零。
\end{xiti}
\begin{proof}
  将$S^2$用极坐标表示,则
  \begin{equation*}
    f(\alpha,\beta)=(\cos\alpha\cos\beta,\cos\alpha\sin\beta,\sin\alpha)
  \end{equation*}

  故
  \begin{equation*}
    f^{\ast}\omega=-\cos\alpha\dd\alpha\wedge\dd\beta, \alpha\in(-\frac{\pi}{2},\frac{\pi}{2})
  \end{equation*}
\end{proof}

\begin{xiti}
  已知$\alpha,\beta$为闭微分式,证明$\alpha\wedge\beta$也是闭微分式;进一步地,若$\beta$还是恰当的,则$\alpha\wedge\beta$也是恰当的。
\end{xiti}
\begin{proof}
  因为$\dd\alpha=\dd\beta=0$,故
  \begin{equation*}
    \dd(\alpha\wedge\beta)=\dd\alpha\wedge\beta+(-1)^{\deg\alpha}\alpha\wedge\dd\beta=0
  \end{equation*}

  设$\beta=\dd\omega$,则
  \begin{equation*}
    \dd(\alpha\wedge\omega)=\dd\alpha\wedge\omega+(-1)^{\deg\alpha}\alpha\wedge\dd\omega=(-1)^{\deg\alpha}\alpha\wedge\beta
  \end{equation*}

  故$\alpha\wedge\beta=\dd((-1)^{\deg\alpha}\alpha\wedge\omega)$。
\end{proof}

\begin{xiti}
  在$S^3$上构造三个整体定义的$1$形式$\omega^1,\omega^2,\omega^3$,使得
  \begin{equation*}
    \dd\omega^1=-\omega^2\wedge\omega^3, \dd\omega^2=-\omega^3\wedge\omega^1, \dd\omega^3=-\omega^1\wedge\omega^2
  \end{equation*}
\end{xiti}
\begin{proof}
  首先求出
  \begin{equation*}
  S^3=SU(2)=\left\{\begin{pmatrix}
                     z_1 & z_2 \\
                     -\bar{z_2} & \bar{z_1} \\
                   \end{pmatrix}
                   \middle|
                   |z_1|^2+|z_2|^2=1
                   \right\}
  \end{equation*}
  的李代数。为此取$I$处的曲线
  \begin{equation*}
    \begin{cases}
      |z_1(t)|^2+|z_2(t)|^2=1\\
      z_1(0)=1\\
      z_2(0)=0
    \end{cases}
  \end{equation*}

  故求得李代数为
  \begin{equation*}
  \mathfrak{su}(2)=\left\{\begin{pmatrix}
                     ki & a+bi \\
                     -a+bi & -ki \\
                   \end{pmatrix}
                   \middle|
                   k^2+a^2+b^2=1
                   \right\}
  \end{equation*}

  其一组基为
  \begin{equation*}
    v_1=\begin{pmatrix}
      i &  \\
       & -i \\
    \end{pmatrix}
    , v_2=\begin{pmatrix}
       & 1 \\
      -1 &  \\
    \end{pmatrix}
    , v_3=\begin{pmatrix}
       & i \\
      i &  \\
    \end{pmatrix}
  \end{equation*}

  它们对应的右不变向量场为$e_1,e_2,e_3$,则
  \begin{align*}
    [e_1,e_2] & = [v_1,v_2]^{\wedge} \\
     & =(v_2v_1-v_1v_2)^{\wedge}\\
     &=(-2v_3)^{\wedge}=-2e_3
  \end{align*}

  同理$[e_2,e_3]=-2e_1,[e_3.e_1]=-2e_2$。于是令$e_1,e_2,e_3$对偶的$1$形式是$\omega^1,\omega^2,\omega^3$,则由
  \begin{equation*}
    \dd\omega^k=-\frac{1}{2}C^k_{ij}\omega^i\wedge\omega^j
  \end{equation*}
  得
  \begin{equation*}
    \dd\omega^1=2\omega^2\wedge\omega^3, \dd\omega^2=2\omega^3\wedge\omega^1, \dd\omega^3=2\omega^1\wedge\omega^2
  \end{equation*}

  令$\widetilde{\omega}^i=-2\omega^i$,则$\widetilde{\omega}^i$符合要求。
\end{proof}
\begin{proof}
  将$S^3$嵌入到$\Real^4$,并令
  \begin{equation*}
    \begin{pmatrix}
       \theta^0 \\
       \theta^1 \\
       \theta^2 \\
       \theta^3 \\
     \end{pmatrix}
     =
     \begin{pmatrix}
       x^1 & x^2 & x^3 & x^4 \\
       -x^2 & x^1 & -x^4 & x^3 \\
       -x^3 & x^4 & x^1 & -x^2 \\
       -x^4 & -x^3 & x^2 & x^1 \\
     \end{pmatrix}
     \begin{pmatrix}
       \dd x^1 \\
       \dd x^2 \\
       \dd x^3 \\
       \dd x^4 \\
     \end{pmatrix}
  \end{equation*}

  简记为$\theta=A\dd X$。从而有
  \begin{equation*}
    \dd\theta=\dd A\wedge\dd X=\dd A\wedge(A^{-1}\theta)=(\dd A)A^{-1}\wedge\theta
  \end{equation*}

  故
  \begin{equation*}
    f^{\ast}\dd\theta^1=2f^{\ast}\dd\theta^2\wedge f^{\ast}\dd\theta^3, f^{\ast}\dd\theta^2=2f^{\ast}\dd\theta^3\wedge f^{\ast}\dd\theta^1, f^{\ast}\dd\theta^3=2f^{\ast}\dd\theta^1\wedge f^{\ast}\dd\theta^2
  \end{equation*}
\end{proof}

\begin{xiti}
  在$\Real^5$中取坐标系$(x,y,u,p,q)$,并定义外微分式$\theta^1=\dd u-p\dd x-q\dd y, \theta^2=\dd p\wedge\dd q-\dd x\wedge\dd y$。对于二元函数$z(x,y)$,考虑由$f(x,y)=(x,y,z,z_x,z_y)$定义的浸入映射$f\colon\Real^2\to\Real^5$。证明:当且仅当$z$是方程$z_{xx}z_{yy}-z_{xy}^2=1$的解时,$f^{\ast}\theta^1=0, f^{\ast}\theta^2=0$。
\end{xiti}
\begin{proof}
  \begin{align*}
    f^{\ast}\theta^1&=\dd z-z_x\dd x-z_y\dd y \\
    f^{\ast}\theta^2&=\dd z_x\wedge\dd z_y-\dd x\wedge\dd y\\
                             &=(z_{xx}z_{yy}-z_{xy}^2-1)\dd x\wedge\dd y
  \end{align*}
\end{proof}

\begin{xiti}
  所设同上,定义$\omega^1=\dd u-p\dd x-q\dd y,\omega^2=\dd y\wedge\dd p-\dd x\wedge\dd q$。证明:当且仅当$z$是方程$z_{xx}+z_{yy}=0$的解时,$f^{\ast}\omega^1=0, f^{\ast}\omega^2=0$。
\end{xiti}
\begin{proof}
  \begin{align*}
    f^{\ast}\omega^1&=\dd z-z_x\dd x-z_y\dd y \\
    f^{\ast}\omega^2&=\dd y\wedge\dd z_x-\dd x\wedge\dd z_y\\
                               &=(z_{xx}+z_{yy})\dd y\wedge\dd x
  \end{align*}
\end{proof}

\begin{xiti}
  所设同上两题,考虑微分同胚$\phi\colon\Real^5\to\Real^5$如下
  \begin{equation*}
    \phi(x,y,u,p,q)=(x,q,u-qy,p,-y)
  \end{equation*}
  证明:$\phi^{\ast}\theta^1=\omega^1,\phi^{\ast}\theta^2=\omega^2$。这表明Monge-Ampere方程$z_{xx}z_{yy}-z_{xy}^2=1$与Laplace方程$z_{xx}+z_{yy}=0$的解之间有怎样的联系。
\end{xiti}
\begin{proof}
  \begin{align*}
    \phi^{\ast}\theta^1&=\dd (u-qy)-p\dd x-(-y)\dd q \\
                                  &=\dd u-y\dd q-q\dd y-p\dd x+y\dd q\\
                                  &=\dd u-p\dd x-q\dd y=\omega^1\\
    \phi^{\ast}\theta^2&=\dd p\wedge\dd (-y)-\dd x\wedge\dd q\\
                                  &=\dd y\wedge\dd p-\dd x\wedge\dd q=\omega^2
  \end{align*}

  可见Monge-Ampere方程$z_{xx}z_{yy}-z_{xy}^2=1$与Laplace方程$z_{xx}+z_{yy}=0$的解一一对应。
\end{proof}

\begin{xiti}
  设$\omega\in A^1(M)$处处不为零。若有处处非零的函数$\mu\in C^{\infty}(M)$使得$\mu\omega$为恰当的$1$形式,则称$\mu$为$\omega$的一个{\red 积分因子}\index{积分因子}。证明:$\omega$有积分因子的充要条件是$\dd\omega\wedge\omega=0$。
\end{xiti}
\begin{proof}
  设$\mu\omega=\dd f$,则
  \begin{align*}
    \omega &=\frac{1}{\mu}\dd f \\
    \dd\omega &=\dd(\frac{1}{\mu})\wedge\dd f\\
                      &=\dd(\frac{1}{\mu})\wedge(\mu\omega)\\
    \Longrightarrow&\dd\omega\wedge\omega=0
  \end{align*}

  反之,$\dd\omega\wedge\omega=0$说明存在光滑函数$\mu$,使得$\dd\omega=\mu\omega$。
\end{proof}

\begin{xiti}
  李群$G$的任意右不变向量场$X$都是{\red 完备}\index{完备向量场 }的,即它生成的流$\varphi_t=L_{\exp(tX)}$对任意$t\in\Real$都有定义(请特别注意:右不变向量场生成的流是左移动)。
\end{xiti}
\begin{proof}
  考虑$v\in T_1G$对应的单参数子群$\exp(tv)$。

  则$\varphi_t(a)=\exp(tv)a$相应的向量场为
  \begin{equation*}
    X_{(a)}=\localt{\dt\varphi_t(a)}=R_{a\ast}v
  \end{equation*}

  故满足$\dot{\gamma}(0)=X_1$的积分曲线$\gamma(t)$存在,设它的最大存在区间为$[0,\varepsilon)$,定义
   \begin{equation*}
     \widetilde{\gamma}=
     \begin{cases}
       \gamma(t) & t\in[0,\varepsilon)\\
       \gamma(t-\frac{\varepsilon}{2})\gamma(\frac{\varepsilon}{2}) & t\in[\varepsilon,\frac{3\varepsilon}{2})
     \end{cases}
   \end{equation*}
   则$\widetilde{\gamma}(t)$也是$X$的积分曲线:

   由
   \begin{equation*}
     \dot{\widetilde{\gamma}}(t)=R_{\gamma(\frac{\varepsilon}{2})\ast}\dot{\gamma}(t-\frac{\varepsilon}{2})=R_{\gamma(\frac{\varepsilon}{2})\ast}X(\gamma(t-\frac{\varepsilon}{2}))
   \end{equation*}
   及右不变性,得证。
\end{proof}

\begin{xiti}
  证明李群$G$的指数映射$\exp\colon T_1G\to G$是光滑映射。
\end{xiti}
\begin{proof}
  \emph{只须证明存在$T_1G\times G$上的向量场$Y$,它的流为$\phi_t(v,a)=(v,\exp(tv)a)$,从而$\exp$可写成光滑映射的复合$\pi_2(\phi_1(v,1))$。注意此时不能直接验证$\phi$是流,因为还未证明指数映射是李群同态。}

  注意到$\exp(tv)=\gamma(t)$,其中$\gamma(t)$是由$\dot{\gamma}(0)=v$决定的单参数子群,于是
  \begin{align*}
    \localt{\dt\phi_t(v,a)}&=\localt{\dt(v,\exp(tv)a)}\\
                                   &=(0,\localt{\dt\exp(tv)a})\\
                                   &=(0,R_{a\ast1}(\localt{\dt\gamma(t)}))\\
                                   &=(0,R_{a\ast1}v)\in T_v(T_1G)\times T_aG\cong T_{(v,a)}(T_1G\times G)
  \end{align*}

  故令$Y_{(v,a)}=(0,R_{a\ast1}v)$即可。
\end{proof}

\begin{xiti}
  证明指数映射满足等式$x\exp(v)x^{-1}=\exp(\Ad_xv), \forall v\in T_1G,x\in G$,并由此说明$\tr(\exp(v))\geqslant2, \forall v\in\mathfrak{sl}(2,\Real)$。
\end{xiti}
\begin{proof}[提示]
  考虑$v\in\mathfrak{sl}(2,\Real)$的相似标准型。
\end{proof}
\begin{proof}
  设$\gamma(t)=x\exp(tv)x^{-1}, \widetilde{\gamma}(t)=\exp(\Ad_x(tv))$,则它们都是单参数子群。由于
  \begin{align*}
     \dot{\gamma}(0)&=(L_{x}R_{x^{-1}})_{\ast1}v\\
     \dot{\widetilde{\gamma}}(0)&=\localt{\dt\exp((L_{x}R_{x^{-1}})_{\ast1}tv)}\\
                            &=(L_{x}R_{x^{-1}})_{\ast1}v
  \end{align*}

  故$\gamma(t)=\widetilde{\gamma}(t)$。

  \begin{align*}
    \mathfrak{sl}(2,\Real)&=\{v\in\Real^{2\times 2}\mid\tr(v)=0\}\\
    \tr(\exp(v))&=\tr(x\exp(v)x^{-1})\\
                    &=\tr(\exp(\Ad_xv))\\
                    &=\tr(\exp(xvx^{-1}))
  \end{align*}

  特别地,可选取$x$使得$xvx^{-1}$化为相似下的标准型
  \begin{align*}
    \tr(\exp\begin{pmatrix}
       \lambda &  \\
        & -\lambda \\
     \end{pmatrix})
     &=\tr(\begin{pmatrix}
        e^{\lambda} &  \\
         & e^{-\lambda} \\
      \end{pmatrix})
      =2\cosh\lambda\geqslant2\\
    \tr(\exp\begin{pmatrix}
       0 & \lambda \\
        & 0 \\
     \end{pmatrix})
     &=\tr(\begin{pmatrix}
        1 & \lambda \\
         & 1 \\
      \end{pmatrix})
      =1+1\geqslant2
  \end{align*}
\end{proof}

\begin{xiti}
  考虑$SL(2,\Complex)$中形如
  $\begin{pmatrix}
     x\sec t & y\tan t \\
     \bar{y}\tan t & \bar{x}\sec t \\
   \end{pmatrix}
  $的所有矩阵,其中$t\in(0,\frac{\pi}{2}), x,y\in\Complex$,且$|x|=|y|=1$。证明这些矩阵构成一个李群$G$,并求$T_1G$的一组基$e_1,e_2,e_3$。
\end{xiti}
\begin{proof}
  只须证$G$对乘法封闭:
  \begin{equation*}
   \begin{pmatrix}
     p & q \\
     \bar{q} & \bar{p} \\
   \end{pmatrix}
   \begin{pmatrix}
     a & b \\
     \bar{b} & \bar{a} \\
   \end{pmatrix}
   =\begin{pmatrix}
     pa+q\bar{b} & q\bar{a}+pb \\
     \bar{q}a+\bar{p}\bar{b} & \bar{p}\bar{a}+\bar{q}b \\
   \end{pmatrix}
  \end{equation*}

  令$x=y=1$,则$e_1=
  \begin{pmatrix}
     & 1 \\
    1 &  \\
  \end{pmatrix}
  $;

  $t=0$,则$e_2=
  \begin{pmatrix}
    i &  \\
     & -i \\
  \end{pmatrix}
  $;

  $x=1,t=\frac{\pi}{4}$,则$e_3=
  \begin{pmatrix}
     & i \\
    -i &  \\
  \end{pmatrix}
  $。
\end{proof}

\begin{xiti}
  设$D\defeq\{z\in\Complex\mid|z|<1\}$。考虑上题中的李群$G$在$D$上的左作用
  \begin{equation*}
    \begin{pmatrix}
     x\sec t & y\tan t \\
     \bar{y}\tan t & \bar{x}\sec t \\
    \end{pmatrix}.z\defeq
    \frac{zx\sec t+y\tan t}{z\bar{y}\tan t+\bar{x}\sec t}
  \end{equation*}
  求相应于$e_1,e_2,e_3$的基本向量场。
\end{xiti}
\begin{proof}
  设相应于$e_1,e_2,e_3$的基本向量场为$X, Y, Z$,则
  \begin{align*}
    X(z) &=\localt{\dt\exp(te_1).z} \\
     &=\localt{\dt\frac{z\cosh t+\sinh t}{z\sinh t+\cosh t}}\\
     &=\localt{\frac{(1-z^2)e^t}{(z\sinh t+\cosh t)^2}}\\
     &=1-z^2\\
    Y(z) &=\localt{\dt\exp(te_2).z} \\
     &=\localt{\dt\frac{ze^{it}}{e^{-it}}}\\
     &=\localt{-2z\sin 2t+2zi\cos 2t}\\
     &=2zi\\
    Z(z) &=\localt{\dt\exp(te_3).z} \\
     &=\localt{\dt\frac{z\cos t+i\sin t}{-zi\sin t+\cos t}}\\
     &=\localt{\frac{i(1-z^2)(\cos t^2-\sin t^2)}{(-zi\sin t+\cos t)^2}}\\
     &=i(1-z^2)
  \end{align*}
\end{proof}

\begin{rem}
  计算$\exp\begin{pmatrix}
            & z \\
           z &  \\
         \end{pmatrix}$。
  \begin{align*}
    \sum_{k=0}^{\infty}\frac{z^{2k}}{(2k)!}&=\cosh z=\frac{e^z+e^{-z}}{2} \\
    \sum_{k=0}^{\infty}\frac{z^{2k+1}}{(2k+1)!}&=\sinh z=\frac{e^z-e^{-z}}{2}\\
    \exp\begin{pmatrix}
            & z \\
           z &  \\
         \end{pmatrix}
         &=\begin{pmatrix}
            \cosh z & \sinh z \\
            \sinh z & \cosh z \\
          \end{pmatrix}
  \end{align*}
\end{rem}

\newpage
\phantomsection
\addcontentsline{toc}{section}{Index}
\printindex
\end{document}
